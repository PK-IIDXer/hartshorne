\chapter{可換環論:中級}\label{chapter-commutative-rings-second-course}

第\ref{chapter-commutative_rings_introduction}章では環とイデアルの基礎を、第\ref{chapter-commutative_rings_first_course}章ではNoether性という有限性の概念を学びました。
これくらいで一応、可換環論初級は卒業でしょうか。
初級って言ったって、僕が言ってるだけですけど。
ともかくこれで、\mycite{Har77}の2ページ目までの可換環論は攻略しました。

意気揚々と読み進めていくと、3ページ目Proposition 1.2の証明がほとんど書かれてないです。
大半は簡単なのですが、Prop 1.2(d)は\textbf{Hilbertの零点定理}と呼ばれるまあまあの大定理です。
これも可換環論の結果扱いということで、参考文献にぶん投げられています。
取り上げられているのはLangのでっかい本、我らがアティマク、古典のZariski-Samuelが挙げられています。
んで、例えば君が\mycite{At-Mc-translate}しか持ってなかったら残念!
演習問題です。
代数幾何あるあるですね。

今回はHilbertの零点定理を攻略していきましょう。



\section{環の局所化}

有理数体$\mathbb{Q}$は、整数環$\mathbb{Z}$から「$0$以外の元での割り算」を許すことで作られました。
一般の環でも、特定の元での割り算を許容する操作を考えることができます。

\subsection{積閉集合と分数環の定義}

割り算を許容する環を新たに作るために、分母に``なることができる"集合を一般的に考えます。
\begin{definition}[積閉集合]
  環$A$の部分集合$S$が\textbf{積閉集合}であるとは、以下を満たすときを言う。
  \begin{itemize}
    \item $1 \in S$
    \item $x, y \in S \implies xy \in S$
  \end{itemize}
\end{definition}

環$A$の積閉集合$S$に対して、$S$を分母にすることができる環を
\[
  B=\left\{\frac{a}{s}\mid a\in A, s\in S\right\}
\]
で定義できそうです。
しかしこれでは「約分」が考慮されておらず、同一視されるべき元が別々に見えてしまっています($1/2\neq2/4$みたいに見えている)。
分数が約分できる状況というのは、
\[
  \frac{a}{s}=\frac{b}{t} \iff at=bs \iff at-bs=0
\]
です。これを$B$の同値関係として導入すべきでしょうか?
答えはNO。これでは同値関係のうち、推移律を満たしません。
実際、$a/s=b/t, b/t=c/u$とすると、$at-bs=0$に$u$を、$bu-ct=0$に$s$を掛けて足し合わせることで、
\[
  0=(atu-bsu)+(bsu-cst)=atu-cst=t(au-cs)
\]
となって、$au-cs=0$とは限らないことが分かります。
なぜなら、これがゼロ因子かもしれないからです。

そこで以下の同値関係で割った集合を考えるべきでしょう。
\begin{theorem}
  $A$を環、$S$をその積閉集合とする。
  このとき、$A\times S$上の以下の関係は、同値関係である。
  \[
    (a,s) \sim (b,t) \iff {}^\exists u \in S \text{ s.t. } u(at-bs)=0
  \]
\end{theorem}
\begin{proof}
  \begin{description}
    \item[\textbf{反射律}] $(a,s)\in A\times S$に対して、$1\cdot(as-as)=0$なので$(a,s)\sim(a,s)$
    \item[\textbf{対称律}] $(a,s)\sim(b,t)$すなわち${}^\exists u\in S$ s.t. $u(at-bs)=0$とすると、$u(bs-at)=-u(at-bs)=0$ゆえに$(b,t)\sim(a,s)$
    \item[\textbf{推移律}] $(a_1,s_1)\sim(a_2,s_2)$かつ$(a_2,s_2)\sim(a_3,s_3)$すなわち${}^\exists t,u\in S$ s.t. $t(a_1s_2-a_2s_1)=0$、$u(a_2s_3-a_3s_2)=0$とする。
    このとき、
    \[
      0=s_3u\{t(a_1s_2-a_2s_1)\}+s_1t\{u(a_2s_3-a_3s_2)\}=tus_2a_1s_3-tus_2a_3s_1=tus_2(a_1s_3-a_3s_1)
    \]
    であり、$S$は積閉集合だから$tus_2\in S$より$(a_1,s_1)\sim(a_3,s_3)$
  \end{description}
\end{proof}

ゆえに集合として、$S$を分母として許容する新たな環が定義できました。
\begin{definition}[分数環]
  $A$を環、$S$をその積閉集合とする。
  このとき、$A\times S$上の同値関係
  \[
    (a,s) \sim (b,t) \iff {}^\exists u \in S \text{ s.t. } u(at-bs)=0
  \]
  に関する商集合
  \[
    S^{-1}A:=A\times S/\sim
  \]
  を、積閉集合$S$による$A$の\textbf{分数環}という。
  $S^{-1}A$の元を
  \[
    \frac{a}{s}=a/s
  \]
  などと書き表す。
\end{definition}

分数環は本来、自然に導入される環構造を含めて分数環と呼びます。
\begin{theorem}
  $A$を環、$S$をその積閉集合とする。
  このとき、$S^{-1}A$に以下の演算
  \begin{align*}
    \frac{a}{s}+\frac{b}{t}&:=\frac{at+bs}{st}\\
    \frac{a}{s}\cdot\frac{b}{t}&:=\frac{ab}{st}
  \end{align*}
  を入れることによって、$S^{-1}A$は環である。
  加法単位元は$0/1$、乗法単位元は$1/1$である。
\end{theorem}
\begin{proof}
  \begin{itemize}
    \item \textbf{演算$+$に関して可換群} 
    \begin{itemize}
      \item \textbf{結合法則} $a_1/s_1+(a_2/s_2+a_3/s_3)=a_1/s_1+(a_2s_3+a_3s_2)/s_2s_3=\{a_1s_2s_3+(a_2s_3+a_3s_2)s_1\}/s_1s_2s_3=\{(a_1s_2+a_2s_1)s_3+a_3s_1s_2\}/s_1s_2s_3=(a_1s_2+a_2s_1)/s_1s_2+a_3/s_3=(a_1/s_1+a_2/s_2)+a_3/s_3$
      \item \textbf{単位元} $a/s+0/1=(a\cdot1+0\cdot s)/s\cdot1=a/s$、$0/1+a/s=(0\cdot s+a\cdot1)/1\cdot s=a/s$
      \item \textbf{逆元} $a/s+(-a/s)=(a\cdot s-a\cdot s)/s^2=0/s^2=0/1$ ($\because$ $1\cdot(0\cdot1-0\cdot s^2)=0$)
      $(-a/s)+a/s=0/1$も同様。
      \item \textbf{可換性} $a/s+b/t=(at+bs)/st=(bs+at)/ts=b/t+a/s$
    \end{itemize}
    \item \textbf{演算$\cdot$に関して可換モノイド}
    \begin{itemize}
      \item \textbf{結合法則} $a_1/s_1\cdot (a_2/s_2\cdot a_3/s_3)=a_1/s_1\cdot(a_2a_3/s_2s_3)=a_1(a_2a_3)/s_1(s_2s_3)=(a_1a_2)a_3/(s_1s_2)s_3=(a_1a_2/s_1s_2)\cdot a_3/s_3=(a_1/s_1\cdot a_2/s_2)\cdot a_3/s_3$
      \item \textbf{単位元} $a/s\cdot 1/1=a\cdot1/s\cdot1=a/s$、$1/1\cdot a/s=1\cdot a/1\cdot s=a/s$
      \item \textbf{可換性} $a/s\cdot b/t=ab/st=ba/ts=b/t\cdot a/s$
    \end{itemize}
    \item \textbf{分配法則} $a_1/s_1\cdot(a_2/s_2+a_3/s_3)=a_1/s_1\cdot(a_2s_3+a_3s_2)/s_2s_3=a_1(a_2s_3+a_3s_2)/s_1s_2s_3=(a_1a_2s_3+a_1a_3s_2)/s_1s_2s_3=a_1a_2s_3/s_1s_2s_3+a_1a_3s_2/s_1s_2s_3=a_1a_2/s_1s_2+a_1a_3/s_1s_3=(a_1/s_1\cdot a_2/s_2)+(a_1/s_1\cdot a_3/s_3)$
  \end{itemize}
\end{proof}

さて、僕たちは積閉集合$S$にゼロ因子どころか、0が入ることを許容しています。
もし0が入っていると、分数環$S^{-1}A$は$1=0$の零環に潰れてしまいます。
実際、任意の$a/s\in S^{-1}A$に対して、$0\in S$が存在「してしまい」、
\[
  0\cdot(a\cdot 1-0\cdot s)=0
\]
をみたすので、$a/s=0/1$です。

僕たちは零環、すなわち$1=0$となる環を認めているので、このことに特に問題はないです。
一方$1\neq0$を満たすもののみを環とする主義では、積閉集合の定義に$0\neq S$を追加する必要があります。


\subsection{自然な準同型}

剰余環のときと同様に、局所化の元となった環$A$からの、「自然」と呼ぶことができる準同型が存在します。
\begin{theorem}
  $A$を環、$S$をその積閉集合とする。
  このとき、写像
  \[
    \varphi_S:A\to S^{-1}A;a\mapsto a/1
  \]
  は準同型である。
\end{theorem}
これはもはや自明と言ってよいでしょう。
それよりも、この準同型の性質を見たいです。
もし単射であれば、$A$は$S^{-1}A$の部分環とみなすことができ、$\mathbb{Z}\subset\mathbb{Q}$の直感とも合います。
しかしゼロ因子の存在により、状況は少し複雑になっています。
\begin{theorem}
  $A$を環、$S$をその積閉集合とする。
  このとき、自然な準同型
  \[
    \varphi_S:A\to S^{-1}A;a\mapsto a/1
  \]
  のカーネルは
  \[
    \operatorname{Ker}(\varphi_S)=\{a\in A\mid {}^\exists s\in S \text{ s.t. } as=0\}
  \]
  である。
\end{theorem}
\begin{proof}
  \begin{align*}
    a\in\operatorname{Ker}(\varphi_S)&\iff\varphi(a)=\frac{a}{1}=\frac01\\
    &\iff{}^\exists s \text{ s.t. } s(a\cdot1-0\cdot1)=0\\
    &\iff{}^\exists s \text{ s.t. } as=0
  \end{align*}
\end{proof}

従って、$S$がゼロ因子をもっていると単射ではないのです。
なのでそもそも$A$がゼロ因子を持たない場合、つまり整域の場合、どんな積閉集合$S$をとっても自然に$A\subset S^{-1}A$とみなすことができます。

\subsection{$\mathbb{Z}$から$\mathbb{Q}$、あるいは整域の商体}

$\mathbb{Z}$の部分集合
\[
  S:=\mathbb{Z}\setminus\{0\}
\]
は明らかに積閉集合です。
この分数環$S^{-1}\mathbb{Z}$が有理数体になっているというのが、小学校から学んできた普通の事実です。
\[
  S^{-1}\mathbb{Z}=\mathbb{Q}
\]

一般の整域$A$に対しても、同様のことが言えます。
\begin{theorem}
  $A$を整域、$S:=A\setminus\{0\}$とする。
  このとき、$S$は積閉集合である。
\end{theorem}
\begin{proof}
  もし$s,t\in S$かつ$st=0$を満たすものがあるとすると、$A$は整域であったから、$s=0$または$t=0$を満たす。
  ところがこれは、$s,t\in S=A\setminus\{0\}$であったことに矛盾する。
\end{proof}

また、$A$が整域で$S:=A\setminus\{0\}$のとき、
\[
  K:=S^{-1}A
\]
は明らかに体です(急に明らかと言われてびっくりするかもですが、手を動かし始めると本当に示すべきことがないです)。
このようにして得られる体には名前がついています。
\begin{definition}
  $A$を整域、$S:=A\setminus\{0\}$とする。
  このとき、体
  \[
    S^{-1}A
  \]
  を、整域$A$の\textbf{商体}という。
\end{definition}

\begin{comment}

\subsection{重要な例:$A_f$と$A_\mathfrak{p}$}
素イデアル$\mathfrak{p}$に対して、$S=A\setminus\mathfrak{p}$が積閉集合になることは極めて重要です。
(素イデアルの定義 $xy \in \mathfrak{p} \implies x \in \mathfrak{p} \lor y \in \mathfrak{p}$ の対偶ですね)

\section{イデアルの根基}

次章の零点定理で「イデアルと図形の対応」を考える際、邪魔になるのが冪零元です。
例えば$\mathbb{C}[x]$において、イデアル$(x)$と$(x^2)$は代数的には異なりますが、幾何学的な零点集合(つまり$x=0$となる点と$x^2=0$となる点)はどちらも原点$0$であり、区別がつきません。
この違いを吸収するのが根基です。

\begin{definition}[根基]
  環$A$のイデアル$\mathfrak{a}$に対して、
  \[
    \sqrt{\mathfrak{a}} := \{x \in A \mid \exists n > 0, x^n \in \mathfrak{a}\}
  \]
  を$\mathfrak{a}$の\textbf{根基}(\textit{radical})と呼ぶ。
  $\mathfrak{a} = \sqrt{\mathfrak{a}}$ を満たすイデアルを\textbf{根基イデアル}と呼ぶ。
\end{definition}

($\sqrt{\mathfrak{a}}$がイデアルになる証明、素イデアルの共通部分としての性質 $\sqrt{\mathfrak{a}} = \bigcap_{\mathfrak{a}\subset\mathfrak{p}} \mathfrak{p}$ などを記述)

\section{有限生成$A$-代数}

(環準同型による代数の定義。有限生成代数と有限生成加群の違いを強調)

\section{ヒルベルトの零点定理}

準備は整いました。ここから舞台を代数的閉体$K$上の多項式環$A = K[x_1, \dots, x_n]$に移します。

\subsection{代数的集合とイデアル}

($V(\mathfrak{a})$ と $I(Z)$ の定義)

ここで素朴な疑問が生じます。
イデアル$\mathfrak{a}$から出発して、その零点集合$V(\mathfrak{a})$を考え、さらにその消失イデアル$I(V(\mathfrak{a}))$を考えたとき、元の$\mathfrak{a}$に戻るでしょうか?
答えはNoです。先ほど見たように、$\mathfrak{a}=(x^2)$なら$I(V(\mathfrak{a}))=(x)$になってしまいます。
では、$\sqrt{\mathfrak{a}}$をとれば一致するのか?
その答えが、次の定理です。

\subsection{弱零点定理}

まずは「点」に対応する極大イデアルの形を決定します。

\begin{theorem}[Hilbert's Nullstellensatz, weak form]
  $K$を代数的閉体とする。$K[x_1, \dots, x_n]$の任意の極大イデアル$\mathfrak{m}$は、
  \[
    \mathfrak{m} = (x_1 - a_1, \dots, x_n - a_n), \quad (a_1, \dots, a_n \in K)
  \]
  の形に書ける。
\end{theorem}

この証明には、次の代数的な補題が強力な役割を果たします。
(ザリスキの補題を紹介)

\subsection{強零点定理とRabinowitschのトリック}

変数を一つ増やすことで、魔法のように弱形式から一般系を導くことができます。これをRabinowitschのトリックと呼びます。

\begin{theorem}[Hilbert's Nullstellensatz, strong form]
  $K$を代数的閉体、$A=K[x_1, \dots, x_n]$とする。
  任意のイデアル$\mathfrak{a} \subset A$に対して、
  \[
    I(V(\mathfrak{a})) = \sqrt{\mathfrak{a}}
  \]
\end{theorem}
\end{comment}
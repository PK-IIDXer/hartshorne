\chapter{可換環論:中級}\label{chapter-commutative-rings-second-course}

第\ref{chapter-commutative_rings_introduction}章では環とイデアルの基礎を、第\ref{chapter-commutative_rings_first_course}章ではNoether性という有限性の概念を学びました。
これくらいで一応、可換環論初級は卒業でしょうか。
初級って言ったって、僕が言ってるだけですけど。
ともかくこれで、\mycite{Har77}の2ページ目までの可換環論は攻略しました。

意気揚々と読み進めていくと、3ページ目Proposition 1.2の証明がほとんど書かれてないです。
大半は簡単なのですが、Prop 1.2(d)は\textbf{Hilbertの零点定理}と呼ばれるまあまあの大定理です。
これも可換環論の結果扱いということで、参考文献にぶん投げられています。
取り上げられているのはLangのでっかい本、我らがアティマク、古典のZariski-Samuelが挙げられています。
んで、例えば君が\mycite{At-Mc-translate}しか持ってなかったら残念!
演習問題です。
代数幾何あるあるですね。

今回はHilbertの零点定理を攻略していきましょう。



\section{イデアルの根基}

まず突然現れている「$\sqrt{\mathfrak{a}}$」ってなんやねんという話ですが、これはイデアル$\mathfrak{a}$の\textbf{根基}と呼ばれるもので、イデアルの元の重複具合を可能な限り取り除いたものと言えましょう。
例えば$\mathbb{C}[x]$において、イデアル$(x)$と$(x^2)$は代数的には異なりますが、幾何学的な零点集合(つまり$x=0$となる点と$x^2=0$となる点)はどちらも原点$0$であり、区別がつきません。
この違いを吸収するのが根基です。
\begin{definition}[根基]
  環$A$のイデアル$\mathfrak{a}$に対して、
  \[
    \sqrt{\mathfrak{a}} := \{x \in A \mid {}^\exists n > 0, x^n \in \mathfrak{a}\}
  \]
  を$\mathfrak{a}$の\textbf{根基}(\textit{radical})と呼ぶ。
\end{definition}

根基が実際にイデアルになっていることを証明しましょう。
\begin{theorem}
  環$A$のイデアル$\mathfrak{a}$に対して、
  \[
    \sqrt{\mathfrak{a}} := \{x \in A \mid {}^\exists n > 0, x^n \in \mathfrak{a}\}
  \]
  は$A$のイデアルである。
\end{theorem}
\begin{proof}
  まず、$0^1=0\in\mathfrak{a}$ゆえに$0\in\sqrt{\mathfrak{a}}$。

  次に$x,y\in\sqrt{\mathfrak{a}}$とおくと、ある$n,m>0$が存在して、$x^n,y^m\in\mathfrak{a}$。
  このとき、
  \[
    (x+y)^{n+m}=\sum_{i=0}^{n+m}{}_{n+m}C_ix^{n+m-i}y^i
  \]
  を考えると、各$x^{n+m-i}y^i$は$\mathfrak{a}$に含まれる。
  実際、$i\leq m$ならば$x^{n+m-i}y^i=(x^{m-i}y^i)x^n\in\mathfrak{a}$。
  また$i>m$ならば$x^{n+m-i}y^i=(x^{n+m-i}y^{i-m})y^m\in\mathfrak{a}$。

  最後に、$a\in A$、$x\in\sqrt{\mathfrak{a}}$とすると、ある$n>0$が存在して、$x^n\in\mathfrak{a}$。
  このとき、
  \[
    (ax)^n=a^nx^n\in\mathfrak{a}
  \]
  ゆえに$ax\in\sqrt{\mathfrak{a}}$。
\end{proof}

\subsection{根基の例}

簡単な例をいくつか見ていきましょう。
まずは一般的にすぐに分かる例を挙げます。
\begin{example}
  $A$を環、$\mathfrak{a}$をそのイデアルとすると、$\sqrt{\mathfrak{a}}=A \iff \mathfrak{a}=A$
\end{example}
\begin{proof}
  明らかに$\sqrt{A}=A$ゆえ、$\sqrt{\mathfrak{a}}=A$ならば$\mathfrak{a}=A$を示す。
  仮定より$1\in\sqrt{\mathfrak{a}}$ゆえ、$1=1^n\in\mathfrak{a}$ (${}^\exists n>0$)。
  よって$1\in\mathfrak{a}$となり$\mathfrak{a}=A$が従う。
\end{proof}

より具体的な例を見ていきましょう。
PID $\mathbb{Z}$において、根基は「$\sqrt{\mathfrak{a}}$」という記号のイメージ通りの挙動をします。
\begin{example}
  $\mathbb{Z}$において、
  \begin{itemize}
    \item $\sqrt{(4)}=(2)$
    \item $\sqrt{(9)}=(3)$
    \item $\sqrt{(12)}=(6)$
    \item $\sqrt{(125)}=(5)$
  \end{itemize}
  など。一般に、$a=p_1^{e_1}\cdots p_n^{e_n}$のとき、
  \[
    \sqrt{(p_1^{e_1}\cdots p_n^{e_n})}=(p_1\cdots p_n)
  \]
\end{example}

証明はいらないでしょう。
$\sqrt{(125)}=(5)$のように、単に平方根をとっているのではなく、とにかく重複した素因子のべき乗を排除するのです。

\subsection{根基の性質}

根基の性質で最初に伝えておくべき重要な性質は、次の簡単な定理です。
\begin{theorem}
  環$A$のイデアル$\mathfrak{a}$に対して、
  \[
    \mathfrak{a}\subset\sqrt{\mathfrak{a}}
  \]
\end{theorem}
\begin{proof}
  根基の定義から、$x\in\mathfrak{a}$ならば$x^1=x\in\mathfrak{a}$ゆえに$x\in\sqrt{\mathfrak{a}}$、つまり$\mathfrak{a}\subset\sqrt{\mathfrak{a}}$。
\end{proof}

剰余環の自然な準同型
\[
  \varphi:A\to A/\mathfrak{a};a\mapsto a+\mathfrak{a}
\]
は、「$\mathfrak{a}$を含むイデアルとそれらの包含関係のなす順序集合」と、「$A/\mathfrak{a}$のイデアルとそれらの包含関係のなす順序集合」の間の順序同型でした。
従って$\mathfrak{a}\subset\sqrt{\mathfrak{a}}$から、$\sqrt{\mathfrak{a}}$に対応する$A/\mathfrak{a}$に対応するイデアル
\[
  \mathfrak{N}_{A/\mathfrak{a}}:=\varphi(\sqrt{\mathfrak{a}})
\]
が存在します。
これはほとんど明らかに以下の性質を持ちます。
\begin{theorem}
  $A$を環、$\mathfrak{a}$をイデアルとする。
  $\varphi:A\to A/\mathfrak{a}$を剰余環の自然な準同型とし、$\mathfrak{N}_{A/\mathfrak{a}}:=\varphi(\sqrt{\mathfrak{a}})$とおく。
  このとき、
  \[
    a+\mathfrak{a}\in\mathfrak{N}_{A/\mathfrak{a}} \iff (a+\mathfrak{a})^n=0\; ({}^\exists n>0)
  \]
\end{theorem}
\begin{proof}
  \begin{align*}
    a+\mathfrak{a}\in\mathfrak{N}_{A/\mathfrak{a}} &\iff \varphi(a)\in\varphi(\sqrt{\mathfrak{a}})\\
    &\iff a\in\sqrt{\mathfrak{a}}\\
    &\iff a^n\in\mathfrak{a} ({}^\exists n>0)\\
    &\iff a^n+\mathfrak{a}=0\in A/\mathfrak{a} ({}^\exists n>0)\\
    &\iff (a+\mathfrak{a})^n=0 ({}^\exists n>0)
  \end{align*}
\end{proof}

一般に、
\begin{definition}
  $A$を環とする。
  $a\in A$が\textbf{冪零元}であるとは、ある$n>0$が存在して、$a^n=0$を満たす時をいう。
\end{definition}
\begin{definition}
  $A$を環とする。
  このとき、$A$の冪零元すべての集合がなすイデアル
  \[
    \mathfrak{N}_A:=\sqrt{(0)}=\{a\in A\mid a^n=0\; ({}^\exists n>0)\}
  \]
  を\textbf{冪零元根基}という。
\end{definition}
ので、「自然な準同型$\varphi:A\to A/\mathfrak{a}$による$\sqrt{\mathfrak{a}}$の像は冪零元根基$\mathfrak{N}_{A/\mathfrak{a}}$」ということです。

\subsection{Hilbertの零点定理}

根基の定義だけ把握していれば、Hilbertの零点定理の主張を理解することはできます。
\begin{theorem}
  $k$を代数閉体とし、$A:=k[X_1,\dots,X_n]$を多項式環とする。
  \begin{description}
    \item[弱零点定理(Hilbert's Nullstellensatz, weak form)] $A$の極大イデアルは、必ずある$(a_1,\dots,a_n)\in k^n$に対して、$(X_1-a_1,\dots,X_n-a_n)$という形である。
    \item[強零点定理(Hilbert's Nullstellensatz, strong form)] $\mathfrak{a}$を$A$のイデアル、$Y$を$k^n$の部分集合とするとき、
    \begin{align*}
      Z(\mathfrak{a})&:=\{P\in k^n\mid \forall f\in\mathfrak{a}, f(P)=0\}\\
      I(Y)&:=\{f\in A\mid \forall P\in Y, f(P)=0\}
    \end{align*}
    とする。このとき$I(Y)$は$A$のイデアルとなり、任意のイデアル$\mathfrak{a}\subset A$に対して$I(Z(\mathfrak{a}))=\sqrt{\mathfrak{a}}$となる。
  \end{description}
\end{theorem}



\section{Hilbertの零点定理の証明}

\begin{theorem}[Artin-Tateの補題]
  $A\subset B\subset C$を環の列であり、以下を満たすとする。
  \begin{enumerate}
    \item $A$はNoether環
    \item $C$は有限生成$A$代数
    \item $C$は有限生成$B$代数
  \end{enumerate}
  このとき、$B$は有限生成$A$代数である。
\end{theorem}
\begin{proof}
  TODO: AM prop7.8
\end{proof}

\begin{theorem}[Zariskiの補題]
  $k$を体、体$E$は有限生成$k$代数であるとする。
  このとき、$E$は$k$の有限次代数拡大である。
\end{theorem}
\begin{proof}
  $E$の生成元を$x_1,\dots,x_n\in E$として、背理法で証明する。
  もし$E$が$k$上代数的でないとすると、ある$r>0$が存在して、
  \begin{enumerate}
    \item $x_1,\dots,x_r$は$k$上一次独立
    \item $x_{r+1},\dots,x_n$は$F:=k(x_1,\dots,x_r)$上代数的
  \end{enumerate}
  となるように$x_i$の添え字を入れ替えることができる。
  体拡大$E/F$で考えると、これは有限次代数拡大になるから、$E$は$F$加群として有限生成である\footnote{
    TODO わからん。有限次「代数」拡大でなくても、有限次拡大というだけで有限$F$加群な気がする。
  }。
  体(環)の拡大列
  \[
    k\subset F\subset E
  \]
  に対して、$k$は体ゆえにNoether環かつ、$E$は有限生成$k$代数であり、また$E/F$も有限生成$F$加群であるから、$F$も有限生成$k$代数である\footnote{
    TODO: AM prop7.8の内容。重そう。
  }。
  従って$F$は、$F=k[y_1,\dots,y_s]$と表わされる。
  各$y_j\in F=k(x_1,\dots,x_r)$は
  \[
    y_j:=\frac{f_j(x_1,\dots,x_r)}{g_j(x_1,\dots,x_r)}
  \]
  と書ける。

  多項式環$k[x_1,\dots,x_r]$には、既約多項式が無限個存在するため\footnote{
    TODO: 証明する
  }、$g_1,\dots,g_s$すべてと互いに素な多項式$h\in k[x_1,\dots,x_r]$が存在する。
  このとき、$1/h\in F$であるが、$h$の取り方からこれは$y_j=f_j/g_j$の多項式で表わすことができない。
  これは$F=k[y_1,\dots,y_s]$と表わせていたことに矛盾する。
  よって$E$は$k$上代数的でなければならず、ゆえに$E/k$は有限次代数拡大である。
\end{proof}

\begin{theorem}[弱零点定理(Hilbert's Nullstellensatz, weak form)]
  $k$を代数閉体とし、$A:=k[X_1,\dots,X_n]$を多項式環とする。
  このとき、$A$の極大イデアルは、必ずある$(a_1,\dots,a_n)\in k^n$に対して、$(X_1-a_1,\dots,X_n-a_n)$という形である。
\end{theorem}
\begin{proof}
  $\mathfrak{m}$を$A$の極大イデアルとする。
  このとき$E:=A/\mathfrak{m}$とおくと、これは体であり、$X_i+\mathfrak{m}$で生成されるから有限生成$k$代数である。
  従って$E$は$k$の有限次代数拡大である。
  代数閉体の有限次代数拡大は$k$と同型なものしかないから\footnote{
    TODO: 証明する。
  }、
  \[
    A/\mathfrak{m}\cong k
  \]
  が従う。この同型写像を$\varphi:A/\mathfrak{m}\to k$とおく。

  さらに$a_i:=\varphi(X_i+\mathfrak{m})$とおくとき、
  \[
    \mathfrak{m}=(X_1-a_1,\dots,X_n-a_n)
  \]
  が従うことを示そう。
  $\pi:A\to A/\mathfrak{m}$を自然な準同型とし、$F:=\varphi\circ\pi:A\to A/\mathfrak{m}\to k$とおく。
  このとき$\mathfrak{m}$の任意の元$f$は$F(f)=0$を満たすから$\mathfrak{m}\subset\operatorname{Ker}(F)$である。
  さらに$\mathfrak{m}$は極大イデアルであるから、結局
  \[
    \operatorname{Ker}(F)=\mathfrak{m}
  \]
  が成り立つ。
  また、$a_i$の取り方から$F(X_i)=a_i$すなわち$F(X_i-a_i)=0$であるから
  \[
    (X_1-a_1,\dots,X_n-a_n)\subset\operatorname{Ker}(F)=\mathfrak{m}
  \]
  である。
  $(X_1-a_1,\dots,X_n-a_n)$は明らかに極大イデアルであるから、従って$\mathfrak{m}=(X_1-a_1,\dots,X_n-a_n)$
\end{proof}

\begin{comment}



\section{分数環}

有理数体$\mathbb{Q}$は、整数環$\mathbb{Z}$から「$0$以外の元での割り算」を許すことで作られました。
一般の環でも、特定の元での割り算を許容する操作を考えることができます。

\subsection{積閉集合と分数環の定義}

割り算を許容する環を新たに作るために、分母に``なることができる"集合を一般的に考えます。
\begin{definition}[積閉集合]
  環$A$の部分集合$S$が\textbf{積閉集合}であるとは、以下を満たすときを言う。
  \begin{itemize}
    \item $1 \in S$
    \item $x, y \in S \implies xy \in S$
  \end{itemize}
\end{definition}

環$A$の積閉集合$S$に対して、$S$を分母にすることができる環を
\[
  B=\left\{\frac{a}{s}\mid a\in A, s\in S\right\}
\]
で定義できそうです。
しかしこれでは「約分」が考慮されておらず、同一視されるべき元が別々に見えてしまっています($1/2\neq2/4$みたいに見えている)。
分数が約分できる状況というのは、
\[
  \frac{a}{s}=\frac{b}{t} \iff at=bs \iff at-bs=0
\]
です。これを$B$の同値関係として導入すべきでしょうか?
答えはNO。これでは同値関係のうち、推移律を満たしません。
実際、$a/s=b/t, b/t=c/u$とすると、$at-bs=0$に$u$を、$bu-ct=0$に$s$を掛けて足し合わせることで、
\[
  0=(atu-bsu)+(bsu-cst)=atu-cst=t(au-cs)
\]
となって、$au-cs=0$とは限らないことが分かります。
なぜなら、これがゼロ因子かもしれないからです。

そこで以下の同値関係で割った集合を考えるべきでしょう。
\begin{theorem}
  $A$を環、$S$をその積閉集合とする。
  このとき、$A\times S$上の以下の関係は、同値関係である。
  \[
    (a,s) \sim (b,t) \iff {}^\exists u \in S \text{ s.t. } u(at-bs)=0
  \]
\end{theorem}
\begin{proof}
  \begin{description}
    \item[\textbf{反射律}] $(a,s)\in A\times S$に対して、$1\cdot(as-as)=0$なので$(a,s)\sim(a,s)$
    \item[\textbf{対称律}] $(a,s)\sim(b,t)$すなわち${}^\exists u\in S$ s.t. $u(at-bs)=0$とすると、$u(bs-at)=-u(at-bs)=0$ゆえに$(b,t)\sim(a,s)$
    \item[\textbf{推移律}] $(a_1,s_1)\sim(a_2,s_2)$かつ$(a_2,s_2)\sim(a_3,s_3)$すなわち${}^\exists t,u\in S$ s.t. $t(a_1s_2-a_2s_1)=0$、$u(a_2s_3-a_3s_2)=0$とする。
    このとき、
    \[
      0=s_3u\{t(a_1s_2-a_2s_1)\}+s_1t\{u(a_2s_3-a_3s_2)\}=tus_2a_1s_3-tus_2a_3s_1=tus_2(a_1s_3-a_3s_1)
    \]
    であり、$S$は積閉集合だから$tus_2\in S$より$(a_1,s_1)\sim(a_3,s_3)$
  \end{description}
\end{proof}

ゆえに集合として、$S$を分母として許容する新たな環が定義できました。
\begin{definition}[分数環]
  $A$を環、$S$をその積閉集合とする。
  このとき、$A\times S$上の同値関係
  \[
    (a,s) \sim (b,t) \iff {}^\exists u \in S \text{ s.t. } u(at-bs)=0
  \]
  に関する商集合
  \[
    S^{-1}A:=A\times S/\sim
  \]
  を、積閉集合$S$による$A$の\textbf{分数環}という。
  $S^{-1}A$の元を
  \[
    \frac{a}{s}=a/s
  \]
  などと書き表す。
\end{definition}

分数環は本来、自然に導入される環構造を含めて分数環と呼びます。
\begin{theorem}
  $A$を環、$S$をその積閉集合とする。
  このとき、$S^{-1}A$に以下の演算
  \begin{align*}
    \frac{a}{s}+\frac{b}{t}&:=\frac{at+bs}{st}\\
    \frac{a}{s}\cdot\frac{b}{t}&:=\frac{ab}{st}
  \end{align*}
  を入れることによって、$S^{-1}A$は環である。
  加法単位元は$0/1$、乗法単位元は$1/1$である。
\end{theorem}
\begin{proof}
  まず演算がうまく定義されていることを確認する。
  $a/s=a'/s'$、$b/t=b'/t'$とするとき、すなわちある$u_1,u_2\in S$が存在して、
  \[
    u_1(as'-a's)=0,\quad u_2(bt'-b't)=0
  \]
  である。
  このとき、和は
  \[
    u_1u_2\{(at+bs)s't'-(a't'+b's')st\}=u_2tt'u_1(as'-a's)+u_1ss'u_2(bt'-b't)=0
  \]
  より
  \[
    \frac{at+bs}{st}=\frac{a't'+b's'}{s't'}
  \]
  積は
  \[
    u_1u_2(abs't'-a'b'st)=u_1u_2(abs't'-a'bst'+a'bst'-a'b'st)=u_2bt'u_1(as'-a's)+u_1a'su_2(bt'-b't)=0
  \]
  より
  \[
    \frac{ab}{st}=\frac{a'b'}{s't'}
  \]
  ゆえに演算はwell-definedである。
  \begin{itemize}
    \item \textbf{演算$+$に関して可換群} 
    \begin{itemize}
      \item \textbf{結合法則} $a_1/s_1+(a_2/s_2+a_3/s_3)=a_1/s_1+(a_2s_3+a_3s_2)/s_2s_3=\{a_1s_2s_3+(a_2s_3+a_3s_2)s_1\}/s_1s_2s_3=\{(a_1s_2+a_2s_1)s_3+a_3s_1s_2\}/s_1s_2s_3=(a_1s_2+a_2s_1)/s_1s_2+a_3/s_3=(a_1/s_1+a_2/s_2)+a_3/s_3$
      \item \textbf{単位元} $a/s+0/1=(a\cdot1+0\cdot s)/s\cdot1=a/s$、$0/1+a/s=(0\cdot s+a\cdot1)/1\cdot s=a/s$
      \item \textbf{逆元} $a/s+(-a/s)=(a\cdot s-a\cdot s)/s^2=0/s^2=0/1$ ($\because$ $1\cdot(0\cdot1-0\cdot s^2)=0$)
      $(-a/s)+a/s=0/1$も同様。
      \item \textbf{可換性} $a/s+b/t=(at+bs)/st=(bs+at)/ts=b/t+a/s$
    \end{itemize}
    \item \textbf{演算$\cdot$に関して可換モノイド}
    \begin{itemize}
      \item \textbf{結合法則} $a_1/s_1\cdot (a_2/s_2\cdot a_3/s_3)=a_1/s_1\cdot(a_2a_3/s_2s_3)=a_1(a_2a_3)/s_1(s_2s_3)=(a_1a_2)a_3/(s_1s_2)s_3=(a_1a_2/s_1s_2)\cdot a_3/s_3=(a_1/s_1\cdot a_2/s_2)\cdot a_3/s_3$
      \item \textbf{単位元} $a/s\cdot 1/1=a\cdot1/s\cdot1=a/s$、$1/1\cdot a/s=1\cdot a/1\cdot s=a/s$
      \item \textbf{可換性} $a/s\cdot b/t=ab/st=ba/ts=b/t\cdot a/s$
    \end{itemize}
    \item \textbf{分配法則} $a_1/s_1\cdot(a_2/s_2+a_3/s_3)=a_1/s_1\cdot(a_2s_3+a_3s_2)/s_2s_3=a_1(a_2s_3+a_3s_2)/s_1s_2s_3=(a_1a_2s_3+a_1a_3s_2)/s_1s_2s_3=a_1a_2s_3/s_1s_2s_3+a_1a_3s_2/s_1s_2s_3=a_1a_2/s_1s_2+a_1a_3/s_1s_3=(a_1/s_1\cdot a_2/s_2)+(a_1/s_1\cdot a_3/s_3)$
  \end{itemize}
\end{proof}

さて、僕たちは積閉集合$S$にゼロ因子どころか、0が入ることを許容しています。
もし0が入っていると、分数環$S^{-1}A$は$1=0$の零環に潰れてしまいます。
実際、任意の$a/s\in S^{-1}A$に対して、$0\in S$が存在「してしまい」、
\[
  0\cdot(a\cdot 1-0\cdot s)=0
\]
をみたすので、$a/s=0/1$です。

僕たちは零環、すなわち$1=0$となる環を認めているので、このことに特に問題はないです。
一方$1\neq0$を満たすもののみを環とする主義では、積閉集合の定義に$0\neq S$を追加する必要があります。


\subsection{自然な準同型}

剰余環のときと同様に、局所化の元となった環$A$からの、「自然」と呼ぶことができる準同型が存在します。
\begin{theorem}
  $A$を環、$S$をその積閉集合とする。
  このとき、写像
  \[
    \varphi_S:A\to S^{-1}A;a\mapsto a/1
  \]
  は準同型である。
\end{theorem}
これはもはや自明と言ってよいでしょう。
それよりも、この準同型の性質を見たいです。
もし単射であれば、$A$は$S^{-1}A$の部分環とみなすことができ、$\mathbb{Z}\subset\mathbb{Q}$の直感とも合います。
しかしゼロ因子の存在により、状況は少し複雑になっています。
\begin{theorem}
  $A$を環、$S$をその積閉集合とする。
  このとき、自然な準同型
  \[
    \varphi_S:A\to S^{-1}A;a\mapsto a/1
  \]
  のカーネルは
  \[
    \operatorname{Ker}(\varphi_S)=\{a\in A\mid {}^\exists s\in S \text{ s.t. } as=0\}
  \]
  である。
\end{theorem}
\begin{proof}
  \begin{align*}
    a\in\operatorname{Ker}(\varphi_S)&\iff\varphi(a)=\frac{a}{1}=\frac01\\
    &\iff{}^\exists s \text{ s.t. } s(a\cdot1-0\cdot1)=0\\
    &\iff{}^\exists s \text{ s.t. } as=0
  \end{align*}
\end{proof}

従って、$S$がゼロ因子をもっていると単射ではないのです。
なのでそもそも$A$がゼロ因子を持たない場合、つまり整域の場合、どんな積閉集合$S$をとっても自然に$A\subset S^{-1}A$とみなすことができます。

\subsection{$\mathbb{Z}$から$\mathbb{Q}$、あるいは整域の商体}

$\mathbb{Z}$の部分集合
\[
  S:=\mathbb{Z}\setminus\{0\}
\]
は明らかに積閉集合です。
この分数環$S^{-1}\mathbb{Z}$が有理数体になっているというのが、小学校から学んできた普通の事実です。
\[
  S^{-1}\mathbb{Z}=\mathbb{Q}
\]

一般の整域$A$に対しても、同様のことが言えます。
\begin{theorem}
  $A$を整域、$S:=A\setminus\{0\}$とする。
  このとき、$S$は積閉集合である。
\end{theorem}
\begin{proof}
  もし$s,t\in S$かつ$st=0$を満たすものがあるとすると、$A$は整域であったから、$s=0$または$t=0$を満たす。
  ところがこれは、$s,t\in S=A\setminus\{0\}$であったことに矛盾する。
\end{proof}

また、$A$が整域で$S:=A\setminus\{0\}$のとき、
\[
  K:=S^{-1}A
\]
は明らかに体です(急に明らかと言われてびっくりするかもですが、手を動かし始めると本当に示すべきことがないです)。
このようにして得られる体には名前がついています。
\begin{definition}
  $A$を整域、$S:=A\setminus\{0\}$とする。
  このとき、体
  \[
    S^{-1}A
  \]
  を、整域$A$の\textbf{商体}という。
\end{definition}

\subsection{局所化$A_\mathfrak{p}$}

素イデアルが作る積閉集合、およびその分数環は代数幾何学では重要です。

\begin{theorem}
  $A$を環、$\mathfrak{p}$をその素イデアルとする。
  このとき、$S=A\setminus\mathfrak{p}$は$A$の積閉集合である。
\end{theorem}
\begin{proof}
  $1\in\mathfrak{p}$とすると$\mathfrak{p}=A$となっておかしいため、$1\notin\mathfrak{p}$。すなわち$1\in S=A\setminus\mathfrak{p}$。

  $x,y\in S$ならば$xy\in S$であることは、対偶を取れば「$xy\in \mathfrak{p}$ならば$x \in \mathfrak{p} \lor y \in \mathfrak{p}$」であり、これは$\mathfrak{p}$が素イデアルであることの定義に他ならない。
\end{proof}

\begin{definition}[局所化]
  $A$を環、$\mathfrak{p}$をその素イデアルとする。
  積閉集合$S:=A\setminus\mathfrak{p}$による$A$の分数環を$A_\mathfrak{p}$と書き、$A$の素イデアル$\mathfrak{p}$による\textbf{局所化}(\textbf{localization})という。
\end{definition}

「局所化」というネーミングの意味合いを説明するには、幾何学的な観点を導入せざるを得ないでしょう。
代数幾何学では、環$A$は多様体全体で「正則」な関数全体のなす環をイメージして構成されます。
一方で局所環は、多様体(variety)上の1点、ないし、部分多様体の近傍において正則な関数全体のなす環をイメージしています。

例えば$k$を代数閉体、$P:=(a,b)$を$k^2$の点とします。
$k^2$に対応する環は、$A=k[x,y]$です。
このとき、$P$の近傍では、$(x,y)=(a,b)$で0にならない正則関数で割っても正則であるはずです。
つまり、$k^2$の点$P$の近傍で正則な関数は一般的に、
\[
  \frac{f}{g},\quad\text{ただし$g$は$(x-a)$でも$(y-b)$でも割り切れない}
\]
と書けてしかるべきです。
一方で点$P$に対応する素イデアルは$\mathfrak{p}=(x-a,y-b)$ですから、$\mathfrak{p}$による$A=k[x,y]$の局所化は上記の環に一致します。
これが局所化のイメージです。
$A$が整域だったら、$A$は自然に$A_\mathfrak{p}=S^{-1}A$の部分環とみなせるのでしたから、大方において局所化すると環が大きくなります。

多様体(manifold)論から、ほとんど同じことではあるのですが、違った見方を紹介します。
ある点$P\in M$の近傍で定義される関数は、ようは点$P$で発散したりしてなければよいのです。
そこで、点$P$の近傍で定義された関数全体のなす環は
\[
  \mathcal{O}_{M,P}:=\{\left<U,f\right>\mid \text{$U\subset M$: 開集合}, P\in U, f:U\to\mathbb{C}\}
\]
と書いてもよいことがわかります。
ここで、各元は
\begin{align*}
  \left<U,f\right>=\left<V,g\right> \iff f|_W=g|_W,\quad\exists W\subset U\cap V\text{:$P$の開近傍}
\end{align*}
という同値関係で同一視されています。
環$\mathcal{O}_{M,P}$は複雑かもしれませんが、次のような意味できわめて単純です。
\begin{theorem}
  $M$を多様体、$P\in M$とする。
  環$\mathcal{O}_{M,P}$の極大イデアルは
  \[
    \mathfrak{m}_P:=\{\left<U,f\right>\mid f(P)=0\}
  \]
  の一つのみである。
\end{theorem}
多様体の定義はろくにやってないので、証明は適当に述べます。
まず$\mathfrak{m}_P$が極大であることは次のようにわかります。
$\mathfrak{m}_P\subsetneq\mathfrak{a}\subset\mathcal{O}_{M,P}$とすると、$\mathfrak{a}$には$P\in M$で0にならない関数$f$が入ってくるはずです。
$f$は連続関数なので、$P$の十分小さな近傍$U$で常に0になりません。
そしたら$\left<U,1/f\right>\in\mathcal{O}_{M,P}$であり、
\[
  1=\left<U,1/f\right>\cdot\left<U,f\right>\in\mathfrak{a}
\]
ゆえに$\mathfrak{a}=\mathcal{O}_{M,P}$がわかり、$\mathfrak{m}_P$が極大イデアルであることがわかります。
またこのように、$\mathcal{O}_{M,P}\setminus\mathfrak{m}_P$の元はすべて単元なので、$\mathcal{O}_{M,P}$でない任意のイデアルは$\mathfrak{m}_P$に含まれなければいけません。

この性質を環の言葉に一般化して、次のように定義します。
\begin{definition}
  環$A$が\textbf{局所環}(\textit{local ring})であるとは、$A$が唯一の極大イデアルをもつときをいう。
\end{definition}

環の局所化は、実際に局所環になっています。
\begin{theorem}
  環$A$の素イデアル$\mathfrak{p}$による局所化$A_\mathfrak{p}$は局所環である。
\end{theorem}
\begin{proof}
  \[
    \mathfrak{m}:=\left\{\frac{a}{s}\in A_\mathfrak{p}\mid a\in\mathfrak{p}, s\in A\setminus\mathfrak{p}\right\}
  \]
  とおく。このとき、$\mathfrak{m}$が$A_\mathfrak{p}$の唯一の極大イデアルであることを示す。

  \begin{itemize}
    \item イデアルであること\\
    $0\in\mathfrak{p}$より$0=0/1\in\mathfrak{m}$は自明。
    $a/s,b/t\in\mathfrak{m}$とおくと、すなわち$a,b\in\mathfrak{p}$より$at+bs\in\mathfrak{p}$。
    ゆえに$a/s+b/t=(at+bs)/st\in\mathfrak{m}$。
    最後に、$a/s\in A_\mathfrak{p}$かつ$x/t\in\mathfrak{m}$とすると、$x\in\mathfrak{p}$ゆえ$a/s\cdot x/t=ax/st\in\mathfrak{m}$。
    以上より$\mathfrak{m}$はイデアルである。
    \item $A_\mathfrak{p}\setminus\mathfrak{m}$が単元のみからなること、従って$\mathfrak{m}$が唯一の極大イデアルであること\\
    $a/s\in A_\mathfrak{p}\setminus\mathfrak{m}$とする。
    このとき、$\mathfrak{m}$の定義から$a\notin\mathfrak{p}$ゆえ、$a\in S$。
    従って$s/a\in A_\mathfrak{p}$。
  \end{itemize}
\end{proof}



\section{$A$代数}

体に対するベクトル空間のように、環に対する以下の概念を考えることができます。
\begin{definition}[加群]
  $A$を環、$M$を加法群とする。
  $M$が\textbf{$A$加群}(\textit{$A$-module})であるとは、スカラー倍と呼ばれる演算$A\times M\to M$が定義されていて、以下を満たす時を言う。
  任意の$a,b\in A$および$x,y\in M$に対して、
  \begin{itemize}
    \item (結合法則) $a(bx)=(ab)x$
    \item (分配法則)
    \begin{itemize}
      \item $(a+b)x=ax+bx$
      \item $a(x+y)=ax+ay$
    \end{itemize}
    \item (1倍) $1\cdot x=x$
  \end{itemize}
\end{definition}

\begin{example}
  $\mathbb{Z}^n$は、演算
  \begin{itemize}
    \item $(x_1,\dots,x_n)+(y_1,\dots,y_n)=(x_1+y_1,\dots,x_n+y_n)$
    \item $a(x_1,\dots,x_n)=(ax_1,\dots,ax_n)$
  \end{itemize}
  によって$\mathbb{Z}$加群である。
  一般に、環$A$に対して$A^n$は$A$加群である。
\end{example}
また、加群を軸にみたとき、整数環$\mathbb{Z}$はかなり特別な存在です。
\begin{example}
  任意の加法群$M$ (単位元は$1_M$とする)は、以下の演算で自然に$\mathbb{Z}$加群である。
  \[
    n\cdot x:=\begin{cases}
      (\underbrace{1_M+\cdots+1_M}_{n\text{個}})\cdot x & n>0 \\
      0 & n=0 \\
      -(\underbrace{1_M+\cdots+1_M}_{|n|\text{個}})\cdot x & n<0
    \end{cases}
  \]
\end{example}

体$K$上のベクトル空間は、$K$を環とみた$K$加群と言い直すこともできます。
またベクトル空間のように、$A^n$は、次のような基底をもつと言えます。
\[
  e_1=(1,0,\dots,0),\dots,e_n=(0,\dots,0,1)
\]

一方でベクトル空間と違う点は、例えば次のような加群があることです。
\begin{example}
  整数$n$に対して、$\mathbb{Z}/n\mathbb{Z}$は$\mathbb{Z}$加群である。
  一般に$r\geq0$、$m\geq0$、$n_1,\dots,n_m>1$を整数とするとき、
  \[
    M=\mathbb{Z}^r\times\mathbb{Z}/n_1\mathbb{Z}\times\cdots\times\mathbb{Z}/n_m\mathbb{Z}
  \]
  は$\mathbb{Z}$加群である。
\end{example}
この例の何がベクトル空間と違うかというと、体$K$上のベクトル空間$V$の場合、$x\in V\setminus\{0\}$なら
\[
  ax=0 \iff a=0
\]
は正しいのですが、上記の加群の例ではそうとは限りません。
いずれにしても、上記の例はどれも以下のような性質をもっています。
\begin{definition}
  $M$を環$A$の加群とする。
  $M$が\textbf{有限生成$A$加群}であるとは、$M$の有限個の元$e_1,\dots,e_n$が存在して、任意の$x\in M$に対して、ある$a_1,\dots,a_n\in A$が存在して、
  \[
    x=\sum_{i=1}^na_ie_i
  \]
  を満たす時を言う。
\end{definition}

\subsection{加群の準同型}

ふたつの$A$加群が似ているかどうかを、今まで通り加法群とみた準同型で比べるのは良さそうです。
しかし、準同型には$A$加群であるという情報をのせたいところ。
この場合、次のような準同型を考えるのが良いでしょう。
\begin{definition}
  $A$を環、$M,N$を$A$加群とする。
  群準同型$f:M\to N$が\textbf{$A$加群の射}であるとは、以下を満たす時を言う。
  \[
    f(ax+by)=af(x)+bf(y), \forall a,b\in A,\;\forall x,y\in M
  \]
  \textbf{$A$加群の同型射}(誤解がなければ単に同型という)とは、群準同型として同型な$A$加群の射をいう。
  $A$加群$M$と$N$の間に$A$加群の同型射があるとき、$M$と$N$は互いに\textbf{$A$加群として同型である}という(誤解がなければ単に同型であるという)。
\end{definition}
ベクトル空間の線形写像は、$K$代数の射と言い直すことができます。

\begin{example}
  有限生成$\mathbb{Z}$加群$M$は
\end{example}

\subsection{$A$代数}

加群というもの自体、加法群を環というレンズを通して調べる方法であって、加群を導入した時点で環論が一段上のレベルに上がるものと思います。
例えば、環自体も和によって加法群なので、環を加群とみなすことができます。
一般に、環準同型$f:A\to B$があるとき、環$B$に以下のような$A$加群の構造を見出すことができます。
\begin{theorem}
  $A,B$を環、$f:A\to B$を環準同型とする。
  このとき、スカラー倍$A\times B\to B$を
  \[
    a\cdot x:=f(a)x
  \]
  で定義すると、$B$は$A$加群である。
\end{theorem}
\begin{proof}
  $a,b\in A$、$x,y\in B$に対して、
  \begin{itemize}
    \item (結合法則) $a(bx)=a(f(b)x)=f(a)f(b)x=f(ab)x=(ab)x$
    \item (分配法則)
    \begin{itemize}
      \item $(a+b)x=f(a+b)x=(f(a)+f(b))x=f(a)x+f(b)x=ax+bx$
      \item $a(x+y)=f(a)(x+y)=f(a)x+f(a)y=ax+ay$
    \end{itemize}
    \item (1倍) $1\cdot x=f(1)x=1\cdot x=x$
  \end{itemize}
\end{proof}

このように、環準同型$f:A\to B$があるときはいつでも、$B$を$A$加群と思うことができます。
さらに$B$にはベクトル空間的な構造のほかに、環由来の掛け算も備えているので、特別に以下のような名前がついています。
\begin{definition}
  $A,B$を環、$f:A\to B$を環準同型とするとき、$A$加群$B$を\textbf{$A$代数}(\textit{$A$-Algebra})という。
\end{definition}
特に環準同型が単射のとき、自然に$A\subset B$と思うことができます。
これはちょうど
\begin{example}
  $\mathbb{Z}\subset\mathbb{Q}\subset\mathbb{R}\subset\mathbb{C}$
\end{example}
などの単純な状況です。
代数幾何学でより重要なのは以下でしょう。
\begin{example}
  $A$を環とするとき、多項式環$A[X_1,\dots,X_n]$は$A$代数である。
\end{example}

また、$A$代数$B$に対して、$B$加群$M$があるとき、次のように$M$は$A$加群になります。
\begin{theorem}
  $A,B$を環、$M$を$B$加群、$f:A\to B$を環準同型とする。
  このとき、$M$は次の演算により$A$加群である。
  \[
    a\cdot x=f(a)x,\quad\forall a\in A,\;\forall x\in B
  \]
\end{theorem}
\begin{proof}
  さすがに自明。
\end{proof}



\section{ヒルベルトの零点定理}

準備は整いました。ここから舞台を代数的閉体$K$上の多項式環$A = K[x_1, \dots, x_n]$に移します。

\subsection{代数的集合とイデアル}

($V(\mathfrak{a})$ と $I(Z)$ の定義)

ここで素朴な疑問が生じます。
イデアル$\mathfrak{a}$から出発して、その零点集合$V(\mathfrak{a})$を考え、さらにその消失イデアル$I(V(\mathfrak{a}))$を考えたとき、元の$\mathfrak{a}$に戻るでしょうか?
答えはNoです。先ほど見たように、$\mathfrak{a}=(x^2)$なら$I(V(\mathfrak{a}))=(x)$になってしまいます。
では、$\sqrt{\mathfrak{a}}$をとれば一致するのか?
その答えが、次の定理です。

\subsection{弱零点定理}

まずは「点」に対応する極大イデアルの形を決定します。

\begin{theorem}[Hilbert's Nullstellensatz, weak form]
  $K$を代数的閉体とする。$K[x_1, \dots, x_n]$の任意の極大イデアル$\mathfrak{m}$は、
  \[
    \mathfrak{m} = (x_1 - a_1, \dots, x_n - a_n), \quad (a_1, \dots, a_n \in K)
  \]
  の形に書ける。
\end{theorem}

この証明には、次の代数的な補題が強力な役割を果たします。
(ザリスキの補題を紹介)

\subsection{強零点定理とRabinowitschのトリック}

変数を一つ増やすことで、魔法のように弱形式から一般系を導くことができます。これをRabinowitschのトリックと呼びます。

\begin{theorem}[Hilbert's Nullstellensatz, strong form]
  $K$を代数的閉体、$A=K[x_1, \dots, x_n]$とする。
  任意のイデアル$\mathfrak{a} \subset A$に対して、
  \[
    I(V(\mathfrak{a})) = \sqrt{\mathfrak{a}}
  \]
\end{theorem}
\end{comment}
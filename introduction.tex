\chapter*{序論}

代数幾何学の古典的名著、HartshoneのAlgebraic Geometryを精読します。
この本は1977年出版ということで、若干古いものの、現在でも代数幾何学を勉強するときに多くの学生がこれを参考に勉強をしています。
本書の技術的なコアはおそらく第3章のCohomologyなのですが、僕は今まで何度もこの本にアタックして、第2章のSchemeでつまずいています。
この原因は、まず僕が英語苦手すぎなこと。
次に可換環論について理解が浅いことだと思います。
結果として僕は可換環論の本をたくさん買いましたが、多くの重要な結果が互いに演習問題にしあっていたり、導入の動機が不明な概念を多く含んでいることに目を白黒させているうちに、読了をあきらめていました。
今回はこれを頑張って精読するという狂気のプロジェクトを立ち上げました。

この本のPrefaceには、Hartshone先生がバークレー大学で行った大学院向けの授業で使っていることが書かれています。
日本の大学だと、これほど丁寧に代数幾何について講義を受けられる機会は少ないので、うらやましいですね。
いや、これは代数幾何の基礎が整った今だからかもしれません。
今だとHartshone、Liu、あるいは邦書の代数幾何学の本がいくつかあるので、それを師匠からポンと渡されて、「これでセミナーしてください」って言われてボコボコにされながら読む感じがします。
それを僕は、数学科から還俗してお気楽に一人でやろうっていうことです。

じゃあ、やってみましょう。

\chapter*{序論}

代数幾何学の古典的名著、\mycite{Har77}を精読します。
この本は1977年出版ということで、若干古いものの、現在でも代数幾何学を勉強するときに多くの学生がこれを参考に勉強をしています。
本書の技術的なコアはおそらく第3章のCohomologyなのですが、僕は今まで何度もこの本にアタックして、第2章のSchemeでつまずいています。
この原因は、まず僕が英語苦手すぎなこと。
次に可換環論について理解が浅いことだと思います。
結果として僕は可換環論の本をたくさん買いましたが、多くの重要な結果が互いに演習問題にしあっていたり、導入の動機が不明な概念を多く含んでいることに目を白黒させているうちに、読了をあきらめていました。
今回はこれを頑張って精読するという狂気のプロジェクトを立ち上げました。

この本のPrefaceには、Hartshone教授がバークレー大学で行った大学院向けの授業で使っていることが書かれています。
日本の大学だと、これほど丁寧に代数幾何について講義を受けられる機会は少ないので、うらやましいですね。
いや、これは代数幾何の基礎が整った今だからかもしれません。
今だとHartshone、Liu、あるいは邦書の代数幾何学の本がいくつかあるので、それを師匠からポンと渡されて、「これでセミナーしてください」って言われてボコボコにされながら読む感じがします。
それを僕は、数学科から還俗してお気楽に一人でやろうっていうことです。

そんなお気楽な身分だからこそ、本に取り組む前に「いったい何故この概念を導入する必要があるのだろう?」という疑問にもチャレンジできます。
学生時代にこんなことやっていたら、時間がいくらあっても足りませんからね\footnote{
  世の中には「Hartshorneは2章と3章だけ読めばいい」と言ってはばからない人すらいます。
}。
そういうわけで、第0章では自分が調べたり想像したりした、現代の代数幾何学がなんでこんなことになってしまっているのか?ということを書きまくりました。
一言でいうと\textbf{Riemann-Rochの定理の任意の体への拡張とWeil予想の解決}が目標だったのですが、これらを説明しようとするとまあ長くなっちゃいました。
でもこれを知っておくと、他の人よりもモチベが違うと思うので、読んでおくと良いかもしれません!

じゃあ、やってみましょう。

\chapter{多様体}

\footnote{
  いきなりですがVarietyとManifoldを同じ「多様体」と訳したのは誰なんですかね?
  Hartshorneに出てくる多様体はVarietyです。
}
この章の目標は、少ない知識から代数幾何の重要な概念をいくつか紹介することだそうです。
ここでは代数閉体$k$上のアフィン空間や射影空間に含まれる、本当に方程式のゼロ点集合を扱います。
そこから始めて、多様体の、
\begin{itemize}
  \item 次元
  \item 正則関数
  \item 有理写像
  \item 非特異多様体
  \item 射影多様体の次数
\end{itemize}
といったものを紹介していきます。

親切(心折)にも、各節の終わりにたくさんの演習問題を用意してくれています。
これも頑張って解いていきましょう。
なお\textbf{ちまちま未解決問題が含まれている}ので、それはさすがに手を付けません。
あと、この本は代数幾何学の本なので、可換環論の結果はあんまり証明を書いてくれてません。
そこも頑張って埋めていきたいところですね。

最後の節では、代数幾何学が結局どこに向かっているのかについて解説しています。
これは僕のお気持ちである第0章もあるので、ざっくり省きます。

\section{アフィン多様体}

\begin{definition}
  $k$を代数閉体とする。このとき
  \[
    \mathbb{A}_k^n:=\{(a_1,\dots,a_n)\mid a_i\in k\}
  \]
  を\textbf{アフィン空間}という。
  体が明らかなときは、単に$\mathbb{A}^n:=\mathbb{A}_k^n$と書くこともある。
\end{definition}

$k$を代数閉体、$A:=k[x_1,\dots,x_n]$を$k$上の多項式環とします。
このとき、任意の$f\in A$に対して、代入写像
\[
  f:\mathbb{A}^n\to k; P\mapsto f(P)
\]
を考えることができます。
より具体的には、$P=(a_1,\dots,a_n)$のとき、$f(P)$は、$f$に現れている変数$x_i$をすべて$a_i$に置き換えて得られる$k$の元です。

よって$f(P)=0$となる$P$の集合、つまり\textbf{ゼロ点集合}を定義することができます。
より一般に、$A$の部分集合$T$に対して次のようにゼロ点集合を定義します。
\begin{definition}
  $k$を代数閉体、$\mathbb{A}^n$をアフィン空間、$A=k[x_1,\dots,x_n]$を多項式環、$T\subset A$を部分集合とする。
  このとき、$T$のゼロ点集合を
  \[
    Z(T):=\{P\in\mathbb{A}^n\mid f(P)=0 ({}^\forall f\in T)\}
  \]
  で定義する。
\end{definition}

以下の主張は、著者は明らかと言っています。
確かに明らかといえば明らかですが、セミナーで聞かれたらちゃんと答えられない学生もいるでしょうから、証明しましょう\footnote{「明らか」はセミナーでは地雷ワードです。}。
\begin{remark}
  $k$を代数閉体、$\mathbb{A}^n$をアフィン空間、$A=k[x_1,\dots,x_n]$を多項式環、$T\subset A$を部分集合とする。
  $\mathfrak{a}$が$T$で生成される$A$のイデアルならば
  \[
    Z(T)=Z(\mathfrak{a})
  \]
\end{remark}
\begin{proof}
  $P\in Z(T)$とする。
  すなわち、任意の$f\in T$に対して、$f(P)=0$。
  $\mathfrak{a}$は$T$で生成されたイデアルであるから、その任意の元$F$は
  \[
    F=\sum_{i=1}^rg_if_i,\quad \text{ただし }f_i\in T, g_i\in A
  \]
  と書かれる。
  任意の$f\in T$に対して$f(P)=0$であったから、
  \[
    F(P)=\sum_{i=1}^rg_i(P)f_i(P)=\sum_{i=1}^rg_i(P)\cdot0=0
  \]
  ゆえに$P\in Z(\mathfrak{a})$。
  従って$Z(T)\subset Z(\mathfrak{a})$。

  逆に$P\in Z(\mathfrak{a})$とする。
  $\mathfrak{a}$は$T$で生成されたイデアルであるから、$T\subset\mathfrak{a}$。
  任意の$f\in T$に対して$f\in\mathfrak{a}$であるから、$f(P)=0$となる。
  ゆえに$P\in Z(T)$。
  従って$Z(\mathfrak{a})\subset Z(T)$。
  以上より$Z(T)=Z(\mathfrak{a})$。
\end{proof}


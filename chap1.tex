\chapter{多様体}

\footnote{
  いきなりですがVarietyとManifoldを同じ「多様体」と訳したのは誰なんですかね?
  Hartshorneに出てくる多様体はVarietyです。
}
この章の目標は、少ない知識から代数幾何の重要な概念をいくつか紹介することだそうです。
ここでは代数閉体$k$上のアフィン空間や射影空間に含まれる、本当に方程式のゼロ点集合を扱います。
そこから始めて、多様体の、
\begin{itemize}
  \item 次元
  \item 正則関数
  \item 有理写像
  \item 非特異多様体
  \item 射影多様体の次数
\end{itemize}
といったものを紹介していきます。

親切(心折)にも、各節の終わりにたくさんの演習問題を用意してくれています。
これも頑張って解いていきましょう。
なお\textbf{ちまちま未解決問題が含まれている}ので、それはさすがに手を付けません。
あと、この本は代数幾何学の本なので、可換環論の結果はあんまり証明を書いてくれてません。
そこも頑張って埋めていきたいところですね。

最後の節では、代数幾何学が結局どこに向かっているのかについて解説しています。
これは僕のお気持ちである第0章もあるので、ざっくり省きます。

\section{アフィン多様体}

\begin{definition}
  $k$を代数閉体とする。このとき
  \[
    \mathbb{A}_k^n:=\{(a_1,\dots,a_n)\mid a_i\in k\}
  \]
  を\textbf{アフィン空間}という。
  体が明らかなときは、単に$\mathbb{A}^n:=\mathbb{A}_k^n$と書くこともある。
\end{definition}

$k$を代数閉体、$A:=k[x_1,\dots,x_n]$を$k$上の多項式環とします。
このとき、任意の$f\in A$に対して、代入写像
\[
  f:\mathbb{A}^n\to k; P\mapsto f(P)
\]
を考えることができます。
より具体的には、$P=(a_1,\dots,a_n)$のとき、$f(P)$は、$f$に現れている変数$x_i$をすべて$a_i$に置き換えて得られる$k$の元です。

よって$f(P)=0$となる$P$の集合、つまり\textbf{ゼロ点集合}を定義することができます。
より一般に、$A$の部分集合$T$に対して次のようにゼロ点集合を定義します。
\begin{definition}
  $k$を代数閉体、$\mathbb{A}^n$をアフィン空間、$A=k[x_1,\dots,x_n]$を多項式環、$T\subset A$を部分集合とする。
  このとき、$T$のゼロ点集合を
  \[
    Z(T):=\{P\in\mathbb{A}^n\mid f(P)=0 ({}^\forall f\in T)\}
  \]
  で定義する。
\end{definition}

以下の主張は、著者は明らかと言っています。
確かに明らかといえば明らかですが、セミナーで聞かれたらちゃんと答えられない学生もいるでしょうから、証明しましょう\footnote{「明らか」はセミナーでは地雷ワードです。}。
\begin{remark}
  $k$を代数閉体、$\mathbb{A}^n$をアフィン空間、$A=k[x_1,\dots,x_n]$を多項式環、$T\subset A$を部分集合とする。
  $\mathfrak{a}$が$T$で生成される$A$のイデアルならば
  \[
    Z(T)=Z(\mathfrak{a})
  \]
\end{remark}
\begin{proof}
  $P\in Z(T)$とする。
  すなわち、任意の$f\in T$に対して、$f(P)=0$。
  $\mathfrak{a}$は$T$で生成されたイデアルであるから、その任意の元$F$は
  \[
    F=\sum_{i=1}^rg_if_i,\quad \text{ただし }f_i\in T, g_i\in A
  \]
  と書かれる。
  任意の$f\in T$に対して$f(P)=0$であったから、
  \[
    F(P)=\sum_{i=1}^rg_i(P)f_i(P)=\sum_{i=1}^rg_i(P)\cdot0=0
  \]
  ゆえに$P\in Z(\mathfrak{a})$。
  従って$Z(T)\subset Z(\mathfrak{a})$。

  逆に$P\in Z(\mathfrak{a})$とする。
  $\mathfrak{a}$は$T$で生成されたイデアルであるから、$T\subset\mathfrak{a}$。
  任意の$f\in T$に対して$f\in\mathfrak{a}$であるから、$f(P)=0$となる。
  ゆえに$P\in Z(T)$。
  従って$Z(\mathfrak{a})\subset Z(T)$。
  以上より$Z(T)=Z(\mathfrak{a})$。
\end{proof}

$A=k[x_1,\dots,x_n]$はHilbertの基底定理よりNoether環でしたので、イデアル$\mathfrak{a}$は有限個の生成元$f_1,\dots,f_n\in A$をもちます。
つまり$Z(T)$は多項式$f_1,\dots,f_n$の共通ゼロ点集合ということです。
別に$T$は有限集合と限っているわけではないのに、結局は有限個の多項式のゼロ点集合で表わせるのは面白いですね。

今まではイデアルからゼロ点集合を作りましたが、逆にイデアルのゼロ点集合になれる部分集合を定義します。
\begin{definition}
  $k$を代数閉体、$\mathbb{A}^n$をアフィン空間、$A=k[x_1,\dots,x_n]$とする。
  $Y\subset\mathbb{A}^n$が\textbf{代数的集合}(\textit{algebraic set})であるとは、ある部分集合$T\subset A$が存在して、$Y=Z(T)$を満たす時を言う。
\end{definition}
代数的集合からなる$\mathbb{A}^n$の部分集合族は位相をなすことがわかります。
\begin{theorem}[本文Proposition 1.1]
  代数的集合の集合族は閉集合の公理を満たす。すなわち
  \begin{enumerate}
    \item 二つの代数的集合の和集合はまた代数的集合
    \item 任意の代数的集合の族の共通集合はまた代数的集合
    \item 空集合と全体$\mathbb{A}^n$は代数的集合
  \end{enumerate}
\end{theorem}
\begin{proof}
  (1.) $Y_1=Z(T_1)$、$Y_2=Z(T_2)$を代数的集合とする。ここで$T_1T_2$で、$T_1$の元と$T_2$の元をひとつずつかけ合わせたもの全体の集合とする。
  このとき、$Y_1\cup Y_2=Z(T_1T_2)$であることが次のように示される。
  $P\in Y_1\cup Y_2$とすると、$P\in Y_1$または$P\in Y_2$。
  $P\in Y_1$とすれば、任意の$f\in T_1$に対して$f(P)=0$だから、任意の$fg\in T_1T_2$に対して$f(P)g(P)=0$ゆえ$P\in Z(T_1T_2)$。
  $P\in Y_2$としても同様であるから、$Y_1\cup Y_2\subset Z(T_1T_2)$。
  逆に$P\in Z(T_1T_2)$かつ$P\notin Y_1$であるとすると、ある$f\in T_1$が存在して、$f(P)\neq0$。
  このとき、$p\in Z(T_1T_2)$であるから、任意の$g\in T_2$に対して$f(P)g(P)=0$であるはずであるが、$f(P)\neq0$であるから$g(P)=0$が任意の$g\in T_2$について成り立つということである。
  従って$P\in Y_2$となる。
  同様に$P\notin Y_2$ならば$P\in Y_1$が分かるから、$Z(T_1T_2)\subset Z(T_1T_2)$。
  従って$Y_1\cup Y_2=Z(T_1T_2)$。

  (2.) $Y_\alpha=Z(T_\alpha)$を代数的集合の族とする。
  このとき$\bigcap_\alpha Y_\alpha=Z(\bigcup_\alpha T_\alpha)$であることが次のように示せる。\\
  $P\in\bigcap_\alpha Y_\alpha=\bigcap_\alpha Z(T_\alpha)$\\
  $\iff$ 任意の$f\in\bigcup_\alpha T_\alpha$に対して、$f(P)=0$。\\
  $\iff$ $P\in Z(\bigcup_\alpha T_\alpha)$。

  (3.) $\varnothing =Z(\{1\})$、$\mathbb{A}^n=Z(\{0\})$である。
\end{proof}

\begin{definition}
  代数閉体上のアフィン空間$\mathbb{A}^n$の閉集合族が誘導する位相を\textbf{ザリスキー(Zariski)位相}という。
\end{definition}
Zariski位相における開集合は、代数的集合の補集合です。
いくつか具体例を見ていきましょう。

\begin{example}[本文 Example1.1.1]
  $\mathbb{A}^1$のZariski位相を考える。
  $A=k[x]$はPIDだから、任意のイデアルは一つの多項式$f$で生成される。
  ところで$k$は代数閉体だから、$f=c(x-a_1)\cdots(x-a_n)$と分解される。
  よって任意の閉集合は、
  \[
    Y=Z(\{f\})=Z(\{c(x-a_1)\cdots(x-a_n)\})=\{a_1,\dots,a_n\}
  \]
  となる。すなわち、$\mathbb{A}^1$のZariski位相は、有限個の点(および全体$\mathbb{A}^1$)を閉集合とする位相が入っている。
  言い換えれば、任意の開集合は有限個の点の補集合(あるいは$\varnothing $)である。

  この位相はハウスドルフ(Hausdorff)ではない。
  実際、任意の異なる二点$a,b\in\mathbb{A}^1$をとり、これらを分割する開集合$U_a$、$U_b$があったとすると、$U_a\cap U_b=\varnothing $。
  ところが閉集合$F_a:=\mathbb{A}^1\setminus U_a$と$F_b:=\mathbb{A}^1\setminus U_b$は有限集合であり、従って$F_a\cup F_b$も有限集合。
  ところが$\mathbb{A}^1=k$は代数閉体だから無限集合であるから、
  \[
    \varnothing=U_a\cap U_b=X\setminus(F_a\cup F_b)\neq\varnothing
  \]
  ゆえ矛盾する。
\end{example}
Zariski位相がHausdorffでないことは、この位相が有益な情報をもつコホモロジーを作るほど豊かであるとは言い難いことを示唆しています。
これを解決していくのがこれからのHartshorneの課題の一つです。
第一章ではこの問題はとりあえずおいておいて、先に進んでいきます。
位相を定義したので、そちら方向から多様体の構造に迫っていきます。

\begin{definition}
  位相空間$X$において、空でない部分集合$Y$が\textbf{既約}(\textit{irreducible})であるとは、$Y$の誘導位相による二つの閉かつ真なる部分集合$Y_1,Y_2$であって、$Y=Y_1\cup Y_2$となるものが存在しない時を言う。
  空集合は既約とはみなさない。
\end{definition}
ちょっと何言ってるかわからない人のために、いくつか例を用意してくれています。

\begin{example}[本文 Example1.1.2]
  $\mathbb{A}^1$は既約。
  なぜなら、$\mathbb{A}^1$の閉集合は有限集合しかなく、$\mathbb{A}^1=k$は代数閉体だから無限集合であり、二つの閉集合の和集合で表わしえないため。
\end{example}

\begin{example}[本文 Example1.1.3]
  既約空間の空でない開集合は既約かつ稠密。
\end{example}
証明書いてないのでイミフです。
これは著者の「手を動かせ」というメッセージです。
\begin{proof}
  背理法で証明する。
  $X$を既約空間とし、$Y$を空でない開部分集合とする。
  $Y_1,Y_2\subset Y$を空かつ真でない閉集合であり、$Y=Y_1\cup Y_2$と満たすとする。
  $Y_1,Y_2$は閉集合だから、ある閉集合$C_1,C_2\subset X$が存在して、$Y_1=Y\cap C_1$、$Y_2=Y\cap C_2$。
  このとき、二つの$X$における閉集合
  \begin{align*}
    Z_1&:=C_1\cup(X\setminus Y)\\
    Z_2&:=C_2\cup(X\setminus Y)
  \end{align*}
  を考えると、
  \[
    Z_1\cup Z_2=C_1\cup C_2\cup(X\setminus Y)
  \]
  は$X$に一致する。
  実際、$Y=Y_1\cup Y_2\subset C_1\cup C_2$であるから、$C_1\cup C_2\cup(X\setminus Y)=X$。
  $X$は既約だったから、$Z_1$か$Z_2$のうち少なくとも一方は$X$に一致しなければならない。
  仮に$Z_1=X$だとすると、$Y=X\cap Y=Z_1\cap Y$から
  \begin{align*}
    Y&=X\cap Y=Z_1\cap Y\\
    &=(C_1\cup(X\setminus Y))\cap Y\\
    &=(C_1\cap Y)\cup((X\setminus Y)\cap Y)\\
    &=Y_1\cup\varnothing\\
    &=Y_1
  \end{align*}
  となり、$Y_1$が$Y$の真部分集合であったことに矛盾する。
  $Z_2=X$であっても同様であるから、$Y$は既約でなければならない。

  また、$Y$が稠密であること、すなわち$\overline{Y}=X$を示す。
  閉集合$X\setminus Y$は真部分集合であり、$Y\subset \overline{Y}$であるから、$X$は
  \[
    X=(X\setminus Y)\cup \overline{Y}
  \]
  と閉集合で覆える。
  ところが$X$は既約であり、$X\setminus Y$は真部分集合だから、$\overline{Y}=X$とならなければならない。
\end{proof}

\begin{example}[本文 Example1.1.4]
  $Y$を$X$の既約部分集合とするとき、閉包$\overline{Y}$も$X$の既約部分集合である。
\end{example}
これも証明が書いてないですね。
示していきましょう。
\begin{proof}
  背理法で証明する。
  $\overline{Y}$の真閉集合$Y_1,Y_2$が存在して、$\overline{Y}=Y_1\cup Y_2$を満たすとする。
  $\overline{Y}$は閉集合だから、$Y_1,Y_2$は$X$でも閉集合である。
  また、
  \[
    Y=Y\cap\overline{Y}=Y\cap(Y_1\cup Y_2)=(Y\cap Y_1)\cup(Y\cap Y_2)
  \]
  であることから、$Y$の既約性から$Y=Y\cap Y_1$または$Y=Y\cap Y_2$が成り立たねばならない($Y$の相対位相で考えることに注意)。
  もし$Y=Y\cap Y_1$とすると$Y\subset Y_1$である。
  $\overline{Y}$は$Y$を含む最小の閉集合であったから、$\overline{Y}\subset Y_1\subset\overline{Y}$となり、$Y_1=\overline{Y}$。
  これは$Y_1$が$\overline{Y}$の真部分集合であったことに矛盾する。
  $Y=Y\cap Y_2$としても同様に$Y_2=\overline{Y}$となって矛盾。
\end{proof}

既約集合の感覚が分かったでしょうか?
ぶっちゃけ名前から察しはついているでしょうけれども、素イデアルの代数的集合などが既約集合をなします。
ここからはそんな感じの話だと思います。

\begin{definition}
  $k$を代数閉体、$\mathbb{A}^n$をアフィン空間とする。
  \textbf{アフィン代数多様体}(\textit{affine algenraic variety})とは、$\mathbb{A}^n$の既約閉部分集合に誘導位相を入れたものをいう。
  単に\textbf{アフィン多様体}(\textit{affine variety})とも呼ぶ。
  またアフィン多様体の開部分集合を\textbf{準アフィン多様体}(\textit{quasi-affine variety})と呼ぶ。
\end{definition}

アフィン多様体が代数幾何学における最初の議論対象なのですが、これを代数的に扱うための一つのツールを導入します。
\begin{definition}
  $k$を代数閉体、$Y$を$\mathbb{A}^n=k^n$の部分集合とする。
  このとき、\textbf{$Y$の$A=k[x_1,\dots,x_n]$におけるイデアル}を
  \[
    I(Y):=\{f\in A\mid f(P)=0\;{}^\forall P\in Y\}
  \]
  と定義する。
\end{definition}
もしセミナーなら、誰かから「$I(Y)$が本当にイデアルなのか?」と聞かれてしまうでしょう。
難しくはないですが、一応ちゃんと証明しましょう。
\begin{lemma}
  $Y$を$\mathbb{A}^n=k^n$の部分集合とする。
  このとき、
  \[
    I(Y):=\{f\in A\mid f(P)=0\;{}^\forall P\in Y\}
  \]
  は$A$のイデアルである。
\end{lemma}
\begin{proof}
  $0\in I(Y)$はあきらか。
  $f,g\in I(Y)$のとき、任意の$P\in Y$に対して$f(P)=g(P)=0$であるから、$(f+g)(P)=f(P)+g(P)=0$。
  ゆえに$f+g\in I(Y)$。
  最後に、$a\in A$、$f\in I(Y)$とすると、任意の$P\in Y$に対して$f(P)=0$であるから、$(af)(P)=a(P)f(P)=0$。
  ゆえに$af\in I(Y)$。
\end{proof}

ここに
\begin{align*}
  \text{$A$の部分集合}&\overset{Z}{\longrightarrow}\text{$\mathbb{A}^n$の代数的集合}\\
  \text{$A$のイデアル}&\overset{I}{\longleftarrow}\text{$\mathbb{A}^n$の部分集合}
\end{align*}
という二つの写像が得られました。
これらの間にはきっと素晴らしい関係があるはずです。
いや、あります。見ていきましょう。

\begin{theorem}[本文 Proposition1.2]
  $k$を代数閉体、$\mathbb{A}^n$をアフィン空間、$A=k[x_1,\dots,x_n]$とする。
  このとき、以下が成り立つ。
  \begin{itemize}
    \item[(a)] $T_1\subset T_2$が$A$の部分集合ならば、$Z(T_1)\supset Z(T_2)$
    \item[(b)] $Y_1\subset Y_2$が$\mathbb{A}^n$の部分集合ならば、$I(Y_1)\supset I(Y_2)$
    \item[(c)] $Y_1,Y_2$を$\mathbb{A}^n$の二つの部分集合とすると、$I(Y_1\cup Y_2)=I(Y_1)\cap I(Y_2)$
    \item[(d)] 任意のイデアル$\mathfrak{a}\subset A$に対して、$I(Z(\mathfrak{text}))=\sqrt{\mathfrak{a}}$ ($\sqrt{\mathfrak{a}}$は$\mathfrak{a}$の\textbf{根基})
    \item[(e)] 任意の部分集合$Y\subset\mathbb{A}^n$に対して、$Z(I(Y))=\overline{Y}$
  \end{itemize}
\end{theorem}
\begin{comment}
(a),(b),(c)は自明と言っていますが、僕たちは一応ちゃんと証明しましょう。
(d)は\textbf{Hilbertの零点定理}なるものが必要というか、これはほぼその定理そのものです。
これはまあまあ重い定理なので、後ろのほうを読んでいただければ幸いです。
ゆえに著者がまともに証明を書いているのは(e)だけです。
\end{comment}

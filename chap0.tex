\chapter{Grothendieckが目指したこと}

\section{代数幾何学とは}

代数幾何学という分野は、素朴に言って
\begin{center}
  方程式のゼロ点集合がなす図形の幾何学
\end{center}
です。
例えば
\[
  x^2+y^2=1
\]
という方程式が$\mathbb{R}^2$上に描く図形は円ですし、
\[
  x^2-y^2=1
\]
という方程式が$\mathbb{R}^2$上に描く図形は双曲線と言われます。
こういった「方程式が定める図形」を「方程式の形だけから調べよう」というのが代数幾何学です。

円や放物線のほかに、例えば次のような問題を考えることができます。
\begin{problem}
  $x,y$を実数とする。このとき、
  \[
    (x^2+y^2)^2-(x^2-y^2)=0
  \]
  を満たす$(x,y)$の集合がつくる図形の穴の数は何個あるか?
\end{problem}
これは\textbf{レムニスケート}と呼ばれる、二つの「葉っぱ」がくっついたような図形になっています\footnote{
  この図形は8の字型(レムニスケート)で、原点に特異点を持ちます。トポロジー的な「穴」(1次のホモロジー群の次元、Betti数$b_1$)は、その特異点をどう扱うかで変わってきますが($b_1=2$)、直感的に「2つの領域」を囲んでいると見ることができます。
}。
当然、円$x^2+y^2=1$とは大きく異なる図形となっています。
というところから、方程式の形だけから穴の数を数えることはできるだろうか?という問題が浮かんで当然でしょう。
要するに、
\begin{center}
  \textbf{代数多様体}\footnote{
    いくつかの方程式の解のなす集合のこと。普通は代数閉体で考えますので、今回挙げた円や双曲線、レムニスケートの例は「実代数多様体」といった方が正確かもしれません。
  }を、ある基準で分類したい
\end{center}
ということです。
今回説明に用いている「穴の数」というのは、すなわち\textbf{ホモロジー}ないし、より性質の良い\textbf{コホモロジー}に他なりません。
従って代数幾何学では、まず代数多様体からコホモロジーを作らねばなりません。



\section{曲線や曲面の分類}

複素1次元の(コンパクトな)代数多様体、すなわち\textbf{曲線}は、古典的によく分類されています。
複素1次元ですので、実多様体としては2次元の曲面になっていることに注意しましょう。
非特異な代数曲線は、位相的にはその\textbf{種数}$g$と呼ばれる不変量で完全に分類できることが知られています。
しかしこれに代数多様体としての構造を付与したとき、代数曲線は
\begin{itemize}
  \item $g=0 \implies$ (双有理同値を除けば)一種類のみ
  \item $g=1 \implies$ 1次元のひろがりを持った多様体で"滑らかに"分類できる
  \item $g\geq2 \implies$ $3g-3$次元のひろがりを持った多様体で"滑らかに"分類できる
\end{itemize}
という結果があります。

次元を一つ上げて、\textbf{代数曲面}についてならどうでしょうか?
この場合、代数曲面を種数のような不変量で分類するのは、いきなり難しくなってしまいます。
そこで考え出されたのが、19世紀末から20世紀初頭に活躍したイタリア学派と呼ばれる人々による\textbf{双有理同値}による分類です。
雑に双有理同値を説明すると、
\begin{center}
  ほとんどの部分では同型だけれども、有限個の点をつぶしたり、\textbf{blow-up}(爆発)したりする程度の違いは同じとみなしましょう
\end{center}
「潰れちゃってる点」というのは、しばしば\textbf{特異点}として現れます。
blow-upは双有理幾何学の重要な道具で、大体の特異点はこれで解消されます。
この分類によって、かなり良い結果が存在します。
\begin{itemize}
  \item 任意の代数曲面$X$に対して、ある非特異代数曲面$\widetilde{X}$が存在して、$X$と$\widetilde{X}$は双有理同値
  \item 任意の双有理射$f:X\to Y$は、有限個の点の\textbf{blow-up}および\textbf{blow-down}という操作に分解できる。
\end{itemize}
これだけでもすごいですが、代数曲面については以下の意味で素性の良い双有理同値類が存在することも分かっています。
以下は\textbf{極小モデル理論}と呼ばれます。
\begin{itemize}
  \item ある体$K/k$を固定したとき、$K$を関数体にもつ代数曲面$K(X)$の双有理同値類は、双有理射について半順序集合になる。
  \item $K(\mathbb{P}^2)$や、代数曲線$C$に対して$K(\mathbb{P}^1\times C)$という形の代数曲面には、2つ以上(無限個もありうる)の極小元が存在する。
  \item 上記以外の代数曲面には、唯一の極小元が存在する(\textbf{極小モデル})。
\end{itemize}
このような分類を、より高次元でも行いたいという動機は、例えば現代でも活発な分野である超弦理論の\textbf{ミラー対称性}で現れてきます。
ミラー対称性の主題のひとつは、代数多様体上のある種の\textbf{交叉理論}で、これは微分形式の積分で表されます。
そのためには特異点を除いて解析的道具が使えるようにする必要があるのですが、そのためにblow-upして滑らかな多様体にすることが非常に重要になります。



\section{Riemann-Rochの定理}

複素曲線論における\textbf{Riemann-Rochの定理}は、「方程式が定める図形の穴の数を数える」という興味の逆問題、すなわち、
\begin{center}
  穴の数が決まっているとき、方程式にどのような制限が課されるか?
\end{center}
という問題をほとんど解決する、興味深い結果です。
厳密には次のような主張です。
\begin{theorem}[Riemann-Roch]
  $X$をコンパクトRiemann面(複素1次元正則多様体)、$g$を$X$の種数、$D$を$X$の因子、$K$を$X$の標準因子とする。
  このとき、次の等式が成り立つ。
  \[
    \ell(D)-\ell(K-D)=\deg(D)-g+1
  \]
\end{theorem}
この定理のインパクトを説明するためには、それぞれの用語を説明しなければならないでしょう。
まずコンパクトRiemann面$X$ですが、複素1次元なので、実2次元の広がりを持った図形です。
例えば球面$S^2$やトーラス$T^2=\mathbb{C}/\mathbb{Z}^2$、あるいは$g\geq0$個の浮き輪状の穴が開いた図形はコンパクトRiemann面です(種数$g$というのは、このような「2次元的な」穴の数です)。
Riemann-Rochの定理は、このような図形に対する主張です。

因子$D$とは、$X$上の有限個の点$\{P_i\}_{i=1}^n$に整数分の重み$w_i$をつけた線形和
\[
  D=\sum_{i=1}^nw_iP_i
\]
です。その次数というのは、単に重みを足したものです。
\[
  \deg(D):=\sum_{i=1}^nw_i
\]

$\ell(D)$とは、因子$D$にある意味で縛られている関数全体のなすベクトル空間$L(D)$の次元です。
「ある意味で」というのは、例えば$D=3P-2Q$だとしたら、$L(D)$に含まれる関数は
\[
  f(z)=\frac{v_1}{(z-P)^3}+v_2(z-Q)^2,\quad(v_1,v_2\in\mathbb{C})
\]
などです\footnote{$P,Q$が十分近くにあるとして、$f$はその局所的な関数とみています。}。
\textbf{$\ell(D)$が$X$上の関数の自由度を定めていること}が重要です。

標準因子$K$は若干複雑なので保留しておきます。
$g=0,1$かつ、因子$D$に含まれる重みがすべて正なら必ず$\ell(K-D)=0$なので、今回は$g=0,1$の場合についてより深く見ていきましょう。

\subsection{球面($g=0$)の場合}

種数0のコンパクトRiemann面は球面と同相なのですが、多くの場合\textbf{射影空間}
\[
  X=\mathbb{C}P^1:=\{[x:y]\mid (x,y)\in\mathbb{C}^2\setminus\{(0,0)\}\}
\]
という形で書かれます。
$[x,y]$は、$c\in\mathbb{C}\setminus\{0\}$倍を同一視する組ですよ、という記号です。
つまり
\[
  [cx:cy]=[x:y],\quad{}^\forall (x,y)\in\mathbb{C}^2\setminus\{(0,0)\}, c\in\mathbb{C}
\]
ということです。
$X=\mathbb{C}P^1$は$[x:1]$の周りで
\[
  \{[x:1]\in\mathbb{C}P^1\mid x\in\mathbb{C}\}\cong\mathbb{C}
\]
$[1:y]$の周りで
\[
  \{[1:y]\in\mathbb{C}P^1\mid y\in\mathbb{C}\}\cong\mathbb{C}
\]
という二枚の複素平面で覆うことができ、その共通部分で
\[
  x=\frac1y \iff y=\frac1x
\]
で変数変換できます。
$X=\mathbb{C}P^1$上の関数は定数倍の違いを吸収してくれる$z=x/y$の級数のみが許可されます。
$y=0$のとき$z$は\textbf{無限遠点}$[1:0]$を指し、例えば
\[
  f(z)=\frac1z
\]
のときは無限遠点で$f([1:0])=0$の値を取ります。
ちなみに$f(0)$は0除算となり、通常は定義されませんが、$f(0)=\infty$と定義されているとします。
雑に言って、無限に発散する点を\textbf{極}といいます。
因子$D$の重みが正の項は、関数でいうと極に対応していて、重みは$1/z^w$の次数$w$に対応しています。
重み$w$を、\textbf{極$z=0$の位数}とも言います。

さて今回の場合、Riemann-Rochの定理がいうことは、
\[
  \ell(D)=\deg(D)+1
\]
です\footnote{$D$の重みがすべて正の場合。これを$D\geq0$と書きます。}。
ここでは因子を$D=w\cdot[1:0]$として、$w\in\mathbb{Z}_{\geq0}$を動かしてみて、$X$上で許される関数について調べてみましょう。
この場合、無限遠点$[1:0]$に$w$次の極を持つ関数というのは、とりもなおさず
\[
  f(z)=a_wz^w+a_{w-1}z^{w-1}+\cdots+a_0
\]
という形になっています。
従って、$w$次の極を持つ関数の自由度$\ell(D)$は、高々$w+1$次元です。
これを踏まえてRiemann-Rochの定理を確認してみましょう。
\begin{itemize}
  \item $w=0 \implies \ell(D)=1$: 定数関数全体のなすベクトル空間が1次元。これしかない
  \item $w=1 \implies \ell(D)=2$: $f(z)=az+b$という形の方程式しか許されない。$(a,b)\in\mathbb{C}^2$の2次元
  \item $w=2 \implies \ell(D)=3$: $f(z)=az^2+bz+c$で3次元。
\end{itemize}
以下同様です。
今回は「種数0のときにRiemann-Rochの定理が成り立つこと」を雑に確かめた感じになってしまいました。

\subsection{$g=1$の場合}

種数1のコンパクトRiemann面$X$は、位相的にはトーラスと同相です。
これがすごく面白いです。
種数0の場合と違って、$X$には特別わかりやすい点があるわけじゃないので、適当に一点$P\in X$をとって、因子
\[
  D=w\cdot P,\quad w>0
\]
を考えます。
このときRiemann-Rochの定理は
\[
  \ell(D)=\deg(D)=w
\]
を主張します。
先ほどと同様に$w$をひとつずつ増やしていって、どういう様子か調べてみましょう。
\begin{itemize}
  \item $w=1$: 種数0のときと同様、定数関数のみ。
  \item $w=2$: $P$に2位の極を持つ関数が存在する。これを$x$とおく。\\
  $L(D)$は$\{1,x\}$を基底にもつベクトル空間になる。
  \item $w=3$: $P$に3位の極を持つ関数が存在する。これを$y$とおく。\\
  $L(D)$は$\{1,x,y\}$を基底にもつベクトル空間になる\footnote{$L(D)$は\textbf{高々}$w$次の極をもつ関数の集合であることに注意}。
  \item $w=4$: $L(D)$の基底は$\{1,x,y,x^2\}$。既存の関数を組み合わせるだけでOK。
  \item $w=5$: $L(D)$の基底は$\{1,x,y,x^2,xy\}$。既存の関数を組み合わせるだけでOK。
  \item $w=6$: $L(D)$の\textbf{生成元}は$\{1,x,y,x^2,xy,x^3,y^2\}$。\\
  7個あるが、$\dim L(D)=6$ということは、これらの間に一つの関係式がある。
\end{itemize}
$w=6$の場合に特異な現象が起きています。
つまり
\[
  a_0+a_1x+a_2y+a_3x^2+a_4xy+a_5x^3+a_6y^2=0
\]
となる$a_i$が存在するということです。
これはほぼ\textbf{楕円曲線のワイエルシュトラス標準形}です。
そして種数1のコンパクトRiemann面は、この方程式を満たす点集合が表す図形として表現できるということです。
たまりませんな。

\subsection{一般化したい}\label{chap0-RR-generalize}

幾何学的対象は実のところ、位相的性質は代数トポロジーや微分幾何学、複素幾何学などでかなり分かる部分があります。
例えば$x^2+y^2=1$は円である、というのはもう分かっているのです。
これを複素数の範囲で考えても、(斉次化することで)種数0の曲面になることは知られています。
であれば、種数や因子の次数などの「位相的性質」から先に決めて、その位相的性質を満たしうる「代数方程式」がどのような制約を受けるのか?という問題自体は、高次元の場合でも問いうるものです。

単純な高次元版としては、\textbf{Hirzebruch-Riemann-Rochの定理}として知られているものがあります。
これは準備が大変なので書ききれませんが、とりあえず、高次元の一般化があります。

Grothendieckは、もっと一般の代数多様体についてRiemann-Rochの定理を拡張できないか考えました。
例えば先ほど「実数の範囲で考えていた方程式を、複素数の範囲で考えると…」ということを言いました。
また代数幾何学というのは代数方程式の解の集合がつくる図形の幾何学と始めに言いました。
であれば、代数方程式の解をどの体で考えるかという自由度も同時に考えたいところです。
そういう興味は、例えばFermat予想
\begin{theorem}
  自然数$k\geq3$に対して、$x^k+y^k=1$を満たす有理数$x,y$は、$xy=0$を満たすもの以外に存在しない。
\end{theorem}
に現れています。
つまり、$\mathbb{Q}$上の方程式
\[
  x^k+y^k=1
\]
を幾何学的対象とみて研究したら、Fermat予想が進展するのではないか?というチャレンジです。
実際、Grothendieckが用意したスキームとその上の層、また導来関手などの準備によって、決定的な進展が得られることになったのです。

またGrothendieckは、次に述べる\textbf{Weil予想}を解決するために、極度の一般化を行ったとされます。
Weil予想は、言わずと知れた難問\textbf{Riemann予想}の亜種として考えられたものです。



\section{Riemann予想と素数分布}

Riemann予想を説明するには、$\zeta$関数について説明しなければなりません。
これは、古典も古典問題である\textbf{バーゼル問題}の一般化であると言われます。
バーゼル問題は
\begin{problem}
  \[
    1+\frac1{2^2}+\frac1{3^2}+\cdots
  \]
  は収束するか?収束するなら、どんな値に収束するか?
\end{problem}
という問題です。
これはいくつかの厳密な議論に目をつむれば、簡単に解決して、
\[
  1+\frac1{2^2}+\frac1{3^2}+\cdots=\frac{\pi^2}{6}
\]
になることが知られています。

Eulerはバーゼル問題を解決しただけでなく、自然数の偶数乗の逆数和について解決したとのことです。
つまり
\[
  1+\frac1{2^{2m}}+\frac1{3^{2m}}+\cdots,\quad (m\in\mathbb{Z}_{>0})
\]
の値を求めることに成功しました。

Riemannは「じゃあ$s$が複素数を動いたらどうなるねん?」という着想を得て
\[
  \zeta(s):=1+\frac1{2^s}+\frac1{3^s}+\cdots
\]
という関数を作ってみたようです。
これを\textbf{Riemannのゼータ関数}といいます。
ゼータ関数の最初の重要な性質は、$s=1$に極をもち、それ以外では正則に\textbf{解析接続}できるということです。
また、負の偶整数、すなわち$-2m$ ($m\in\mathbb{Z}_{>0}$)という形の整数で0になるということです。
\[
  \zeta(-2m)=0
\]
これをRiemannは、ゼータ関数の\textbf{自明な零点}と呼びました。

Eulerは既にこのような問題に似たことを考えていたので、以下のような不思議な等式を得ていました。
まず、任意の自然数は素数の積に一意的に分解します。
であれば、
\[
  \left(1+\frac{1}{2^s}+\frac{1}{2^{2s}}+\cdots\right)\left(1+\frac{1}{3^s}+\frac{1}{3^{2s}}+\cdots\right)\left(1+\frac{1}{5^s}+\frac{1}{5^{2s}}+\cdots\right)\cdots
\]
という積を展開すれば、各$n\in\mathbb{Z}_{>0}$に対して$n^{-s}$という項はちょうど1つずつしか現れないはずです。
つまり、
\begin{align*}
  \zeta(s)&=1+\frac1{2^s}+\frac1{3^s}+\cdots\\
  &=\left(1+\frac{1}{2^s}+\frac{1}{2^{2s}}+\cdots\right)\left(1+\frac{1}{3^s}+\frac{1}{3^{2s}}+\cdots\right)\left(1+\frac{1}{5^s}+\frac{1}{5^{2s}}+\cdots\right)\cdots\\
  &=\left(\sum_{n=0}^\infty 2^{-ns}\right)\left(\sum_{n=0}^\infty 3^{-ns}\right)\left(\sum_{n=0}^\infty 5^{-ns}\right)\cdots\\
  &=\frac{1}{1-2^{-s}}\frac{1}{1-3^{-s}}\frac{1}{1-5^{-s}}\cdots\\
  &=\prod_{p:\text{prime}}\frac{1}{1-p^{-s}}
\end{align*}
が成り立ちます\footnote{
  この等式は、実際には$s\in\mathbb{C}$の実部が1より大きいときのみ成り立ちます。
}。

このことからRiemannは、ゼータ関数と素数の関係を結びつけます。
二つの関数を用意します。
\begin{definition}[素数計数関数]
  正の実数$x$に対して、
  \[
    \pi(x)
  \]
  を、$x$以下の素数の個数を返す関数とする。
\end{definition}
\begin{definition}[(第2)Chebyshev関数]
  正の実数$X$に対して、
  \[
    \psi(x):=\sum_{p^k\leq x}\log(p)
  \]
  とする。ただし、和は素数$p$と正の整数$k$全体にわたる。
\end{definition}
Riemannは研究結果として、$x$が1より大かつ、素数べきではないとき、
\[
  \psi(x)=x-\sum_\rho \frac{x^\rho}{\rho} -\log(2\pi) -\frac12\log(1-x^{-2})
\]
という公式を得ました。
ここで第二項の和は、\textbf{ゼータ関数の非自明な零点$\rho$に亘る(重複度を考慮した)和}です。
上式は\textbf{Riemannの明示公式}とも呼ばれます\footnote{
  この形は表現の取り方によって定数項や最後の小さな項の表示が変わります。
}。
証明は追いきれないほどではないのですが、かなり解析的な道具を駆使するので割愛します。

Riemannの明示公式において最も興味がある項は第二項
\[
  \sum_\rho\frac{x^\rho}{\rho}
\]
になります。
ここがどのような挙動をするかによって、$\psi(x)=\sum_{p^k\leq x}\log(p)$の挙動が決まります。
Riemannは複素解析的アプローチによって、
\begin{conjecture}[Riemann予想(RH)]
  ゼータ関数の非自明な零点の実部は$1/2$であろう
\end{conjecture}
と予想しました。
もしそうだとすると、Chebyshev関数は少々複雑な式変形で
\[
  \psi(x)=x+O(x^{1/2}(\log(x))^2)
\]
と評価することができます\footnote{これはちゃんと勉強しないと追えないです。}。
このことから、素数計数関数と(第1,第2)Chebyshev関数の一般論\footnote{これも少々ややこしいです。}から、
\[
  \pi(x)={\rm Li}(x)+O(x^{1/2}\log(x))
\]
という評価を得ます。

既に証明されている素数定理がいうことは、現在知られている最も良い評価で
\begin{theorem}[素数定理(PNT)]
  \[
    \pi(x)={\rm Li}(x) + O\left(x\exp(-0.2098 * \log(x)^{3/5}/\log(\log(x))^{1/5})\right)
  \]
\end{theorem}
となっているそうです\footnote{
  Titchmarsh, \textit{Theory of the Riemann Zeta-function} (Oxford University Press)、またはプレプリント:https://arxiv.org/pdf/1910.08209
}。
Riemann予想から導かれる素数計数関数の評価は、これよりも良いことが、Excelなどを使えば簡単に確かめられます。
実際、$x=100$ぐらいからそれぞれの誤差項を30個ほど抜き出してみると次のようになっています。
なお、以下に示す「誤差項」は、$\pi(x)$が${\rm Li}(x)$との誤差の\textbf{上限}を与えるものであって、\textbf{実際の誤差}はこれよりももっと小さくなることに注意してください。
\begin{table}[!ht]
  \centering
  \begin{tabular}{|l|l|l|}
  \hline
      x & RH誤差項 & PNT誤差項\\\hline\hline
      100 & 46.05170186 & 61.75965223 \\ \hline
      101 & 46.38138717 & 62.34680255 \\ \hline
      102 & 46.70993577 & 62.93369044 \\ \hline
      103 & 47.03736194 & 63.52031875 \\ \hline
      104 & 47.36367965 & 64.10669029 \\ \hline
      105 & 47.68890257 & 64.6928078 \\ \hline
      106 & 48.0130441 & 65.27867396 \\ \hline
      107 & 48.33611731 & 65.86429142 \\ \hline
      108 & 48.65813504 & 66.44966275 \\ \hline
      109 & 48.97910983 & 67.03479047 \\ \hline
  \end{tabular}
\end{table}
こんな感じで、どんどん差は広がっていくばかりです。
$x=1,000,000$とかだと
\begin{itemize}
  \item RH: 13815.5
  \item PNT: 433468.1
\end{itemize}
などとなっていて、文字通り桁が違う誤差が広がっています\footnote{
  $x=1,000,000$のときは$\pi(x)=78,498$。$Li(x)=78,627.549\cdots$なので、$\pi(x)-Li(x)=-129.549\cdots$。
  RHにしろPNTにしろ、実測誤差と上限にはまだまだ差がありますね。
}。
このことから、Riemann予想がいかに重要な定理かが理解できます。



\section{合同ゼータ関数}

整数論の問題は、$\mathbb{Z}$上で考えると難しいけれども、有限体$\mathbb{F}_q$上で考えると何とか解ける、ということがよくあります。
これを\textbf{問題を局所化する}といいます。
例えばDiophantus方程式を解くということは、$\mathbb{Z}$上ではきわめて難しいことですが、有限体で考えたときにどのような解をもちうるかという問題は、\textbf{代数的整数論}という分野でよくまとまっています。
整数の範囲で解けない\textbf{大域的な問題}は、いったん有限体上に局所化して性質を調べてみましょう、というのは、代数的整数論の一つの指針なのです。

例えばFermat予想は、$k$が3以上の整数のとき、
\[
  C_k:x^k+y^k=1
\]
という方程式を満たす非自明な点、つまり$xy\neq0$を満たす点は存在しないだろう、と問い直すことができます。
この問題を少し変えて、$\mathbb{F}_q$上で考えたFermat曲線を$C_k(\mathbb{F}_q)$と書くとき、そこに乗っている点の個数$N_1=\#(C_k(\mathbb{F}_q))$を考えることには単純に興味がわきます。
このように代数多様体上の$\mathbb{F}_q$有理点の個数というのは、Fermat予想を背景にして数学者の興味を引きつづけていました。

この問題を解くために、高校数学でも時々裏技として使用することがある\textbf{母関数}を作って調べる、という手法が考えられました。
その中でも、$N_n:=\#(C_k(\mathbb{F}_{q^n}))$について
\[
  Z(C_k,t):=\exp\left(\sum_{n=1}^\infty\frac{N_n}{n}t^n\right)
\]
という関数は、先述のゼータ関数の類似として、始めは1924年にArtinによって定義されました\footnote{
  もちろん、これによってある特殊な場合において、Riemann予想の類似を証明しています。
}。
そしてSchmidtやHasseによって1930年ごろに研究が進み、1940年末ごろにWeilによって一般化され、\textbf{合同ゼータ関数}と呼ばれるようになります。
Weilの予想は、合同ゼータ関数が実は取り扱いやすい関数になっており、しかもRiemann予想に類似した主張が成り立つだろうということでした。
そのためWeil予想は、20世紀半ばの数学世界の中心に位置する問題の一つとなったのです。
以下では、Weilがどのようにしてゼータ関数の類似として合同ゼータ関数を定義したか、簡単に説明しましょう。

\subsection{代数多様体上の点}

まず、$X$上の「素数」にあたるものを作らなければなりません。
そのために、一旦$X$を$\mathbb{C}$上で考えてみましょう。
このとき、任意の$x\in X$のまわりで
\[
  \mathcal{O}_{X,x}:=\{\left<f,U\right>\mid U:\text{$x$の開近傍}、f:\text{$U$上の関数}\}
\]
という環を考えることができます。
ここで$\left<f,U\right>$という記法は、
\[
  \left<f,U\right>=\left<g,V\right> \iff {}^\exists W:\text{$x$の開近傍 s.t. } f|_W=g|_W
\]
という同一視をしますよ、という記法です。
$\mathcal{O}_{X,x}$は唯一の極大イデアル
\[
  \mathfrak{m}_x:=\{\left<f,U\right>\in\mathcal{O}_{X,x}\mid f(x)=0\}
\]
を持つので、剰余環
\[
  k(x):=\mathcal{O}_{X,x}/\mathfrak{m}_x
\]
は体になります。
$X=X(\mathbb{F}_q)$でもこのアナロジーを考えることができ、$X$の各\textbf{閉点}$x$\footnote{
  現代の代数幾何学では、代数多様体は環から作られるスキームという空間の貼り合わせで定義され、閉点はその極大イデアルに対応します。
}に対して、その剰余体$k(x)$は$\mathbb{F}_q$上の有限次拡大体になっています。
そこで$x\in X$の\textbf{次数}をその拡大次数
\[
  \deg(x):=[k(x):\mathbb{F}_q]
\]
として定義します。

この定義が重要なのは、以下の理由です。
\begin{itemize}
  \item $\deg(x)=1$ の場合:
  $[k(x):\mathbb{F}_q]=1$は$k(x) \cong \mathbb{F}_q$を意味します。
  これは、僕たちが直感的に「$\mathbb{F}_q$ 上の解」と呼ぶ\textbf{$\mathbb{F}_q$有理点}($X(\mathbb{F}_q)$ の点)の厳密な定義と一致します。

  \item $\deg(x) > 1$ の場合:
  これは $k(x) \cong \mathbb{F}_{q^{\deg(x)}}$ となる場合です。
  この閉点 $x$ は、$\mathbb{F}_q$ の世界には座標を持たないため、$\mathbb{F}_q$-有理点ではありません。
  これは、局所的には $\deg(x)$ 次の既約多項式に対応し、$\mathbb{F}_q$ の世界では区別できない「ガロア共役な $\deg(x)$ 個の"根"の集団」が、$X$の中であたかも1つの点 $x$ であるかのように振る舞っている状態を表します。
  これらの"根"は、体を $\mathbb{F}_{q^{\deg(x)}}$ やその拡大体に拡大して初めて、個別の点($\mathbb{F}_{q^n}$-有理点)として姿を現します。
\end{itemize}

$\deg(x) > 1$ の状況は、トポロジーにおける被覆空間と基本群の関係をイメージすると、より深く理解できます。
代数多様体の係数体を $\mathbb{F}_q$ からその代数的閉包 $\overline{\mathbb{F}_q}$ へと拡大した空間 $\overline{X}$ を考えると、これは元の $X$ の「無限被覆空間」のようなものになっています。
そこでは、次数 $n$ の閉点は $n$ 個の幾何学的点に分解しており、それらは被覆変換群(デッキ変換群)としてのガロア群 $\operatorname{Gal}(\overline{\mathbb{F}_q}/\mathbb{F}_q)$ の作用によって互いに入れ替わります。
トポロジーにおける「被覆空間の理論」と代数の「ガロア理論」が驚くほど似ているのは偶然ではなく、このような幾何学的背景によって深く結びついているからなのです。

\subsection{合同ゼータ関数の導入}

閉点$x\in|X|$を素数のアナロジーとみなして、ゼータ関数の完全なアナロジーとして
\[
  Z(X,t):=\prod_{x\in|X|}\frac1{1-t^{\deg(x)}}
\]
を定義しました。
この積の形で書かれた$Z(X,t)$を級数の形に直すには、古典的なゼータ関数でやったそれを逆になぞればよいです。

まず$\log(Z(X,t))$を考えて、$\log(1-x)$のテイラー展開を使うと、
\[
  \log(Z(X,t))=-\sum_{x\in|X|}\log(1-t^{\deg(x)})=\sum_{x\in|X|}\sum_{m=1}^\infty\frac{t^{m\deg(x)}}{m}
\]
を得ます。
ここで和の順序を入れ替えたいのですが、$t$の指数$n$を固定するとその係数は、$t^{m\deg(x)}$という形から
\[
  m\deg(x)=n \iff \deg(x) \mid n
\]
を満たす$x$に亘る和となります。
このようなとき、$1/m=\deg(x)/n$の和がどうなるかというと、
\[
  \sum_{x\in|X|, \deg(x)\mid n}\frac1m=\sum_{x\in|X|, \deg(x)\mid n}\frac{\deg(x)}{n}=\frac1n\sum_{x\in|X|, \deg(x)\mid n}\deg(x)
\]
となります。

最後の和
$$N'_n := \sum_{x\in|X|,\deg(x)\mid n}\deg(x)$$
を考えましょう。
実はこれこそ、冒頭で定義した「安直な解の個数」$N_n=\#(X(\mathbb{F}_{q^n}))$ に厳密に他なりません。

この等式 $N_n = N'_n$ は、スキーム論(とガロア理論)を用いなければ得られない等式ですが、軽く説明しましょう。
$N_n$ は $X(\mathbb{F}_{q^n})$ という$\mathbb{F}_{q^n}$-有理点($P\in X(\mathbb{F}_{q^n})$と書く)の集合の個数です。
一方 $N'_n$ は $|X|$ という閉点($x$ と書く)の集合を使って定義されています。
これら2つの異なる集合は、以下の厳密な対応関係によって結びつきます。
\begin{enumerate}
  \item 全ての $\mathbb{F}_{q^n}$-有理点 $P \in X(\mathbb{F}_{q^n})$ は、その「土台」となるただ一つの閉点 $x \in |X|$ に(自然に)対応します。
  \item このとき、$P$ の剰余体($\cong \mathbb{F}_{q^n}$)は $x$ の剰余体 $k(x) \cong \mathbb{F}_{q^d}$ ($d=\deg(x)$) の拡大体であり、従って $d \mid n$ です。
  \item 逆に、 $d \mid n$ を満たす1個の閉点 $x$(次数 $d$)を考えると、ガロア理論から、 $k(x)$ を $\mathbb{F}_{q^n}$ に埋め込む方法はちょうど $d$ 個あります。
  \item この $d$ 個の異なる埋め込みが、 $x$ に対応する $d$ 個の異なる $\mathbb{F}_{q^n}$-有理点 $P_1, \dots, P_d$ を生み出します。
\end{enumerate}
したがって、「$\mathbb{F}_{q^n}$-有理点の総数」を数えることは、「$d \mid n$ を満たす閉点 $x$ をすべて見つけ、それぞれが提供する $d = \deg(x)$ 個の点を合計する」ことと完全に同値です。
よって
$$N_n=\#(X(\mathbb{F}_{q^n}))=\sum_{x\in|X|,\deg(x)\mid n}\deg(x)$$
が成り立つということです!

というわけで収束性は一旦おいておいて\footnote{
  代数的には\textbf{形式的べき級数環}で考えていることになります。
}、和の順序を入れ替えると
\begin{align*}
  \log(Z(X,t))=\sum_{n=1}^\infty\frac{t^n}{n}\sum_{x\in|X|,\deg(x)\mid n}\deg(x)=\sum_{n=1}^\infty\frac{N_nt^n}{n}
\end{align*}
となり、あとは両辺の指数関数をとれば、
\[
  Z(X,t)=\exp\left(\sum_{n=1}^\infty\frac{N_nt^n}{n}\right)
\]
が得られるというわけです。

\subsection{合同ゼータ関数の計算例 - 射影空間}

合同ゼータ関数の性質を知るために、簡単な例で計算してみましょう。
射影直線$X=\mathbb{P}_{\mathbb{F}_q}^1$を考えましょう。
射影直線に乗っている点の個数は、単に$\mathbb{F}_q$の個数に無限遠点を付け加えた$q+1$個です。
つまり
\[
  N_1=q+1
\]
です。
$X(\mathbb{F}_{q^n})=\mathbb{P}_{\mathbb{F}_{q^n}}^1$に関しても全く同様で、
\[
  N_n=q^n+1
\]
となります。
ゆえに合同ゼータ関数は
\[
  Z(\mathbb{P}_{\mathbb{F}_q},t)=\exp\left(\sum_{n=1}^\infty\frac{(q^n+1)t^n}{n}\right)
\]
となります。
ここで、対数関数の基本的なテイラー展開
\begin{align*}
  \log(1-x)=-\sum_{n=1}^\infty\frac{x^n}{n}
\end{align*}
を使うと、
\begin{align*}
  Z(\mathbb{P}_{\mathbb{F}_q},t)&=\exp\left(\sum_{n=1}^\infty\frac{(q^n+1)t^n}{n}\right)\\
  &=\exp\left(\sum_{n=1}^\infty\frac{q^nt^n}{n}+\sum_{n=1}^\infty\frac{q^nt^n}{n}\right)\\
  &=\exp\left(-\log(1-qt)-\log(1-t)\right)\\
  &=\frac1{(1-qt)(1-t)}
\end{align*}
となります。
ここまで複雑な定義を経てきましたが、そのわりにえらくシンプルな数式が出てきましたね。

\subsection{合同ゼータ関数の計算例 - 円}

$\mathbb{F}_q$上の
\[
  C_2:x^2+y^2=1
\]
を満たす曲線を考えましょう。
ただし、$q=2$だと
\[
  x^2+y^2-1=(x+y+1)^2
\]
と因数分解してしまって面倒ですので、$q>2$とします。

いきなり$C_2(\mathbb{F}_{q^n})$の点の個数を考えましょう。
高校数学でやったことがあるかもしれませんが、点$(-1,0)$を通る直線との交点を数えることで、$N_n$が計算できるはずです。
つまるところ
\[
  \mathbb{F}_{q^n}\setminus\{a\in\mathbb{F}_{q^n}\mid a^2\neq-1\}\to C_2(\mathbb{F}_{q^n})\setminus\{(-1,0)\};a\mapsto\left(\frac{1-a^2}{1+a^2},\frac{2a}{1+a^2}\right)
\]
が全単射になっています。
従って、
\[
  N_n=q^n+1-\#\{a\in\mathbb{F}_{q^n}\mid a^2\neq-1\}
\]
となっています。
$\#\{a\in\mathbb{F}_{q^n}\mid a^2\neq-1\}$は厄介そうですが、こんなことはガウスあたりがとっくにやりつくしてます。
次のような結果が残されています。
まず素数$q>2$は4で割って1余るか、または3余るしかないことに注意してください。
このことによる分類があります。
\begin{table}[!ht]
  \centering
  \begin{tabular}{|l|l|}
  \hline
      条件 & $\#\{a^2=-1\}$\\\hline\hline
      $q\equiv1 \pmod4$ & 2 \\ \hline
      $q\equiv3 \pmod4$ かつ $n$ が偶数 & 2 \\ \hline
      $q\equiv3 \pmod4$ かつ $n$ が奇数 & 0 \\ \hline
  \end{tabular}
\end{table}
上記の表から、$q$が4で割って1あまる素数のときは、合同ゼータ関数は単純で、
\begin{align*}
  Z(C_2,t)&=\exp\left(\sum_{n=1}^\infty\frac{(q^n-1)t^n}{n}\right)\\
  &=\exp\left(\sum_{n=1}^\infty\frac{q^nt^n}{n}-\sum_{n=1}^\infty\frac{t^n}{n}\right)\\
  &=\exp\left(-\log(1-qt)+\log(1-t)\right)\\
  &=\frac{1-t}{1-qt}
\end{align*}
となります。
$q$が4で割って3あまる素数のときは、$n$を偶奇に分けて考えねばなりません。
\begin{align*}
  Z(C_2,t)&=\exp\left(\sum_{n=1}^\infty\frac{N_nt^n}{n}\right)\\
  &=\exp\left(\sum_{n=1}^\infty\frac{N_{2n-1}t^{2n-1}}{2n-1}+\sum_{n=1}^\infty\frac{N_{2n}t^{2n}}{2n}\right)\\
  &=\exp\left(\sum_{n=1}^\infty\frac{(q^{2n-1}+1)t^{2n-1}}{2n-1}+\sum_{n=1}^\infty\frac{(q^{2n}-1)t^{2n}}{2n}\right)\\
  &=\exp\left(\sum_{n=1}^\infty\frac{q^{2n-1}t^{2n-1}}{2n-1}+\sum_{n=1}^\infty\frac{q^{2n}t^{2n}}{2n}+\sum_{n=1}^\infty\frac{t^{2n-1}}{2n-1}+\sum_{n=1}^\infty\frac{-t^{2n}}{2n}\right)\\
  &=\exp\left(\sum_{n=1}^\infty\frac{q^nt^n}{n}-\sum_{n=1}^\infty\frac{(-t)^n}{n}\right)\\
  &=\exp(-\log(1-qt)+\log(1+t))\\
  &=\frac{1+t}{1-qt}
\end{align*}

\subsection{Frobenius写像とLefschetzの不動点定理}

$q$を素数、$n>0$を整数とするとき、
\[
  {\rm Frob}_q^n:\overline{\mathbb{F}_q}\to\overline{\mathbb{F}_q};x\mapsto x^{q^n}
\]
は体準同型となります($\overline{\mathbb{F}_q}$は$\mathbb{F}_q$の代数閉包)。
これを\textbf{Frobenius写像}(を$n$回合成したもの)といいます。
この写像の不動点の集合
\[
  \{x\in\overline{\mathbb{F}_q}\mid x=x^{q^n}\}
\]
は、多項式$X^{q^n}-X$の根の集合に一致する等のことから、実は$\mathbb{F}_{q^n}$に一致します(ちょい雑ですが正しいです)。

このことは$\mathbb{F}_{q^n}$上の代数多様体上でも言えます!
つまり、
\[
  {\rm Frob}_q^n:X(\overline{\mathbb{F}_q})\to X(\overline{\mathbb{F}_q})
\]
の不動点を考えると、これはちょうど$X(\mathbb{F}_{q^n})$に一致します。
ここに、\textbf{代数多様体の有理点を数えるには、Frobenius写像の不動点を数えればよい}という発想の転換が得られます。
そして写像の不動点を数えるというジャストな知見は、\textbf{多様体論(manifold)}でまとまっていました。

同じ多様体同士の可微分写像に対して、その不動点を求める問題はPoincar\'{e}が天体の三体問題を考えていたときから数学者の興味を引いてきました。
Lefschetzの不動点定理を述べるために、特異ホモロジーについて簡単に復習しましょう。
標準$r$単体とは、
\[
  \Delta_r:=\{(x_1,\dots,x_r)\in\mathbb{R}^r\mid x_i\geq0({}^\forall i), x_1+\cdots+x_r\leq 1\}
\]
をいいます。
多様体$M$の特異$r$単体とは、連続写像
\[
  \sigma:\Delta_r\to M
\]
のことをいいます。
特異$r$単体の実係数線形和の集合を
\[
  C_r(M):=\left\{\sum_{i=1}^sa_i\sigma_i\mid a_i\in\mathbb{R}, \sigma_i:\Delta_r\to M\right\}
\]
と書きます。
このとき、特異$r$単体$\sigma$の境界作用素$\partial_r(\sigma)$が次のように定義されます。
\[
  \partial(\sigma)=\sum_{j=0}^r(-1)^j\sigma_{0,\dots,j-1,j+1,\dots,r}
\]
ここで$\sigma_{0,\dots,j-1,j+1,\dots,r}$は、$j$番目の座標のみ1の点$(0,\dots,0,1,0,\dots,0)\in\Delta_r$を含まない$\Delta_r$の面に、$\sigma$を制限した特異$r-1$単体です。
ただし$j=0$のときは原点を含まない面への制限とします。
境界作用素を線形に拡張して、$\partial:C_r(M)\to C_{r-1}(M)$を考えると、系列
\[
  \cdots\to C_{r+1}(M)\overset{\partial}{\to}C_r(M)\overset{\partial}{\to}C_{r-1}\to\cdots
\]
は$\partial\circ\partial=0$を満たすので、
\[
  {\rm Im}(\partial:C_{r+1}(M)\to C_r(M))\subset{\rm Ker}(\partial:C_r(M)\to C_{r-1}(M))
\]
が成り立ちます。
なので、
\[
  H_r(M):={\rm Ker}(\partial:C_r(M)\to C_{r-1}(M))/{\rm Im}(\partial:C_{r+1}(M)\to C_r(M))
\]
という群を考えることができます。
これが$M$の\textbf{ホモロジー群}というものです。
今回は係数を$\mathbb{R}$にとっておいたので、これはベクトル空間の構造をもちます。

さて、$f:M\to M$を連続写像とすると、任意の特異$r$単体$\sigma:\Delta_r\to M$に対して
\[
  f\circ\sigma:\Delta_r\to M\to M
\]
という別の特異$r$単体を考えることができます。
これを$C_r(M)$上に線形に拡大してみると、境界作用素との間に
\[
  f\circ\partial(\sigma)=\sum_{j=0}^r(-1)^jf\circ\sigma_{0,1,\dots,j-1,j+1,\dots,r}=\partial(f\circ\sigma)
\]
が成り立っていることから、ホモロジー群への群準同型
\[
  f_*:H_r(M)\to H_r(M);[\sigma]\mapsto[f\circ\sigma]
\]
がうまく定義されます。
これは定義から明らかに$\mathbb{R}$線形写像になっています。
Lefschetzの不動点定理の文脈では、ホモロジーよりも便利な\textbf{コホモロジー}$H^r(M)$を考えることが多く、その際$f:M\to M$が誘導する写像
\[
  f^*:H^r(M)\to H^r(M)
\]
を考えることもできます。

さて$M$がコンパクトな多様体であれば、$H^r(M)$は有限次元です。
そこで$H^r(M)$の適当な基底をとれば、$f^*$は行列で表現できます。
このとき、その行列のトレース${\rm Tr}(f^*)$は、線形代数学の簡単な帰結として、基底の取り方に依存しないのでした。

これでLefshetzの不動点定理の準備が整いました。
\begin{theorem}[Lefschetzの不動点定理]
  $X$を多様体、$f:M\to M$を連続写像とする。
  $\Gamma:=\{(x,f(x))\mid x\in M\}\subset M\times M$を$f$のグラフとする。
  $\Delta:=\{(x,x)\mid x\in X\}\subset X\times X$を対角集合とする。
  このとき、次が成り立つ。
  \[
    \#(\Gamma\cap\Delta)=\sum_{r}(-1)^r{\rm Tr}(f^*)
  \]
  ただし$\#(\Gamma\cap\Delta)$は、は各交点に向き($\pm1$の重み)を付けて数えた個数を表す。
\end{theorem}

話を代数多様体に戻しますと、Frobenius写像の不動点の個数が$\mathbb{F}_{q^n}$で考えた有理点の個数と一致するのでした。
ただしここでは、\textbf{有限体上の代数多様体に対して、Lefschetzの不動点定理が成り立つ程度のコホモロジーが定義できる}としています。
いま、これが、Frobenius写像が$X$上のコホモロジーに誘導された写像のトレースを調べることに帰着しました。
どういうことかというと、まず$({\rm Frob}_q^*)^n$を$H^r(X)$に対する${\rm Frob}_q^*$の$n$回作用としたとき、
\[
  N_n=\sum_{r}(-1)^r{\rm Tr}(({\rm Frob}_q^*)^n)
\]
が成り立つということです。
これを用いて合同ゼータ関数を計算してみましょう。
\begin{align*}
  \log(Z(X,t))&=\sum_{n=1}^\infty\frac{N_nt^n}{n}\\
  &=\sum_{n=1}^\infty\sum_{r}(-1)^r\frac{{\rm Tr}(({\rm Frob}_q^*)^n)t^n}{n}\\
  &=\sum_r(-1)^r\sum_{n=1}^\infty\frac{{\rm Tr}(({\rm Frob}_q^*)^n)t^n}{n}\\
\end{align*}
さて
\[
  \sum_{n=1}^\infty\frac{{\rm Tr}(({\rm Frob}_q^*)^n)t^n}{n}
\]
の部分ですが、もっと綺麗に整理できます。
まず行列とみた線形写像$({\rm Frob}_q^*)^n$のトレースというのは、その固有値の和と一致するのでした。
これらを$\lambda_1,\dots,\lambda_m$とすると、$({\rm Frob}_q^*)^n$は${\rm Frob}_q^*$を$n$回合成したものですから、その固有値は$\lambda_1^n,\dots,\lambda_m^n$となります。
従って
\[
  {\rm Tr}(({\rm Frob}_q^*)^n)=\lambda_1^n+\cdots+\lambda_m^n
\]
となります。
ゆえに
\begin{align*}
  &\sum_{n=1}^\infty\frac{{\rm Tr}(({\rm Frob}_q^*)^n)t^n}{n}\\
  =&\sum_{n=1}^\infty\frac{(\lambda_1^n+\cdots+\lambda_m^n)t^n}{n}\\
  =&-\log(1-\lambda_1t)-\cdots-\log(1-\lambda_mt)\\
  =&-\log\left(\prod_{j=1}^m(1-\lambda_jt)\right)
\end{align*}
そして$1-({\rm Frob}_q^*)^nt$という行列を対角化して行列式をとったと考えれば、
\[
  \sum_{n=1}^\infty\frac{{\rm Tr}(({\rm Frob}_q^*)^n)t^n}{n}=-\log\left(\det(1-({\rm Frob}_q^*)^nt)\right)
\]
と書けます。
ここで、$t$の多項式$\det(1-({\rm Frob}_q^*)^n)t)$を
\[
  P_r(t):=\det(1-({\rm Frob}_q^*)^n)t)
\]
とおくと、結局
\begin{align*}
  \log(Z(X,t))&=\sum_{n=1}^\infty\frac{N_nt^n}{n}\\
  &=\sum_r(-1)^{r+1}\log(P_r(t))\\
\end{align*}
となり、
\[
  Z(X,t)=\frac{P_1(t)P_3(t)\cdots P_{2d-1}(t)}{P_0(t)P_2(t)\cdots P_{2d}(t)}
\]
が\textbf{予想}されます。
この予想には、
\begin{center}
  Frobenius写像に対するLefschetzの不動点定理が成り立つ程度の\\
  代数多様体上のコホモロジー理論がないといけない
\end{center}
という大きな障害があります。



\section{Weil予想}

代数多様体上のコホモロジーを考えるということは、そのトポロジーの構造を考えるということです。
Weil当時、代数多様体に入っている位相というのは、$\mathbb{R}$や$\mathbb{C}$上の滑らかなmanifoldでない限り、\textbf{Zariski位相}と呼ばれる貧弱な位相しかありませんでした。
これはHausdorffですらなく、直感的な「図形」を表しているとすら考えづらいです。
そこで、Lefschetz不動点定理が機能するようなコホモロジー理論がいずれ構築されるだろうという期待を込めつつ、Weilは以下の予想を提案しました。

\begin{conjecture}
  $X$を体$k=\mathbb{F}_q$上の非特異射影代数多様体とする。
  このとき、以下が成り立つ。
  \begin{enumerate}
    \item \textbf{有理性} 合同ゼータ関数$Z(X,t)$は
    \[
      Z(X,t)=\frac{P_1(t)P_3(t)\cdots P_{2d-1}(t)}{P_0(t)P_2(t)\cdots P_{2d}(t)}
    \]
    という形の有理多項式である。
    ここで$P_i(t)$は整数係数多項式である。
    \item \textbf{Riemann予想のアナロジー} 上記の表式において、$\mathbb{C}$上
    \[P_i(t)=\prod_i(1-\alpha_{ij}t)\]
    とおくと、$\alpha_{ij}$は代数的整数であって、$|\alpha_{ij}|=q^{i/2}$を満たす。
    \item \textbf{Betti数について} $X$が複素射影代数多様体とするとき、以下が成り立つ。
    \[
      b_i(X):=\dim_{\mathbb{R}}H^i(X)=\deg(P_i(t))
    \]
    \item \textbf{関数等式} $\chi:=\sum_i(-1)^i\deg(P_i(t))$を$X$のEuler標数とする。
    このとき、以下の関係式が成り立つ。
    \[
      Z\left(X,\frac1{q^dt}\right)=\pm q^{d\chi/2}t^\chi Z(X,t)
    \]
  \end{enumerate}
\end{conjecture}

\subsection{総括}

ここまでWeil予想に至る道を知り、ある程度数学を知っていると、かなりとんでもなく都合のいいことを言っているように見えるのです。
僕は幾何系なので、特にBetti数のあたりがやばすぎると思いました。
数論としてはおそらく、Riemann予想のアナロジーが成り立ってしまうことにインパクトがあるのでしょう。
関数等式は、この問題に関数解析の知識が必要であることを示唆しているものと思われます。
とにかく数学のありとあらゆる分野の道具なら何でも使いそうです。

ともかく、Weil予想には有限体上の代数多様体であっても、Lefschetzの不動点定理が成り立つぐらい「豊富な」コホモロジー理論が必要です。
多分そのためにGrothendieckはスキームというものを、代数幾何学を行う「場」として定義しました。

Weil予想にチャレンジするにはLefschetzの不動点定理が必要です。
これは交点理論なので、点同士の交わり方がどうなっているかを議論できる場でなければなりません。
そのために、単純な方程式の解の集合として取り扱うのではなく、\textbf{係数体を拡大してみたときに始めてその根として現れる点}を考慮しなければなりません。
この既約多項式が、係数拡大したときに新たに表れる点の次数なのでした。
そのため、代数多様体を単なる方程式いくつかの解の集合ととらえるのではなく、\textbf{素イデアルが代数多様体を定めている}といったことに近い発想が芽生えます。

たぶん決定的なことは、将来示す次の定理です。
\begin{theorem}
  代数閉体$k$上のアフィン多様体$X$\footnote{
    つまり、有限個の方程式の解で定められた$\mathbb{A}_k^n=k^n$上の集合。
  }に対して、
  \[
    A(X)=k[x_1,\dots,x_n]/I(X),\quad I(X):=\{f\in k[x_1,\dots,x_n]\mid f(P)=0 ({}^\forall P\in X)\}
  \]
  を対応させる対応は、「$k$上のアフィン多様体の圏」から「有限生成$k$代数であるような整域」への反変な圏同型を誘導する。
\end{theorem}
この事実を$k$が代数閉体でない場合に一般化しようとすると、スキームをどう定義するべきかという問いに自然な答えを与えます。
Grothendieckのアイデアは、$X$ という「点の集合」を一度忘れ、\textbf{$A(X)$ という「環」こそが本質である}と考えることでした。
これによってアフィン多様体の対応物を、代数閉体 $k$ 上だけでなく、$\mathbb{F}_q$ や $\mathbb{Q}$、さらには一般の環 $R$ といった、より一般の「基底」の上で展開しようと試みました。
その結果として導入されたのがスキーム(Scheme)です。
最も基本的なスキームは、任意の環 $A$ に対して定義されるアフィンスキーム $\operatorname{Spec}(A)$ です。
$\operatorname{Spec}(A)$ の「点」の集合は、$A$ の素イデアル $\mathfrak{p}$ 全体の集合として定義されます。
$$ X = \operatorname{Spec}(A) := \{ \mathfrak{p} \mid \mathfrak{p} \text{ は } A \text{ の素イデアル} \}$$
私たちがこれまで「閉点」と呼んできたものは、この定義における極大イデアルに対応します。
スキームでは、極大イデアルではない素イデアル(例えば $\mathbb{F}_q[T]$ における $(0)$ イデアルなど)も、多様体全体の情報を担う「生成点」として、幾何学的な「点」の一つとみなします。
そして、一般のスキームとは、ちょうど多様体がEuclid空間の一部を貼り合わせて定義されるように、これらのアフィンスキーム $\operatorname{Spec}(A_i)$ を「貼り合わせる」ことで得られる、極めて広範な幾何学的対象として定義されます。
この「スキーム」という新しい「場」を導入したことで、
\begin{itemize}
\item $\mathbb{F}_q$ 上の代数多様体(Weil予想の舞台)
\item $\mathbb{Q}$ 上の代数多様体(Fermat予想の舞台)
\item $\mathbb{Z}$ 上の代数多様体(Diophantus方程式そのもの)
\end{itemize}
といった、従来は異なる分野(数論、幾何学)で扱われていた対象を、すべて「スキーム」という単一の言語で統一的に記述することが可能になりました。
そして、この広大な基盤の上でこそ、Grothendieckは「エタールコホモロジー」と呼ばれる、Weil予想を解決するに足る強力なコホモロジー理論を構築することに成功したのです。

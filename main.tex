\documentclass[dvipdfmx, 11pt, oneside, openany]{jsbook}

\usepackage{amsmath}
\usepackage{amsfonts}
\usepackage{amsthm}
\usepackage{amssymb}
\usepackage{comment}
\usepackage{tikz-cd}

% BibLaTeXの設定
\usepackage[style=numeric]{biblatex}
\addbibresource{references.bib} % 参考文献ファイルの指定
% 出力イメージ: Hartshorne, "Algebraic Geometry" [1]
\newcommand{\mycite}[1]{%
  \citeauthor{#1}, ``\citetitle{#1}''~\cite{#1}%
}

\theoremstyle{definition}
\newtheorem{theorem}{定理}[section]
\newtheorem{corollary}[theorem]{系}
\newtheorem{definition}[theorem]{定義}
\newtheorem{lemma}[theorem]{補題}
\newtheorem{problem}[theorem]{問題}
\newtheorem{example}[theorem]{例}
\newtheorem{conjecture}[theorem]{予想}
\newtheorem{remark}{補足}
\renewcommand{\proofname}{\textbf{証明}}

\setlength{\textwidth}{\fullwidth}
\setcounter{chapter}{-1}

\title{【精読ノート】Robin Hartshorne著\\\textit{Algebraic Geometry}}
\author{ずんだもん博士}
\date{\today}

\begin{document}
  \maketitle
  \tableofcontents
  \chapter*{序論}

代数幾何学の古典的名著、HartshoneのAlgebraic Geometryを精読します。
この本は1977年出版ということで、若干古いものの、現在でも代数幾何学を勉強するときに多くの学生がこれを参考に勉強をしています。
本書の技術的なコアはおそらく第3章のCohomologyなのですが、僕は今まで何度もこの本にアタックして、第2章のSchemeでつまずいています。
この原因は、まず僕が英語苦手すぎなこと。
次に可換環論について理解が浅いことだと思います。
結果として僕は可換環論の本をたくさん買いましたが、多くの重要な結果が互いに演習問題にしあっていたり、導入の動機が不明な概念を多く含んでいることに目を白黒させているうちに、読了をあきらめていました。
今回はこれを頑張って精読するという狂気のプロジェクトを立ち上げました。

この本のPrefaceには、Hartshone先生がバークレー大学で行った大学院向けの授業で使っていることが書かれています。
日本の大学だと、これほど丁寧に代数幾何について講義を受けられる機会は少ないので、うらやましいですね。
いや、これは代数幾何の基礎が整った今だからかもしれません。
今だとHartshone、Liu、あるいは邦書の代数幾何学の本がいくつかあるので、それを師匠からポンと渡されて、「これでセミナーしてください」って言われてボコボコにされながら読む感じがします。
それを僕は、数学科から還俗してお気楽に一人でやろうっていうことです。

じゃあ、やってみましょう。

  \chapter{Grothendieckが目指したこと}

\section{代数幾何学とは}

代数幾何学という分野は、素朴に言って
\begin{center}
  方程式のゼロ点集合がなす図形の幾何学
\end{center}
です。
例えば
\[
  x^2+y^2=1
\]
という方程式が$\mathbb{R}^2$上に描く図形は円ですし、
\[
  x^2-y^2=1
\]
という方程式が$\mathbb{R}^2$上に描く図形は双曲線と言われます。
こういった「方程式が定める図形」を「方程式の形だけから調べよう」というのが代数幾何学です。

円や放物線のほかに、例えば次のような問題を考えることができます。
\begin{problem}
  $x,y$を実数とする。このとき、
  \[
    (x^2+y^2)^2-(x^2-y^2)=0
  \]
  を満たす$(x,y)$の集合がつくる図形の穴の数は何個あるか?
\end{problem}
これは\textbf{レムニスケート}と呼ばれる、二つの「葉っぱ」がくっついたような図形になっています\footnote{
  この図形は8の字型(レムニスケート)で、原点に特異点を持ちます。トポロジー的な「穴」(1次のホモロジー群の次元、Betti数$b_1$)は、その特異点をどう扱うかで変わってきますが($b_1=2$)、直感的に「2つの領域」を囲んでいると見ることができます。
}。
当然、円$x^2+y^2=1$とは大きく異なる図形となっています。
というところから、方程式の形だけから穴の数を数えることはできるだろうか?という問題が浮かんで当然でしょう。
要するに、
\begin{center}
  \textbf{代数多様体}\footnote{
    いくつかの方程式の解のなす集合のこと。普通は代数閉体で考えますので、今回挙げた円や双曲線、レムニスケートの例は「実代数多様体」といった方が正確かもしれません。
  }を、ある基準で分類したい
\end{center}
ということです。
今回説明に用いている「穴の数」というのは、すなわち\textbf{ホモロジー}ないし、より性質の良い\textbf{コホモロジー}に他なりません。
従って代数幾何学では、まず代数多様体からコホモロジーを作らねばなりません。



\section{曲線や曲面の分類}

複素1次元の(コンパクトな)代数多様体、すなわち\textbf{曲線}は、古典的によく分類されています。
複素1次元ですので、実多様体としては2次元の曲面になっていることに注意しましょう。
非特異な代数曲線は、位相的にはその\textbf{種数}$g$と呼ばれる不変量で完全に分類できることが知られています。
しかしこれに代数多様体としての構造を付与したとき、代数曲線は
\begin{itemize}
  \item $g=0 \implies$ (双有理同値を除けば)一種類のみ
  \item $g=1 \implies$ 1次元のひろがりを持った多様体で"滑らかに"分類できる
  \item $g\geq2 \implies$ $3g-3$次元のひろがりを持った多様体で"滑らかに"分類できる
\end{itemize}
という結果があります。

次元を一つ上げて、\textbf{代数曲面}についてならどうでしょうか?
この場合、代数曲面を種数のような不変量で分類するのは、いきなり難しくなってしまいます。
そこで考え出されたのが、19世紀末から20世紀初頭に活躍したイタリア学派と呼ばれる人々による\textbf{双有理同値}による分類です。
雑に双有理同値を説明すると、
\begin{center}
  ほとんどの部分では同型だけれども、有限個の点をつぶしたり、\textbf{blow-up}(爆発)したりする程度の違いは同じとみなしましょう
\end{center}
「潰れちゃってる点」というのは、しばしば\textbf{特異点}として現れます。
blow-upは双有理幾何学の重要な道具で、大体の特異点はこれで解消されます。
この分類によって、かなり良い結果が存在します。
\begin{itemize}
  \item 任意の代数曲面$X$に対して、ある非特異代数曲面$\widetilde{X}$が存在して、$X$と$\widetilde{X}$は双有理同値
  \item 任意の双有理射$f:X\to Y$は、有限個の点の\textbf{blow-up}および\textbf{blow-down}という操作に分解できる。
\end{itemize}
これだけでもすごいですが、代数曲面については以下の意味で素性の良い双有理同値類が存在することも分かっています。
以下は\textbf{極小モデル理論}と呼ばれます。
\begin{itemize}
  \item ある体$K/k$を固定したとき、$K$を関数体にもつ代数曲面$K(X)$の双有理同値類は、双有理射について半順序集合になる。
  \item $K(\mathbb{P}^2)$や、代数曲線$C$に対して$K(\mathbb{P}^1\times C)$という形の代数曲面には、2つ以上(無限個もありうる)の極小元が存在する。
  \item 上記以外の代数曲面には、唯一の極小元が存在する(\textbf{極小モデル})。
\end{itemize}
このような分類を、より高次元でも行いたいという動機は、例えば現代でも活発な分野である超弦理論の\textbf{ミラー対称性}で現れてきます。
ミラー対称性の主題のひとつは、代数多様体上のある種の\textbf{交叉理論}で、これは微分形式の積分で表されます。
そのためには特異点を除いて解析的道具が使えるようにする必要があるのですが、そのためにblow-upして滑らかな多様体にすることが非常に重要になります。



\section{Riemann-Rochの定理}

複素曲線論における\textbf{Riemann-Rochの定理}は、「方程式が定める図形の穴の数を数える」という興味の逆問題、すなわち、
\begin{center}
  穴の数が決まっているとき、方程式にどのような制限が課されるか?
\end{center}
という問題をほとんど解決する、興味深い結果です。
厳密には次のような主張です。
\begin{theorem}[Riemann-Roch]
  $X$をコンパクトRiemann面(複素1次元正則多様体)、$g$を$X$の種数、$D$を$X$の因子、$K$を$X$の標準因子とする。
  このとき、次の等式が成り立つ。
  \[
    \ell(D)-\ell(K-D)=\deg(D)-g+1
  \]
\end{theorem}
この定理のインパクトを説明するためには、それぞれの用語を説明しなければならないでしょう。
まずコンパクトRiemann面$X$ですが、複素1次元なので、実2次元の広がりを持った図形です。
例えば球面$S^2$やトーラス$T^2=\mathbb{C}/\mathbb{Z}^2$、あるいは$g\geq0$個の浮き輪状の穴が開いた図形はコンパクトRiemann面です(種数$g$というのは、このような「2次元的な」穴の数です)。
Riemann-Rochの定理は、このような図形に対する主張です。

因子$D$とは、$X$上の有限個の点$\{P_i\}_{i=1}^n$に整数分の重み$w_i$をつけた線形和
\[
  D=\sum_{i=1}^nw_iP_i
\]
です。その次数というのは、単に重みを足したものです。
\[
  \deg(D):=\sum_{i=1}^nw_i
\]

$\ell(D)$とは、因子$D$にある意味で縛られている関数全体のなすベクトル空間$L(D)$の次元です。
「ある意味で」というのは、例えば$D=3P-2Q$だとしたら、$L(D)$に含まれる関数は
\[
  f(z)=\frac{v_1}{(z-P)^3}+v_2(z-Q)^2,\quad(v_1,v_2\in\mathbb{C})
\]
などです\footnote{$P,Q$が十分近くにあるとして、$f$はその局所的な関数とみています。}。
\textbf{$\ell(D)$が$X$上の関数の自由度を定めていること}が重要です。

標準因子$K$は若干複雑なので保留しておきます。
$g=0,1$かつ、因子$D$に含まれる重みがすべて正なら必ず$\ell(K-D)=0$なので、今回は$g=0,1$の場合についてより深く見ていきましょう。

\subsection{球面($g=0$)の場合}

種数0のコンパクトRiemann面は球面と同相なのですが、多くの場合\textbf{射影空間}
\[
  X=\mathbb{C}P^1:=\{[x:y]\mid (x,y)\in\mathbb{C}^2\setminus\{(0,0)\}\}
\]
という形で書かれます。
$[x,y]$は、$c\in\mathbb{C}\setminus\{0\}$倍を同一視する組ですよ、という記号です。
つまり
\[
  [cx:cy]=[x:y],\quad{}^\forall (x,y)\in\mathbb{C}^2\setminus\{(0,0)\}, c\in\mathbb{C}
\]
ということです。
$X=\mathbb{C}P^1$は$[x:1]$の周りで
\[
  \{[x:1]\in\mathbb{C}P^1\mid x\in\mathbb{C}\}\cong\mathbb{C}
\]
$[1:y]$の周りで
\[
  \{[1:y]\in\mathbb{C}P^1\mid y\in\mathbb{C}\}\cong\mathbb{C}
\]
という二枚の複素平面で覆うことができ、その共通部分で
\[
  x=\frac1y \iff y=\frac1x
\]
で変数変換できます。
$X=\mathbb{C}P^1$上の関数は定数倍の違いを吸収してくれる$z=x/y$の級数のみが許可されます。
$y=0$のとき$z$は\textbf{無限遠点}$[1:0]$を指し、例えば
\[
  f(z)=\frac1z
\]
のときは無限遠点で$f([1:0])=0$の値を取ります。
ちなみに$f(0)$は0除算となり、通常は定義されませんが、$f(0)=\infty$と定義されているとします。
雑に言って、無限に発散する点を\textbf{極}といいます。
因子$D$の重みが正の項は、関数でいうと極に対応していて、重みは$1/z^w$の次数$w$に対応しています。
重み$w$を、\textbf{極$z=0$の位数}とも言います。

さて今回の場合、Riemann-Rochの定理がいうことは、
\[
  \ell(D)=\deg(D)+1
\]
です\footnote{$D$の重みがすべて正の場合。これを$D\geq0$と書きます。}。
ここでは因子を$D=w\cdot[1:0]$として、$w\in\mathbb{Z}_{\geq0}$を動かしてみて、$X$上で許される関数について調べてみましょう。
この場合、無限遠点$[1:0]$に$w$次の極を持つ関数というのは、とりもなおさず
\[
  f(z)=a_wz^w+a_{w-1}z^{w-1}+\cdots+a_0
\]
という形になっています。
従って、$w$次の極を持つ関数の自由度$\ell(D)$は、高々$w+1$次元です。
これを踏まえてRiemann-Rochの定理を確認してみましょう。
\begin{itemize}
  \item $w=0 \implies \ell(D)=1$: 定数関数全体のなすベクトル空間が1次元。これしかない
  \item $w=1 \implies \ell(D)=2$: $f(z)=az+b$という形の方程式しか許されない。$(a,b)\in\mathbb{C}^2$の2次元
  \item $w=2 \implies \ell(D)=3$: $f(z)=az^2+bz+c$で3次元。
\end{itemize}
以下同様です。
今回は「種数0のときにRiemann-Rochの定理が成り立つこと」を雑に確かめた感じになってしまいました。

\subsection{$g=1$の場合}

種数1のコンパクトRiemann面$X$は、位相的にはトーラスと同相です。
これがすごく面白いです。
種数0の場合と違って、$X$には特別わかりやすい点があるわけじゃないので、適当に一点$P\in X$をとって、因子
\[
  D=w\cdot P,\quad w>0
\]
を考えます。
このときRiemann-Rochの定理は
\[
  \ell(D)=\deg(D)=w
\]
を主張します。
先ほどと同様に$w$をひとつずつ増やしていって、どういう様子か調べてみましょう。
\begin{itemize}
  \item $w=1$: 種数0のときと同様、定数関数のみ。
  \item $w=2$: $P$に2位の極を持つ関数が存在する。これを$x$とおく。\\
  $L(D)$は$\{1,x\}$を基底にもつベクトル空間になる。
  \item $w=3$: $P$に3位の極を持つ関数が存在する。これを$y$とおく。\\
  $L(D)$は$\{1,x,y\}$を基底にもつベクトル空間になる\footnote{$L(D)$は\textbf{高々}$w$次の極をもつ関数の集合であることに注意}。
  \item $w=4$: $L(D)$の基底は$\{1,x,y,x^2\}$。既存の関数を組み合わせるだけでOK。
  \item $w=5$: $L(D)$の基底は$\{1,x,y,x^2,xy\}$。既存の関数を組み合わせるだけでOK。
  \item $w=6$: $L(D)$の\textbf{生成元}は$\{1,x,y,x^2,xy,x^3,y^2\}$。\\
  7個あるが、$\dim L(D)=6$ということは、これらの間に一つの関係式がある。
\end{itemize}
$w=6$の場合に特異な現象が起きています。
つまり
\[
  a_0+a_1x+a_2y+a_3x^2+a_4xy+a_5x^3+a_6y^2=0
\]
となる$a_i$が存在するということです。
これはほぼ\textbf{楕円曲線のワイエルシュトラス標準形}です。
そして種数1のコンパクトRiemann面は、この方程式を満たす点集合が表す図形として表現できるということです。
たまりませんな。

\subsection{一般化したい}\label{chap0-RR-generalize}

幾何学的対象は実のところ、位相的性質は代数トポロジーや微分幾何学、複素幾何学などでかなり分かる部分があります。
例えば$x^2+y^2=1$は円である、というのはもう分かっているのです。
これを複素数の範囲で考えても、(斉次化することで)種数0の曲面になることは知られています。
であれば、種数や因子の次数などの「位相的性質」から先に決めて、その位相的性質を満たしうる「代数方程式」がどのような制約を受けるのか?という問題自体は、高次元の場合でも問いうるものです。

単純な高次元版としては、\textbf{Hirzebruch-Riemann-Rochの定理}として知られているものがあります。
これは準備が大変なので書ききれませんが、とりあえず、高次元の一般化があります。

Grothendieckは、もっと一般の代数多様体についてRiemann-Rochの定理を拡張できないか考えました。
例えば先ほど「実数の範囲で考えていた方程式を、複素数の範囲で考えると…」ということを言いました。
また代数幾何学というのは代数方程式の解の集合がつくる図形の幾何学と始めに言いました。
であれば、代数方程式の解をどの体で考えるかという自由度も同時に考えたいところです。
そういう興味は、例えばFermat予想
\begin{theorem}
  自然数$k\geq3$に対して、$x^k+y^k=1$を満たす有理数$x,y$は、$xy=0$を満たすもの以外に存在しない。
\end{theorem}
に現れています。
つまり、$\mathbb{Q}$上の方程式
\[
  x^k+y^k=1
\]
を幾何学的対象とみて研究したら、Fermat予想が進展するのではないか?というチャレンジです。
実際、Grothendieckが用意したスキームとその上の層、また導来関手などの準備によって、決定的な進展が得られることになったのです。

またGrothendieckは、次に述べる\textbf{Weil予想}を解決するために、極度の一般化を行ったとされます。
Weil予想は、言わずと知れた難問\textbf{Riemann予想}の亜種として考えられたものです。



\section{Riemann予想と素数分布}

Riemann予想を説明するには、$\zeta$関数について説明しなければなりません。
これは、古典も古典問題である\textbf{バーゼル問題}の一般化であると言われます。
バーゼル問題は
\begin{problem}
  \[
    1+\frac1{2^2}+\frac1{3^2}+\cdots
  \]
  は収束するか?収束するなら、どんな値に収束するか?
\end{problem}
という問題です。
これはいくつかの厳密な議論に目をつむれば、簡単に解決して、
\[
  1+\frac1{2^2}+\frac1{3^2}+\cdots=\frac{\pi^2}{6}
\]
になることが知られています。

Eulerはバーゼル問題を解決しただけでなく、自然数の偶数乗の逆数和について解決したとのことです。
つまり
\[
  1+\frac1{2^{2m}}+\frac1{3^{2m}}+\cdots,\quad (m\in\mathbb{Z}_{>0})
\]
の値を求めることに成功しました。

Riemannは「じゃあ$s$が複素数を動いたらどうなるねん?」という着想を得て
\[
  \zeta(s):=1+\frac1{2^s}+\frac1{3^s}+\cdots
\]
という関数を作ってみたようです。
これを\textbf{Riemannのゼータ関数}といいます。
ゼータ関数の最初の重要な性質は、$s=1$に極をもち、それ以外では正則に\textbf{解析接続}できるということです。
また、負の偶整数、すなわち$-2m$ ($m\in\mathbb{Z}_{>0}$)という形の整数で0になるということです。
\[
  \zeta(-2m)=0
\]
これをRiemannは、ゼータ関数の\textbf{自明な零点}と呼びました。

Eulerは既にこのような問題に似たことを考えていたので、以下のような不思議な等式を得ていました。
まず、任意の自然数は素数の積に一意的に分解します。
であれば、
\[
  \left(1+\frac{1}{2^s}+\frac{1}{2^{2s}}+\cdots\right)\left(1+\frac{1}{3^s}+\frac{1}{3^{2s}}+\cdots\right)\left(1+\frac{1}{5^s}+\frac{1}{5^{2s}}+\cdots\right)\cdots
\]
という積を展開すれば、各$n\in\mathbb{Z}_{>0}$に対して$n^{-s}$という項はちょうど1つずつしか現れないはずです。
つまり、
\begin{align*}
  \zeta(s)&=1+\frac1{2^s}+\frac1{3^s}+\cdots\\
  &=\left(1+\frac{1}{2^s}+\frac{1}{2^{2s}}+\cdots\right)\left(1+\frac{1}{3^s}+\frac{1}{3^{2s}}+\cdots\right)\left(1+\frac{1}{5^s}+\frac{1}{5^{2s}}+\cdots\right)\cdots\\
  &=\left(\sum_{n=0}^\infty 2^{-ns}\right)\left(\sum_{n=0}^\infty 3^{-ns}\right)\left(\sum_{n=0}^\infty 5^{-ns}\right)\cdots\\
  &=\frac{1}{1-2^{-s}}\frac{1}{1-3^{-s}}\frac{1}{1-5^{-s}}\cdots\\
  &=\prod_{p:\text{prime}}\frac{1}{1-p^{-s}}
\end{align*}
が成り立ちます\footnote{
  この等式は、実際には$s\in\mathbb{C}$の実部が1より大きいときのみ成り立ちます。
}。

このことからRiemannは、ゼータ関数と素数の関係を結びつけます。
二つの関数を用意します。
\begin{definition}[素数計数関数]
  正の実数$x$に対して、
  \[
    \pi(x)
  \]
  を、$x$以下の素数の個数を返す関数とする。
\end{definition}
\begin{definition}[(第2)Chebyshev関数]
  正の実数$X$に対して、
  \[
    \psi(x):=\sum_{p^k\leq x}\log(p)
  \]
  とする。ただし、和は素数$p$と正の整数$k$全体にわたる。
\end{definition}
Riemannは研究結果として、$x$が1より大かつ、素数べきではないとき、
\[
  \psi(x)=x-\sum_\rho \frac{x^\rho}{\rho} -\log(2\pi) -\frac12\log(1-x^{-2})
\]
という公式を得ました。
ここで第二項の和は、\textbf{ゼータ関数の非自明な零点$\rho$に亘る(重複度を考慮した)和}です。
上式は\textbf{Riemannの明示公式}とも呼ばれます\footnote{
  この形は表現の取り方によって定数項や最後の小さな項の表示が変わります。
}。
証明は追いきれないほどではないのですが、かなり解析的な道具を駆使するので割愛します。

Riemannの明示公式において最も興味がある項は第二項
\[
  \sum_\rho\frac{x^\rho}{\rho}
\]
になります。
ここがどのような挙動をするかによって、$\psi(x)=\sum_{p^k\leq x}\log(p)$の挙動が決まります。
Riemannは複素解析的アプローチによって、
\begin{conjecture}[Riemann予想(RH)]
  ゼータ関数の非自明な零点の実部は$1/2$であろう
\end{conjecture}
と予想しました。
もしそうだとすると、Chebyshev関数は少々複雑な式変形で
\[
  \psi(x)=x+O(x^{1/2}(\log(x))^2)
\]
と評価することができます\footnote{これはちゃんと勉強しないと追えないです。}。
このことから、素数計数関数と(第1,第2)Chebyshev関数の一般論\footnote{これも少々ややこしいです。}から、
\[
  \pi(x)={\rm Li}(x)+O(x^{1/2}\log(x))
\]
という評価を得ます。

既に証明されている素数定理がいうことは、現在知られている最も良い評価で
\begin{theorem}[素数定理(PNT)]
  \[
    \pi(x)={\rm Li}(x) + O\left(x\exp(-0.2098 * \log(x)^{3/5}/\log(\log(x))^{1/5})\right)
  \]
\end{theorem}
となっているそうです\footnote{
  Titchmarsh, \textit{Theory of the Riemann Zeta-function} (Oxford University Press)、またはプレプリント:https://arxiv.org/pdf/1910.08209
}。
Riemann予想から導かれる素数計数関数の評価は、これよりも良いことが、Excelなどを使えば簡単に確かめられます。
実際、$x=100$ぐらいからそれぞれの誤差項を30個ほど抜き出してみると次のようになっています。
なお、以下に示す「誤差項」は、$\pi(x)$が${\rm Li}(x)$との誤差の\textbf{上限}を与えるものであって、\textbf{実際の誤差}はこれよりももっと小さくなることに注意してください。
\begin{table}[!ht]
  \centering
  \begin{tabular}{|l|l|l|}
  \hline
      x & RH誤差項 & PNT誤差項\\\hline\hline
      100 & 46.05170186 & 61.75965223 \\ \hline
      101 & 46.38138717 & 62.34680255 \\ \hline
      102 & 46.70993577 & 62.93369044 \\ \hline
      103 & 47.03736194 & 63.52031875 \\ \hline
      104 & 47.36367965 & 64.10669029 \\ \hline
      105 & 47.68890257 & 64.6928078 \\ \hline
      106 & 48.0130441 & 65.27867396 \\ \hline
      107 & 48.33611731 & 65.86429142 \\ \hline
      108 & 48.65813504 & 66.44966275 \\ \hline
      109 & 48.97910983 & 67.03479047 \\ \hline
  \end{tabular}
\end{table}
こんな感じで、どんどん差は広がっていくばかりです。
$x=1,000,000$とかだと
\begin{itemize}
  \item RH: 13815.5
  \item PNT: 433468.1
\end{itemize}
などとなっていて、文字通り桁が違う誤差が広がっています\footnote{
  $x=1,000,000$のときは$\pi(x)=78,498$。$Li(x)=78,627.549\cdots$なので、$\pi(x)-Li(x)=-129.549\cdots$。
  RHにしろPNTにしろ、実測誤差と上限にはまだまだ差がありますね。
}。
このことから、Riemann予想がいかに重要な定理かが理解できます。



\section{合同ゼータ関数}

整数論の問題は、$\mathbb{Z}$上で考えると難しいけれども、有限体$\mathbb{F}_q$上で考えると何とか解ける、ということがよくあります。
これを\textbf{問題を局所化する}といいます。
例えばDiophantus方程式を解くということは、$\mathbb{Z}$上ではきわめて難しいことですが、有限体で考えたときにどのような解をもちうるかという問題は、\textbf{代数的整数論}という分野でよくまとまっています。
整数の範囲で解けない\textbf{大域的な問題}は、いったん有限体上に局所化して性質を調べてみましょう、というのは、代数的整数論の一つの指針なのです。

例えばFermat予想は、$k$が3以上の整数のとき、
\[
  C_k:x^k+y^k=1
\]
という方程式を満たす非自明な点、つまり$xy\neq0$を満たす点は存在しないだろう、と問い直すことができます。
この問題を少し変えて、$\mathbb{F}_q$上で考えたFermat曲線を$C_k(\mathbb{F}_q)$と書くとき、そこに乗っている点の個数$N_1=\#(C_k(\mathbb{F}_q))$を考えることには単純に興味がわきます。
このように代数多様体上の$\mathbb{F}_q$有理点の個数というのは、Fermat予想を背景にして数学者の興味を引きつづけていました。

この問題を解くために、高校数学でも時々裏技として使用することがある\textbf{母関数}を作って調べる、という手法が考えられました。
その中でも、$N_n:=\#(C_k(\mathbb{F}_{q^n}))$について
\[
  Z(C_k,t):=\exp\left(\sum_{n=1}^\infty\frac{N_n}{n}t^n\right)
\]
という関数は、先述のゼータ関数の類似として、始めは1924年にArtinによって定義されました\footnote{
  もちろん、これによってある特殊な場合において、Riemann予想の類似を証明しています。
}。
そしてSchmidtやHasseによって1930年ごろに研究が進み、1940年末ごろにWeilによって一般化され、\textbf{合同ゼータ関数}と呼ばれるようになります。
Weilの予想は、合同ゼータ関数が実は取り扱いやすい関数になっており、しかもRiemann予想に類似した主張が成り立つだろうということでした。
そのためWeil予想は、20世紀半ばの数学世界の中心に位置する問題の一つとなったのです。
以下では、Weilがどのようにしてゼータ関数の類似として合同ゼータ関数を定義したか、簡単に説明しましょう。

\subsection{代数多様体上の点}

まず、$X$上の「素数」にあたるものを作らなければなりません。
そのために、一旦$X$を$\mathbb{C}$上で考えてみましょう。
このとき、任意の$x\in X$のまわりで
\[
  \mathcal{O}_{X,x}:=\{\left<f,U\right>\mid U:\text{$x$の開近傍}、f:\text{$U$上の関数}\}
\]
という環を考えることができます。
ここで$\left<f,U\right>$という記法は、
\[
  \left<f,U\right>=\left<g,V\right> \iff {}^\exists W:\text{$x$の開近傍 s.t. } f|_W=g|_W
\]
という同一視をしますよ、という記法です。
$\mathcal{O}_{X,x}$は唯一の極大イデアル
\[
  \mathfrak{m}_x:=\{\left<f,U\right>\in\mathcal{O}_{X,x}\mid f(x)=0\}
\]
を持つので、剰余環
\[
  k(x):=\mathcal{O}_{X,x}/\mathfrak{m}_x
\]
は体になります。
$X=X(\mathbb{F}_q)$でもこのアナロジーを考えることができ、$X$の各\textbf{閉点}$x$\footnote{
  現代の代数幾何学では、代数多様体は環から作られるスキームという空間の貼り合わせで定義され、閉点はその極大イデアルに対応します。
}に対して、その剰余体$k(x)$は$\mathbb{F}_q$上の有限次拡大体になっています。
そこで$x\in X$の\textbf{次数}をその拡大次数
\[
  \deg(x):=[k(x):\mathbb{F}_q]
\]
として定義します。

この定義が重要なのは、以下の理由です。
\begin{itemize}
  \item $\deg(x)=1$ の場合:
  $[k(x):\mathbb{F}_q]=1$は$k(x) \cong \mathbb{F}_q$を意味します。
  これは、僕たちが直感的に「$\mathbb{F}_q$ 上の解」と呼ぶ\textbf{$\mathbb{F}_q$有理点}($X(\mathbb{F}_q)$ の点)の厳密な定義と一致します。

  \item $\deg(x) > 1$ の場合:
  これは $k(x) \cong \mathbb{F}_{q^{\deg(x)}}$ となる場合です。
  この閉点 $x$ は、$\mathbb{F}_q$ の世界には座標を持たないため、$\mathbb{F}_q$-有理点ではありません。
  これは、局所的には $\deg(x)$ 次の既約多項式に対応し、$\mathbb{F}_q$ の世界では区別できない「ガロア共役な $\deg(x)$ 個の"根"の集団」が、$X$の中であたかも1つの点 $x$ であるかのように振る舞っている状態を表します。
  これらの"根"は、体を $\mathbb{F}_{q^{\deg(x)}}$ やその拡大体に拡大して初めて、個別の点($\mathbb{F}_{q^n}$-有理点)として姿を現します。
\end{itemize}

$\deg(x) > 1$ の状況は、トポロジーにおける被覆空間と基本群の関係をイメージすると、より深く理解できます。
代数多様体の係数体を $\mathbb{F}_q$ からその代数的閉包 $\overline{\mathbb{F}_q}$ へと拡大した空間 $\overline{X}$ を考えると、これは元の $X$ の「無限被覆空間」のようなものになっています。
そこでは、次数 $n$ の閉点は $n$ 個の幾何学的点に分解しており、それらは被覆変換群(デッキ変換群)としてのガロア群 $\operatorname{Gal}(\overline{\mathbb{F}_q}/\mathbb{F}_q)$ の作用によって互いに入れ替わります。
トポロジーにおける「被覆空間の理論」と代数の「ガロア理論」が驚くほど似ているのは偶然ではなく、このような幾何学的背景によって深く結びついているからなのです。

\subsection{合同ゼータ関数の導入}

閉点$x\in|X|$を素数のアナロジーとみなして、ゼータ関数の完全なアナロジーとして
\[
  Z(X,t):=\prod_{x\in|X|}\frac1{1-t^{\deg(x)}}
\]
を定義しました。
この積の形で書かれた$Z(X,t)$を級数の形に直すには、古典的なゼータ関数でやったそれを逆になぞればよいです。

まず$\log(Z(X,t))$を考えて、$\log(1-x)$のテイラー展開を使うと、
\[
  \log(Z(X,t))=-\sum_{x\in|X|}\log(1-t^{\deg(x)})=\sum_{x\in|X|}\sum_{m=1}^\infty\frac{t^{m\deg(x)}}{m}
\]
を得ます。
ここで和の順序を入れ替えたいのですが、$t$の指数$n$を固定するとその係数は、$t^{m\deg(x)}$という形から
\[
  m\deg(x)=n \iff \deg(x) \mid n
\]
を満たす$x$に亘る和となります。
このようなとき、$1/m=\deg(x)/n$の和がどうなるかというと、
\[
  \sum_{x\in|X|, \deg(x)\mid n}\frac1m=\sum_{x\in|X|, \deg(x)\mid n}\frac{\deg(x)}{n}=\frac1n\sum_{x\in|X|, \deg(x)\mid n}\deg(x)
\]
となります。

最後の和
$$N'_n := \sum_{x\in|X|,\deg(x)\mid n}\deg(x)$$
を考えましょう。
実はこれこそ、冒頭で定義した「安直な解の個数」$N_n=\#(X(\mathbb{F}_{q^n}))$ に厳密に他なりません。

この等式 $N_n = N'_n$ は、スキーム論(とガロア理論)を用いなければ得られない等式ですが、軽く説明しましょう。
$N_n$ は $X(\mathbb{F}_{q^n})$ という$\mathbb{F}_{q^n}$-有理点($P\in X(\mathbb{F}_{q^n})$と書く)の集合の個数です。
一方 $N'_n$ は $|X|$ という閉点($x$ と書く)の集合を使って定義されています。
これら2つの異なる集合は、以下の厳密な対応関係によって結びつきます。
\begin{enumerate}
  \item 全ての $\mathbb{F}_{q^n}$-有理点 $P \in X(\mathbb{F}_{q^n})$ は、その「土台」となるただ一つの閉点 $x \in |X|$ に(自然に)対応します。
  \item このとき、$P$ の剰余体($\cong \mathbb{F}_{q^n}$)は $x$ の剰余体 $k(x) \cong \mathbb{F}_{q^d}$ ($d=\deg(x)$) の拡大体であり、従って $d \mid n$ です。
  \item 逆に、 $d \mid n$ を満たす1個の閉点 $x$(次数 $d$)を考えると、ガロア理論から、 $k(x)$ を $\mathbb{F}_{q^n}$ に埋め込む方法はちょうど $d$ 個あります。
  \item この $d$ 個の異なる埋め込みが、 $x$ に対応する $d$ 個の異なる $\mathbb{F}_{q^n}$-有理点 $P_1, \dots, P_d$ を生み出します。
\end{enumerate}
したがって、「$\mathbb{F}_{q^n}$-有理点の総数」を数えることは、「$d \mid n$ を満たす閉点 $x$ をすべて見つけ、それぞれが提供する $d = \deg(x)$ 個の点を合計する」ことと完全に同値です。
よって
$$N_n=\#(X(\mathbb{F}_{q^n}))=\sum_{x\in|X|,\deg(x)\mid n}\deg(x)$$
が成り立つということです!

というわけで収束性は一旦おいておいて\footnote{
  代数的には\textbf{形式的べき級数環}で考えていることになります。
}、和の順序を入れ替えると
\begin{align*}
  \log(Z(X,t))=\sum_{n=1}^\infty\frac{t^n}{n}\sum_{x\in|X|,\deg(x)\mid n}\deg(x)=\sum_{n=1}^\infty\frac{N_nt^n}{n}
\end{align*}
となり、あとは両辺の指数関数をとれば、
\[
  Z(X,t)=\exp\left(\sum_{n=1}^\infty\frac{N_nt^n}{n}\right)
\]
が得られるというわけです。

\subsection{合同ゼータ関数の計算例 - 射影空間}

合同ゼータ関数の性質を知るために、簡単な例で計算してみましょう。
射影直線$X=\mathbb{P}_{\mathbb{F}_q}^1$を考えましょう。
射影直線に乗っている点の個数は、単に$\mathbb{F}_q$の個数に無限遠点を付け加えた$q+1$個です。
つまり
\[
  N_1=q+1
\]
です。
$X(\mathbb{F}_{q^n})=\mathbb{P}_{\mathbb{F}_{q^n}}^1$に関しても全く同様で、
\[
  N_n=q^n+1
\]
となります。
ゆえに合同ゼータ関数は
\[
  Z(\mathbb{P}_{\mathbb{F}_q},t)=\exp\left(\sum_{n=1}^\infty\frac{(q^n+1)t^n}{n}\right)
\]
となります。
ここで、対数関数の基本的なテイラー展開
\begin{align*}
  \log(1-x)=-\sum_{n=1}^\infty\frac{x^n}{n}
\end{align*}
を使うと、
\begin{align*}
  Z(\mathbb{P}_{\mathbb{F}_q},t)&=\exp\left(\sum_{n=1}^\infty\frac{(q^n+1)t^n}{n}\right)\\
  &=\exp\left(\sum_{n=1}^\infty\frac{q^nt^n}{n}+\sum_{n=1}^\infty\frac{q^nt^n}{n}\right)\\
  &=\exp\left(-\log(1-qt)-\log(1-t)\right)\\
  &=\frac1{(1-qt)(1-t)}
\end{align*}
となります。
ここまで複雑な定義を経てきましたが、そのわりにえらくシンプルな数式が出てきましたね。

\subsection{合同ゼータ関数の計算例 - 円}

$\mathbb{F}_q$上の
\[
  C_2:x^2+y^2=1
\]
を満たす曲線を考えましょう。
ただし、$q=2$だと
\[
  x^2+y^2-1=(x+y+1)^2
\]
と因数分解してしまって面倒ですので、$q>2$とします。

いきなり$C_2(\mathbb{F}_{q^n})$の点の個数を考えましょう。
高校数学でやったことがあるかもしれませんが、点$(-1,0)$を通る直線との交点を数えることで、$N_n$が計算できるはずです。
つまるところ
\[
  \mathbb{F}_{q^n}\setminus\{a\in\mathbb{F}_{q^n}\mid a^2\neq-1\}\to C_2(\mathbb{F}_{q^n})\setminus\{(-1,0)\};a\mapsto\left(\frac{1-a^2}{1+a^2},\frac{2a}{1+a^2}\right)
\]
が全単射になっています。
従って、
\[
  N_n=q^n+1-\#\{a\in\mathbb{F}_{q^n}\mid a^2\neq-1\}
\]
となっています。
$\#\{a\in\mathbb{F}_{q^n}\mid a^2\neq-1\}$は厄介そうですが、こんなことはガウスあたりがとっくにやりつくしてます。
次のような結果が残されています。
まず素数$q>2$は4で割って1余るか、または3余るしかないことに注意してください。
このことによる分類があります。
\begin{table}[!ht]
  \centering
  \begin{tabular}{|l|l|}
  \hline
      条件 & $\#\{a^2=-1\}$\\\hline\hline
      $q\equiv1 \pmod4$ & 2 \\ \hline
      $q\equiv3 \pmod4$ かつ $n$ が偶数 & 2 \\ \hline
      $q\equiv3 \pmod4$ かつ $n$ が奇数 & 0 \\ \hline
  \end{tabular}
\end{table}
上記の表から、$q$が4で割って1あまる素数のときは、合同ゼータ関数は単純で、
\begin{align*}
  Z(C_2,t)&=\exp\left(\sum_{n=1}^\infty\frac{(q^n-1)t^n}{n}\right)\\
  &=\exp\left(\sum_{n=1}^\infty\frac{q^nt^n}{n}-\sum_{n=1}^\infty\frac{t^n}{n}\right)\\
  &=\exp\left(-\log(1-qt)+\log(1-t)\right)\\
  &=\frac{1-t}{1-qt}
\end{align*}
となります。
$q$が4で割って3あまる素数のときは、$n$を偶奇に分けて考えねばなりません。
\begin{align*}
  Z(C_2,t)&=\exp\left(\sum_{n=1}^\infty\frac{N_nt^n}{n}\right)\\
  &=\exp\left(\sum_{n=1}^\infty\frac{N_{2n-1}t^{2n-1}}{2n-1}+\sum_{n=1}^\infty\frac{N_{2n}t^{2n}}{2n}\right)\\
  &=\exp\left(\sum_{n=1}^\infty\frac{(q^{2n-1}+1)t^{2n-1}}{2n-1}+\sum_{n=1}^\infty\frac{(q^{2n}-1)t^{2n}}{2n}\right)\\
  &=\exp\left(\sum_{n=1}^\infty\frac{q^{2n-1}t^{2n-1}}{2n-1}+\sum_{n=1}^\infty\frac{q^{2n}t^{2n}}{2n}+\sum_{n=1}^\infty\frac{t^{2n-1}}{2n-1}+\sum_{n=1}^\infty\frac{-t^{2n}}{2n}\right)\\
  &=\exp\left(\sum_{n=1}^\infty\frac{q^nt^n}{n}-\sum_{n=1}^\infty\frac{(-t)^n}{n}\right)\\
  &=\exp(-\log(1-qt)+\log(1+t))\\
  &=\frac{1+t}{1-qt}
\end{align*}

\subsection{Frobenius写像とLefschetzの不動点定理}

$q$を素数、$n>0$を整数とするとき、
\[
  {\rm Frob}_q^n:\overline{\mathbb{F}_q}\to\overline{\mathbb{F}_q};x\mapsto x^{q^n}
\]
は体準同型となります($\overline{\mathbb{F}_q}$は$\mathbb{F}_q$の代数閉包)。
これを\textbf{Frobenius写像}(を$n$回合成したもの)といいます。
この写像の不動点の集合
\[
  \{x\in\overline{\mathbb{F}_q}\mid x=x^{q^n}\}
\]
は、多項式$X^{q^n}-X$の根の集合に一致する等のことから、実は$\mathbb{F}_{q^n}$に一致します(ちょい雑ですが正しいです)。

このことは$\mathbb{F}_{q^n}$上の代数多様体上でも言えます!
つまり、
\[
  {\rm Frob}_q^n:X(\overline{\mathbb{F}_q})\to X(\overline{\mathbb{F}_q})
\]
の不動点を考えると、これはちょうど$X(\mathbb{F}_{q^n})$に一致します。
ここに、\textbf{代数多様体の有理点を数えるには、Frobenius写像の不動点を数えればよい}という発想の転換が得られます。
そして写像の不動点を数えるというジャストな知見は、\textbf{多様体論(manifold)}でまとまっていました。

同じ多様体同士の可微分写像に対して、その不動点を求める問題はPoincar\'{e}が天体の三体問題を考えていたときから数学者の興味を引いてきました。
Lefschetzの不動点定理を述べるために、特異ホモロジーについて簡単に復習しましょう。
標準$r$単体とは、
\[
  \Delta_r:=\{(x_1,\dots,x_r)\in\mathbb{R}^r\mid x_i\geq0({}^\forall i), x_1+\cdots+x_r\leq 1\}
\]
をいいます。
多様体$M$の特異$r$単体とは、連続写像
\[
  \sigma:\Delta_r\to M
\]
のことをいいます。
特異$r$単体の実係数線形和の集合を
\[
  C_r(M):=\left\{\sum_{i=1}^sa_i\sigma_i\mid a_i\in\mathbb{R}, \sigma_i:\Delta_r\to M\right\}
\]
と書きます。
このとき、特異$r$単体$\sigma$の境界作用素$\partial_r(\sigma)$が次のように定義されます。
\[
  \partial(\sigma)=\sum_{j=0}^r(-1)^j\sigma_{0,\dots,j-1,j+1,\dots,r}
\]
ここで$\sigma_{0,\dots,j-1,j+1,\dots,r}$は、$j$番目の座標のみ1の点$(0,\dots,0,1,0,\dots,0)\in\Delta_r$を含まない$\Delta_r$の面に、$\sigma$を制限した特異$r-1$単体です。
ただし$j=0$のときは原点を含まない面への制限とします。
境界作用素を線形に拡張して、$\partial:C_r(M)\to C_{r-1}(M)$を考えると、系列
\[
  \cdots\to C_{r+1}(M)\overset{\partial}{\to}C_r(M)\overset{\partial}{\to}C_{r-1}\to\cdots
\]
は$\partial\circ\partial=0$を満たすので、
\[
  {\rm Im}(\partial:C_{r+1}(M)\to C_r(M))\subset{\rm Ker}(\partial:C_r(M)\to C_{r-1}(M))
\]
が成り立ちます。
なので、
\[
  H_r(M):={\rm Ker}(\partial:C_r(M)\to C_{r-1}(M))/{\rm Im}(\partial:C_{r+1}(M)\to C_r(M))
\]
という群を考えることができます。
これが$M$の\textbf{ホモロジー群}というものです。
今回は係数を$\mathbb{R}$にとっておいたので、これはベクトル空間の構造をもちます。

さて、$f:M\to M$を連続写像とすると、任意の特異$r$単体$\sigma:\Delta_r\to M$に対して
\[
  f\circ\sigma:\Delta_r\to M\to M
\]
という別の特異$r$単体を考えることができます。
これを$C_r(M)$上に線形に拡大してみると、境界作用素との間に
\[
  f\circ\partial(\sigma)=\sum_{j=0}^r(-1)^jf\circ\sigma_{0,1,\dots,j-1,j+1,\dots,r}=\partial(f\circ\sigma)
\]
が成り立っていることから、ホモロジー群への群準同型
\[
  f_*:H_r(M)\to H_r(M);[\sigma]\mapsto[f\circ\sigma]
\]
がうまく定義されます。
これは定義から明らかに$\mathbb{R}$線形写像になっています。
Lefschetzの不動点定理の文脈では、ホモロジーよりも便利な\textbf{コホモロジー}$H^r(M)$を考えることが多く、その際$f:M\to M$が誘導する写像
\[
  f^*:H^r(M)\to H^r(M)
\]
を考えることもできます。

さて$M$がコンパクトな多様体であれば、$H^r(M)$は有限次元です。
そこで$H^r(M)$の適当な基底をとれば、$f^*$は行列で表現できます。
このとき、その行列のトレース${\rm Tr}(f^*)$は、線形代数学の簡単な帰結として、基底の取り方に依存しないのでした。

これでLefshetzの不動点定理の準備が整いました。
\begin{theorem}[Lefschetzの不動点定理]
  $X$を多様体、$f:M\to M$を連続写像とする。
  $\Gamma:=\{(x,f(x))\mid x\in M\}\subset M\times M$を$f$のグラフとする。
  $\Delta:=\{(x,x)\mid x\in X\}\subset X\times X$を対角集合とする。
  このとき、次が成り立つ。
  \[
    \#(\Gamma\cap\Delta)=\sum_{r}(-1)^r{\rm Tr}(f^*)
  \]
  ただし$\#(\Gamma\cap\Delta)$は、は各交点に向き($\pm1$の重み)を付けて数えた個数を表す。
\end{theorem}

話を代数多様体に戻しますと、Frobenius写像の不動点の個数が$\mathbb{F}_{q^n}$で考えた有理点の個数と一致するのでした。
ただしここでは、\textbf{有限体上の代数多様体に対して、Lefschetzの不動点定理が成り立つ程度のコホモロジーが定義できる}としています。
いま、これが、Frobenius写像が$X$上のコホモロジーに誘導された写像のトレースを調べることに帰着しました。
どういうことかというと、まず$({\rm Frob}_q^*)^n$を$H^r(X)$に対する${\rm Frob}_q^*$の$n$回作用としたとき、
\[
  N_n=\sum_{r}(-1)^r{\rm Tr}(({\rm Frob}_q^*)^n)
\]
が成り立つということです。
これを用いて合同ゼータ関数を計算してみましょう。
\begin{align*}
  \log(Z(X,t))&=\sum_{n=1}^\infty\frac{N_nt^n}{n}\\
  &=\sum_{n=1}^\infty\sum_{r}(-1)^r\frac{{\rm Tr}(({\rm Frob}_q^*)^n)t^n}{n}\\
  &=\sum_r(-1)^r\sum_{n=1}^\infty\frac{{\rm Tr}(({\rm Frob}_q^*)^n)t^n}{n}\\
\end{align*}
さて
\[
  \sum_{n=1}^\infty\frac{{\rm Tr}(({\rm Frob}_q^*)^n)t^n}{n}
\]
の部分ですが、もっと綺麗に整理できます。
まず行列とみた線形写像$({\rm Frob}_q^*)^n$のトレースというのは、その固有値の和と一致するのでした。
これらを$\lambda_1,\dots,\lambda_m$とすると、$({\rm Frob}_q^*)^n$は${\rm Frob}_q^*$を$n$回合成したものですから、その固有値は$\lambda_1^n,\dots,\lambda_m^n$となります。
従って
\[
  {\rm Tr}(({\rm Frob}_q^*)^n)=\lambda_1^n+\cdots+\lambda_m^n
\]
となります。
ゆえに
\begin{align*}
  &\sum_{n=1}^\infty\frac{{\rm Tr}(({\rm Frob}_q^*)^n)t^n}{n}\\
  =&\sum_{n=1}^\infty\frac{(\lambda_1^n+\cdots+\lambda_m^n)t^n}{n}\\
  =&-\log(1-\lambda_1t)-\cdots-\log(1-\lambda_mt)\\
  =&-\log\left(\prod_{j=1}^m(1-\lambda_jt)\right)
\end{align*}
そして$1-({\rm Frob}_q^*)^nt$という行列を対角化して行列式をとったと考えれば、
\[
  \sum_{n=1}^\infty\frac{{\rm Tr}(({\rm Frob}_q^*)^n)t^n}{n}=-\log\left(\det(1-({\rm Frob}_q^*)^nt)\right)
\]
と書けます。
ここで、$t$の多項式$\det(1-({\rm Frob}_q^*)^n)t)$を
\[
  P_r(t):=\det(1-({\rm Frob}_q^*)^n)t)
\]
とおくと、結局
\begin{align*}
  \log(Z(X,t))&=\sum_{n=1}^\infty\frac{N_nt^n}{n}\\
  &=\sum_r(-1)^{r+1}\log(P_r(t))\\
\end{align*}
となり、
\[
  Z(X,t)=\frac{P_1(t)P_3(t)\cdots P_{2d-1}(t)}{P_0(t)P_2(t)\cdots P_{2d}(t)}
\]
が\textbf{予想}されます。
この予想には、
\begin{center}
  Frobenius写像に対するLefschetzの不動点定理が成り立つ程度の\\
  代数多様体上のコホモロジー理論がないといけない
\end{center}
という大きな障害があります。



\section{Weil予想}

代数多様体上のコホモロジーを考えるということは、そのトポロジーの構造を考えるということです。
Weil当時、代数多様体に入っている位相というのは、$\mathbb{R}$や$\mathbb{C}$上の滑らかなmanifoldでない限り、\textbf{Zariski位相}と呼ばれる貧弱な位相しかありませんでした。
これはHausdorffですらなく、直感的な「図形」を表しているとすら考えづらいです。
そこで、Lefschetz不動点定理が機能するようなコホモロジー理論がいずれ構築されるだろうという期待を込めつつ、Weilは以下の予想を提案しました。

\begin{conjecture}
  $X$を体$k=\mathbb{F}_q$上の非特異射影代数多様体とする。
  このとき、以下が成り立つ。
  \begin{enumerate}
    \item \textbf{有理性} 合同ゼータ関数$Z(X,t)$は
    \[
      Z(X,t)=\frac{P_1(t)P_3(t)\cdots P_{2d-1}(t)}{P_0(t)P_2(t)\cdots P_{2d}(t)}
    \]
    という形の有理多項式である。
    ここで$P_i(t)$は整数係数多項式である。
    \item \textbf{Riemann予想のアナロジー} 上記の表式において、$\mathbb{C}$上
    \[P_i(t)=\prod_i(1-\alpha_{ij}t)\]
    とおくと、$\alpha_{ij}$は代数的整数であって、$|\alpha_{ij}|=q^{i/2}$を満たす。
    \item \textbf{Betti数について} $X$が複素射影代数多様体とするとき、以下が成り立つ。
    \[
      b_i(X):=\dim_{\mathbb{R}}H^i(X)=\deg(P_i(t))
    \]
    \item \textbf{関数等式} $\chi:=\sum_i(-1)^i\deg(P_i(t))$を$X$のEuler標数とする。
    このとき、以下の関係式が成り立つ。
    \[
      Z\left(X,\frac1{q^dt}\right)=\pm q^{d\chi/2}t^\chi Z(X,t)
    \]
  \end{enumerate}
\end{conjecture}

\subsection{総括}

ここまでWeil予想に至る道を知り、ある程度数学を知っていると、かなりとんでもなく都合のいいことを言っているように見えるのです。
僕は幾何系なので、特にBetti数のあたりがやばすぎると思いました。
数論としてはおそらく、Riemann予想のアナロジーが成り立ってしまうことにインパクトがあるのでしょう。
関数等式は、この問題に関数解析の知識が必要であることを示唆しているものと思われます。
とにかく数学のありとあらゆる分野の道具なら何でも使いそうです。

ともかく、Weil予想には有限体上の代数多様体であっても、Lefschetzの不動点定理が成り立つぐらい「豊富な」コホモロジー理論が必要です。
多分そのためにGrothendieckはスキームというものを、代数幾何学を行う「場」として定義しました。

Weil予想にチャレンジするにはLefschetzの不動点定理が必要です。
これは交点理論なので、点同士の交わり方がどうなっているかを議論できる場でなければなりません。
そのために、単純な方程式の解の集合として取り扱うのではなく、\textbf{係数体を拡大してみたときに始めてその根として現れる点}を考慮しなければなりません。
この既約多項式が、係数拡大したときに新たに表れる点の次数なのでした。
そのため、代数多様体を単なる方程式いくつかの解の集合ととらえるのではなく、\textbf{素イデアルが代数多様体を定めている}といったことに近い発想が芽生えます。

たぶん決定的なことは、将来示す次の定理です。
\begin{theorem}
  代数閉体$k$上のアフィン多様体$X$\footnote{
    つまり、有限個の方程式の解で定められた$\mathbb{A}_k^n=k^n$上の集合。
  }に対して、
  \[
    A(X)=k[x_1,\dots,x_n]/I(X),\quad I(X):=\{f\in k[x_1,\dots,x_n]\mid f(P)=0 ({}^\forall P\in X)\}
  \]
  を対応させる対応は、「$k$上のアフィン多様体の圏」から「有限生成$k$代数であるような整域」への反変な圏同型を誘導する。
\end{theorem}
この事実を$k$が代数閉体でない場合に一般化しようとすると、スキームをどう定義するべきかという問いに自然な答えを与えます。
Grothendieckのアイデアは、$X$ という「点の集合」を一度忘れ、\textbf{$A(X)$ という「環」こそが本質である}と考えることでした。
これによってアフィン多様体の対応物を、代数閉体 $k$ 上だけでなく、$\mathbb{F}_q$ や $\mathbb{Q}$、さらには一般の環 $R$ といった、より一般の「基底」の上で展開しようと試みました。
その結果として導入されたのがスキーム(Scheme)です。
最も基本的なスキームは、任意の環 $A$ に対して定義されるアフィンスキーム $\operatorname{Spec}(A)$ です。
$\operatorname{Spec}(A)$ の「点」の集合は、$A$ の素イデアル $\mathfrak{p}$ 全体の集合として定義されます。
$$ X = \operatorname{Spec}(A) := \{ \mathfrak{p} \mid \mathfrak{p} \text{ は } A \text{ の素イデアル} \}$$
私たちがこれまで「閉点」と呼んできたものは、この定義における極大イデアルに対応します。
スキームでは、極大イデアルではない素イデアル(例えば $\mathbb{F}_q[T]$ における $(0)$ イデアルなど)も、多様体全体の情報を担う「生成点」として、幾何学的な「点」の一つとみなします。
そして、一般のスキームとは、ちょうど多様体がEuclid空間の一部を貼り合わせて定義されるように、これらのアフィンスキーム $\operatorname{Spec}(A_i)$ を「貼り合わせる」ことで得られる、極めて広範な幾何学的対象として定義されます。
この「スキーム」という新しい「場」を導入したことで、
\begin{itemize}
\item $\mathbb{F}_q$ 上の代数多様体(Weil予想の舞台)
\item $\mathbb{Q}$ 上の代数多様体(Fermat予想の舞台)
\item $\mathbb{Z}$ 上の代数多様体(Diophantus方程式そのもの)
\end{itemize}
といった、従来は異なる分野(数論、幾何学)で扱われていた対象を、すべて「スキーム」という単一の言語で統一的に記述することが可能になりました。
そして、この広大な基盤の上でこそ、Grothendieckは「エタールコホモロジー」と呼ばれる、Weil予想を解決するに足る強力なコホモロジー理論を構築することに成功したのです。

  \part{本文の解説}
  \chapter{多様体}

\footnote{
  いきなりですがVarietyとManifoldを同じ「多様体」と訳したのは誰なんですかね?
  Hartshorneに出てくる多様体はVarietyです。
}
この章の目標は、少ない知識から代数幾何の重要な概念をいくつか紹介することだそうです。
ここでは代数閉体$k$上のアフィン空間や射影空間に含まれる、本当に方程式のゼロ点集合を扱います。
そこから始めて、多様体の、
\begin{itemize}
  \item 次元
  \item 正則関数
  \item 有理写像
  \item 非特異多様体
  \item 射影多様体の次数
\end{itemize}
といったものを紹介していきます。

親切(心折)にも、各節の終わりにたくさんの演習問題を用意してくれています。
これも頑張って解いていきましょう。
なお\textbf{ちまちま未解決問題が含まれている}ので、それはさすがに手を付けません。
あと、この本は代数幾何学の本なので、可換環論の結果はあんまり証明を書いてくれてません。
そこも頑張って埋めていきたいところですね。

最後の節では、代数幾何学が結局どこに向かっているのかについて解説しています。
これは僕のお気持ちである第0章もあるので、ざっくり省きます。

\section{アフィン多様体}

\begin{definition}
  $k$を代数閉体とする。このとき
  \[
    \mathbb{A}_k^n:=\{(a_1,\dots,a_n)\mid a_i\in k\}
  \]
  を\textbf{アフィン空間}という。
  体が明らかなときは、単に$\mathbb{A}^n:=\mathbb{A}_k^n$と書くこともある。
\end{definition}

$k$を代数閉体、$A:=k[x_1,\dots,x_n]$を$k$上の多項式環とします。
このとき、任意の$f\in A$に対して、代入写像
\[
  f:\mathbb{A}^n\to k; P\mapsto f(P)
\]
を考えることができます。
より具体的には、$P=(a_1,\dots,a_n)$のとき、$f(P)$は、$f$に現れている変数$x_i$をすべて$a_i$に置き換えて得られる$k$の元です。

よって$f(P)=0$となる$P$の集合、つまり\textbf{ゼロ点集合}を定義することができます。
より一般に、$A$の部分集合$T$に対して次のようにゼロ点集合を定義します。
\begin{definition}
  $k$を代数閉体、$\mathbb{A}^n$をアフィン空間、$A=k[x_1,\dots,x_n]$を多項式環、$T\subset A$を部分集合とする。
  このとき、$T$のゼロ点集合を
  \[
    Z(T):=\{P\in\mathbb{A}^n\mid f(P)=0 ({}^\forall f\in T)\}
  \]
  で定義する。
\end{definition}

以下の主張は、著者は明らかと言っています。
確かに明らかといえば明らかですが、セミナーで聞かれたらちゃんと答えられない学生もいるでしょうから、証明しましょう\footnote{「明らか」はセミナーでは地雷ワードです。}。
\begin{remark}
  $k$を代数閉体、$\mathbb{A}^n$をアフィン空間、$A=k[x_1,\dots,x_n]$を多項式環、$T\subset A$を部分集合とする。
  $\mathfrak{a}$が$T$で生成される$A$のイデアルならば
  \[
    Z(T)=Z(\mathfrak{a})
  \]
\end{remark}
\begin{proof}
  $P\in Z(T)$とする。
  すなわち、任意の$f\in T$に対して、$f(P)=0$。
  $\mathfrak{a}$は$T$で生成されたイデアルであるから、その任意の元$F$は
  \[
    F=\sum_{i=1}^rg_if_i,\quad \text{ただし }f_i\in T, g_i\in A
  \]
  と書かれる。
  任意の$f\in T$に対して$f(P)=0$であったから、
  \[
    F(P)=\sum_{i=1}^rg_i(P)f_i(P)=\sum_{i=1}^rg_i(P)\cdot0=0
  \]
  ゆえに$P\in Z(\mathfrak{a})$。
  従って$Z(T)\subset Z(\mathfrak{a})$。

  逆に$P\in Z(\mathfrak{a})$とする。
  $\mathfrak{a}$は$T$で生成されたイデアルであるから、$T\subset\mathfrak{a}$。
  任意の$f\in T$に対して$f\in\mathfrak{a}$であるから、$f(P)=0$となる。
  ゆえに$P\in Z(T)$。
  従って$Z(\mathfrak{a})\subset Z(T)$。
  以上より$Z(T)=Z(\mathfrak{a})$。
\end{proof}


  \part{可換環論}
  \chapter{可換環論 基礎編}

Hartshorneに出てくる可換環論の中でも、僕の独断と偏見で基礎的と思われるものを頑張って解説する章です。

\section{可換環の定義と準同型}

\subsection{可換環の定義}

ざっくり説明すると、「足し算」「引き算」「掛け算」までできることを保証して、「割り算」ができるとは限らないような代数系です。
厳密には次の通りです。
\begin{definition}[可換環]
  $A$を空でない集合、$+:A\times A\to A$、$\cdot:A\times A\to A$をそれぞれ写像、$0,1\in A$とする。
  このとき、$(A,+,\cdot,0,1)$が\textbf{可換環}(単に$A$と表記する)であるとは、任意の$a,b,c\in A$に対して次を満たす時を言う。
  \begin{itemize}
    \item \textbf{演算$+$に関して可換群} 
    \begin{itemize}
      \item \textbf{結合法則} $a+(b+c)=(a+b)+c$
      \item \textbf{単位元} $a+0=0+a=a$
      \item \textbf{逆元} $a+(-a)=(-a)+a=0$ (${}^\exists -a\in A$)
      \item \textbf{可換性} $a+b=b+a$
    \end{itemize}
    \item \textbf{演算$\cdot$に関して可換モノイド}
    \begin{itemize}
      \item \textbf{結合法則} $a\cdot (b\cdot c)=(a\cdot b)\cdot c$
      \item \textbf{単位元} $a\cdot 1=1\cdot a=a$
      \item \textbf{可換性} $a\cdot b=b\cdot a$
    \end{itemize}
    \item \textbf{分配法則}\footnote{
      本来$(a+b)\cdot c=a\cdot c+b\cdot c$も含めるべきかもしれませんが、積の可換性があるのでこれで十分です。
    } $a\cdot(b+c)=a\cdot b+a\cdot c$
  \end{itemize}
\end{definition}
省略記法として、
\begin{align*}
  ab&=a\cdot b\\
  a-b&=a+(-b)
\end{align*}
を用いることがほとんどです。

また、基本的に$1=0$の可能性を排除しません。
もし$1=0$ならば、次のことが成り立ちます。
\begin{theorem}
  $A$を可換環とする。もし$1=0$ならば、$A=\{0\}$。
\end{theorem}
\begin{proof}
  任意の$a\in A$に対して、
  \[
    a=1\cdot a=0\cdot a=0
  \]
\end{proof}

\begin{example}
  整数全体$\mathbb{Z}$、有理数全体$\mathbb{Q}$、実数全体$\mathbb{R}$、複素数全体$\mathbb{C}$は可換環。
  自然数全体$\mathbb{N}$は可換環ではない。
\end{example}

\begin{example}
  可換環$A$に対して、$n$個の未知変数$x_1,\dots,x_n$の有限項からなる多項式がなす全体の集合
  \[
    A[x_1,\dots,x_n]=\left\{\sum_{\text{有限和}} a_{i_1\cdots i_n}x_1^{j_1}\cdots x_n^{j_n}\mid a_{i_1\cdots i_n}\in A\right\}
  \]
  は、通常の和と積に関して可換環となる。
\end{example}

\subsection{準同型}

可換環の構造を保つ写像を環準同型写像といいます。
\begin{definition}[環準同型写像]
  $A,B$を環とする。写像$f:A\to B$が\textbf{環準同型写像}であるとは、次を満たす時を言う。
  \begin{itemize}
    \item $f(a+b)=f(a)+f(b)$
    \item $f(ab)=f(a)f(b)$
    \item $f(1)=1$
  \end{itemize}
\end{definition}

次の同型写像が存在するとき、代数的にはそれらの環の構造を区別できないほど同じです。
\begin{definition}
  $A,B$を環とする。写像$f:A\to B$が\textbf{同型写像}であるとは、$f$が全単射な環準同型であって、逆写像も環準同型になっているときをいう。
  可換環$A,B$の間に同型写像が存在するとき、$A$と$B$は互いに\textbf{同型}であるという。
\end{definition}



\section{イデアル}

\subsection{イデアルの定義}

イデアルは、整数環$\mathbb{Z}$における「$n$倍数」を抽象化した概念と言えます。
\begin{definition}[イデアル]
  $A$を可換環とする。部分集合$\mathfrak{a}\subset A$が$A$の\textbf{イデアル}であるとは、次を満たす時を言う。
  \begin{itemize}
    \item $0\in\mathfrak{a}$
    \item $x,y\in \mathfrak{a}\implies x+y\in \mathfrak{a}$
    \item $a\in A, x\in \mathfrak{a}\implies ax\in\mathfrak{a}$
  \end{itemize}
\end{definition}

\subsubsection{イデアルの例}

\begin{example}
  $n$を整数とするとき、$n$の倍数全体の集合
  \[
    (n):=\{an\mid a\in\mathbb{Z}\}
  \]
  は、可換環$\mathbb{Z}$のイデアルである。
\end{example}

上記の例で既に用いてしまいましたが、可換環のいくつかの元が生成するイデアルというものを考えることができます。
\begin{theorem}
  $A$を可換環、$T\subset A$を$A$の部分集合とする。
  このとき、
  \[
    (T):=\left\{\sum_{\text{有限和}}a_ix_i\mid a_i\in A, x_i\in T\right\}
  \]
  は$A$のイデアルである。
\end{theorem}
\begin{proof}
  まず、有限和の係数をすべて0にすれば$0\in(T)$がわかる。
  次に$\sum a_ix_i, \sum b_jy_y\in (T)$ならば$\sum a_ix_i+\sum b_jy_y\in (T)$であることは、有限和ふたつの和が有限和になることからわかる。
  最後に$a\in A$、$\sum b_jy_y\in (T)$ならば、
  \[
    a\sum b_jy_y=\sum (ab_j)y_j\in(T)
  \]
\end{proof}
\begin{definition}
  $A$を可換環、$T\subset A$を$A$の部分集合とする。
  このとき、イデアル
  \[
    (T):=\left\{\sum_{\text{有限和}}a_ix_i\mid a_i\in A, x_i\in T\right\}
  \]
  を、\textbf{$T$が生成するイデアル}と呼ぶ。
  $T$が有限個の元$a_1,\dots,a_n$からなるとき、単に
  \[
    (a_1,\dots,a_n):=(\{a_1,\dots,a_n\})
  \]
  と書く。

  特に一つの元のみから生成されるイデアルを\textbf{単項イデアル}と呼ぶ。
  $a$が生成する単項イデアルは
  \[
    (a),\text{  あるいは  }aA
  \]
  と書かれる。
\end{definition}

\begin{example}
  可換環$\mathbb{Q}$には、イデアルが
  \[(0)=\{0\},\quad (1)=\mathbb{Q}\]
  の二つしか存在しない。
  実際、$\mathfrak{a}$を$(0)$でないイデアルとすると、ある$x\in\mathfrak{a}$が存在して、$x\neq0$。
  従って、$x^{-1}\in\mathbb{Q}$であるから、
  \[
    1=x^{-1}x\in\mathfrak{a}
  \]
  ゆえに、任意の有理数$q$に対して、
  \[
    q=1\cdot q\in\mathfrak{a}
  \]
  ゆえに$\mathfrak{a}=\mathbb{Q}$
\end{example}

\subsection{素イデアルと極大イデアル}

素数は整数の中でも特別な存在でした。
整数環における素数に対応するイデアルが次の概念です。
\begin{definition}
  $A$を可換環、$A$のイデアル$\mathfrak{p}\neq A$が\textbf{素イデアル}であるとは、任意の$x,y\in A$に対して、次を満たす時を言う。
  \[
    xy\in\mathfrak{p} \implies x\in\mathfrak{p} \text{ or } y\in\mathfrak{p}
  \]
\end{definition}

\subsubsection{素イデアルの例}

\begin{example}
  素数$p$に対して、$p\mathbb{Z}\subset\mathbb{Z}$は素イデアルである\footnote{
    $\mathbb{Z}$の単項イデアルを$(p)$などと書くとややこしいことが多いので、個人的には$p\mathbb{Z}$と書きたいです。
  }。
\end{example}
\begin{proof}
  $xy\in p\mathbb{Z}$、すなわち、ある$a\in \mathbb{Z}$が存在して、$xy=ap$が成り立つとする。
  このとき、素因数分解の一意性から、$x$または$y$は$p$で割り切れなければならない。
  すなわち、もし$x$が$p$で割り切れるならば、$x=x'p$ (${}^\exists x'\in\mathbb{Z}$)。
  これはすなわち$x\in p\mathbb{Z}$を意味する。
  $y$が$p$で割り切れるとしても同様である。
\end{proof}

\begin{example}
  $6\mathbb{Z}\subset\mathbb{Z}$は素イデアルではない。
\end{example}
\begin{proof}
  実際、$2\times3=6\in 6\mathbb{Z}$であるが、$2$と$3$は$6$で割り切れないから$2\notin 6\mathbb{Z}$かつ$3\notin 6\mathbb{Z}$。
\end{proof}

\begin{example}
  $\mathbb{C}[X,Y]$を$\mathbb{C}$上の2変数多項式環とする。
  このとき、$(X)\subset\mathbb{C}[X,Y]$は素イデアルである。
\end{example}
\begin{proof}
  $fg\in(X)$、すなわち、$fg=hX$ (${}^\exists h\in\mathbb{C}[X,Y]$)とする。
  このとき、$X$に関するそれぞれの定数項$f(0,Y)$、$g(0,Y)$を考えると、
  \[
    f(0,Y)g(0,Y)=h(0,Y)\cdot0=0
  \]
  であるから、$f,g$の$X$に関する定数項の積は0となる。
  ゆえにどちらかの$X$に関する定数項は0であるから、$f$か$g$どちらかは$X$で括れる形になっている。
  すなわち$f\in(X)$または$g\in(X)$
\end{proof}

\subsection{極大イデアル}

包含関係に関して極大なイデアルを極大イデアルといいます。
\begin{definition}
  $A$を可換環とする。イデアル$\mathfrak{m}\subset A$が\textbf{極大イデアル}であるとは、次を満たす時を言う。
  もし$\mathfrak{m}\subset\mathfrak{a}\subset A$となるイデアル$\mathfrak{a}$が存在するならば、$\mathfrak{a}=\mathfrak{m}$または$\mathfrak{a}=A$。
\end{definition}

\subsubsection{極大イデアルの例}

\begin{example}
  素数$p$に対して、$p\mathbb{Z}\subset\mathbb{Z}$は極大イデアルである。
\end{example}
\begin{proof}
  もし$p\mathbb{Z}\subsetneq\mathfrak{a}\subset\mathbb{Z}$となるイデアル$\mathfrak{a}$が存在したとすると、$\mathfrak{a}$には$p$の倍数でない整数$x$が存在する。
  $p$は素数ゆえ、$x$と$p$は互いに素となる。
  ここでユークリッドの互除法より$pn+xy=1$をみたす$n,y\in\mathbb{Z}$が存在し、$p,x\in\mathfrak{a}$であるから、$1=pn+xy\in\mathfrak{a}$となる。
  従って、任意の$m\in\mathbb{Z}$に対して、$m=1\cdot m\in\mathfrak{a}$となり、$\mathfrak{a}=\mathbb{Z}$
\end{proof}

この証明の中でイデアルに1が含まれると必ず全体になるというふうに読める部分がありますが、実際これは一般に成り立ちます。
\begin{theorem}
  $A$を可換環、$\mathfrak{m}\subset A$をイデアルとする。もし$1\in\mathfrak{a}$ならば、$\mathfrak{a}=A$
\end{theorem}
\begin{proof}
  任意の$a\in A$に対して$a=1\cdot a\in\mathfrak{a}$ゆえ$A\subset\mathfrak{a}$。
  当然$\mathfrak{a}\subset A$でもあるから、$\mathfrak{a}=A$が従う。
\end{proof}

\begin{example}
  $\mathbb{C}[X,Y]$を$\mathbb{C}$上の2変数多項式環とする。
  このとき、$(X,Y)\subset\mathbb{C}[X,Y]$は極大イデアルである。
\end{example}
\begin{proof}
  定義から
  \[
    (X,Y)=\left\{\sum_{i_1=1}^{r_1}\sum_{i_2=1}^{r_2}a_{i_1i_2}X^{i_1}Y^{i_2}\mid a_{i_1i_2}\in\mathbb{C}\right\}
  \]
  であったことから、イデアル$(X,Y)$は定数項が0の多項式全体の集合であることに注意する。
  $(X,Y)\subsetneq\mathfrak{a}\subset\mathbb{C}[X,Y]$とする。
  このとき、ある$f\in\mathfrak{a}$が存在して、$f\notin(X,Y)$。
  つまり$f$は定数項$f(0,0)$が0ではない多項式である。
  \[
    f(X,Y)-f(0,0)\in(X,Y)\subset\mathfrak{a}
  \]
  であるから、
  \[
    1=\frac{f(X,Y)-(f(X,Y)-f(0,0))}{f(0,0)}=\frac{1}{f(0,0)}f(X,Y)-\frac{1}{f(0,0)}(f(X,Y)-f(0,0))\in\mathfrak{a}
  \]
  となり、$\mathfrak{a}=\mathbb{C}[X,Y]$が従う。
\end{proof}

\subsubsection{極大イデアルは素イデアル}
一般に、極大イデアルは素イデアルです。
\begin{theorem}
  $A$を可換環とする。イデアル$\mathfrak{m}\subset A$が極大イデアルならば、$\mathfrak{m}$は素イデアルである。
\end{theorem}
\begin{proof}
  背理法によって証明する。
  $ab\in\mathfrak{m}$かつ、$a\notin\mathfrak{m}$かつ$b\notin\mathfrak{m}$を満たす$a,b\in A$が存在するとする。
  このとき
  \[
    \mathfrak{a}:=\{ax+y\mid x\in A, y\in\mathfrak{m}\}
  \]
  とおくと、これはイデアルである。
  実際、あきらかに$0\in\mathfrak{a}$。
  また、$ax+y,ax'+y'\in\mathfrak{a}$とすると、$x+x'\in A$、$y+y'\in\mathfrak{m}$であるから、
  \[
    (ax+y)+(ax'+y)=a(x+x')+(y+y')\in\mathfrak{a}
  \]
  最後に、$c\in A$、$ax+y\in\mathfrak{a}$とすると、$cx\in A$、$cy\in\mathfrak{m}$であるから、
  \[
    c(ax+y)=a(cx)+cy\in\mathfrak{a}
  \]
  ゆえに$\mathfrak{a}$は$A$のイデアルである。

  また$\mathfrak{m}\subsetneq\mathfrak{a}$である。
  実際、$ax+y$の表式において$x=0$とすれば、任意の$\mathfrak{m}$の元が$\mathfrak{a}$に含まれることがわかる。
  一方で$a\notin\mathfrak{m}$であるが、あきらかに$a\in\mathfrak{a}$であるから、$\mathfrak{m}\subsetneq\mathfrak{a}$である。

  $\mathfrak{m}$は極大イデアルであるから、従って$\mathfrak{a}=A$となる。
  従って、ある$x\in A$と$y\in\mathfrak{m}$が存在して、
  \[
    ax+y=1
  \]
  をみたす。この両辺に$b$を掛けると、
  \[
    abx+by=b
  \]
  であるが、$ab\in\mathfrak{m}$かつ$y\in\mathfrak{m}$だったため、$b\in\mathfrak{m}$となるが、これは矛盾である。
\end{proof}

この逆、つまり「素イデアルは極大イデアル」は一般に成り立ちません。
\begin{example}
  $\mathbb{C}[X,Y]$を$\mathbb{C}$上の2変数多項式環とする。
  このとき、$(X)\subset\mathbb{C}[X,Y]$は素イデアルであるが、極大イデアルではない。
\end{example}
\begin{proof}
  たとえば$(X,Y)$は$(X)$を含む極大イデアルである。
\end{proof}



\begin{comment}
  必要になったら必要になったところに移動する。
\section{イデアル演算}

極大イデアル$\mathfrak{m}$が素イデアルであることを証明するために、$a\in A$に対して
\[
  \{ax+y\mid x\in A, y\in\mathfrak{m}\}
\]
というイデアルを作りました。
この考えを一般化することができます。
\begin{theorem}
  $A$を可換環、$\mathfrak{a},\mathfrak{b}$を$A$のイデアルとする。
  このとき、
  \[
    \mathfrak{c}:=\{x+y\mid x\in\mathfrak{a},y\in\mathfrak{b}\}
  \]
  はイデアルである。
\end{theorem}
\begin{proof}
  $0\in\mathfrak{c}$は自明。
  $x+y,x'+y'\in\mathfrak{c}$とすると、$x+x'\in\mathfrak{a}$、$y+y'\in\mathfrak{b}$より
  \[
    (x+y)+(x'+y')=(x+x')+(y+y')\in\mathfrak{c}
  \]
  最後に、$a\in A$、$x+y\in\mathfrak{c}$とすると、$ax\in\mathfrak{a}$、$ay\in\mathfrak{b}$より
  \[
    a(x+y)=ax+ay\in\mathfrak{c}
  \]
\end{proof}
\begin{definition}
  $A$を可換環、$\mathfrak{a},\mathfrak{b}$を$A$のイデアルとする。
  このとき、イデアル
  \[
    \mathfrak{a}+\mathfrak{b}:=\{x+y\mid x\in\mathfrak{a},y\in\mathfrak{b}\}
  \]
  を$\mathfrak{a}$と$\mathfrak{b}$の和という。
\end{definition}
有限個のイデアルの和も帰納的に定義できます。
無限個のイデアルのときは次のように定義します。
\begin{theorem}
  $A$を可換環、$\{\mathfrak{a}_i\}_{i\in I}$を$A$のイデアルの族とする。
  このとき、
  \[
    \mathfrak{b}:=\{\sum_{i\in I}x_i\mid x_i\in\mathfrak{a}_i, \text{有限個の$i\in I$を除いて$x_i=0$}\}
  \]
  はイデアルである。
\end{theorem}
\end{comment}



\section{剰余環}

高校時代に理系だった人は、大学受験の裏技として$\mod$を習ったかもしれません。
例えば
\[
  x\equiv y\mod n
\]
と書けば、これは「$x$を$n$で割った余りと、$y$を$n$で割った余りは等しい」と読むのでした。
そうなると、整数は
\begin{itemize}
  \item $n$の倍数
  \item $n$で割って$1$余る
  \item $n$で割って$2$余る
  \item \dots
  \item $n$で割って$n-1$余る
\end{itemize}
によって完全に分類できています。
また、
\begin{align*}
  x\equiv y\mod n &\implies x+c\equiv y+c\mod n\\
  x\equiv y\mod n &\implies x-c\equiv y-c\mod n\\
  x\equiv y\mod n &\implies ax\equiv ay\mod n
\end{align*}
が成り立つことから、環の構造が見えてきます。
この考えを一般化したものが剰余環です。
\begin{theorem}
  $A$を可換環、$\mathfrak{a}$をイデアルとする。
  このとき、$x,y\in A$に対して、$x\equiv y\mod\mathfrak{a}$であることを
  \[
    x-y\in\mathfrak{a}
  \]
  であることと定義すると、この関係は同値関係である。
\end{theorem}
\begin{proof}
  $x\in A$に対して、$x-x=0\in\mathfrak{a}$であるから、反射律を満たす。
  $x\equiv y\mod \mathfrak{a}$であるとすると、すなわち$x-y\in\mathfrak{a}$である。
  $\mathfrak{a}$がイデアルであることから、$y-x=-(x-y)\in\mathfrak{a}$。すなわち対称律を満たす。
  $x\equiv y\mod \mathfrak{a}$かつ$y\equiv z\mod \mathfrak{a}$とすると、$x-y,y-z\in\mathfrak{a}$であるから、$x-z=(x-y)+(y-z)\in\mathfrak{a}$。
  ゆえに推移律も満たす。
\end{proof}
\begin{theorem}
  $A$を可換環、$\mathfrak{a}$をイデアルとする。
  同値関係$x\equiv y\mod\mathfrak{a}$による剰余集合を$A/\mathfrak{a}$と書く。
  また、$x\in A$が含まれる同値類を$x+\mathfrak{a}\in A/\mathfrak{a}$と書く。
  このとき、写像
  \[
    A/\mathfrak{a}\times A/\mathfrak{a}\to A/\mathfrak{a};(x+\mathfrak{a},y+\mathfrak{a})\mapsto x+y+\mathfrak{a}
  \]
  はwell-definedである。
\end{theorem}
\begin{proof}
  $x\equiv x'\mod\mathfrak{a}$、$y\equiv y'\mod\mathfrak{a}$とすると、$x-x',y-y'\in\mathfrak{a}$より、
  \[
    (x+y)-(x'+y')=(x-x')+(y-y')\in\mathfrak{a}
  \]
  ゆえに$x+y\equiv x'+y'\mod\mathfrak{a}$
\end{proof}
\begin{theorem}
  $A$を可換環、$\mathfrak{a}$をイデアルとする。
  同値関係$x\equiv y\mod\mathfrak{a}$による剰余集合を$A/\mathfrak{a}$と書く。
  また、$x\in A$が含まれる同値類を$x+\mathfrak{a}\in A/\mathfrak{a}$と書く。
  このとき、写像
  \[
    A/\mathfrak{a}\times A/\mathfrak{a}\to A/\mathfrak{a};(x+\mathfrak{a},y+\mathfrak{a})\mapsto xy+\mathfrak{a}
  \]
  はwell-definedである。
\end{theorem}
\begin{proof}
  $x\equiv x'\mod\mathfrak{a}$、$y\equiv y'\mod\mathfrak{a}$とすると、$x-x',y-y'\in\mathfrak{a}$より、
  \[
    xy-x'y'=xy-x'y+x'y-x'y'=y(x-x')+x'(y-y')\in\mathfrak{a}
  \]
  ゆえに$xy\equiv x'y'\mod\mathfrak{a}$
\end{proof}
\begin{theorem}
  $A$を可換環、$\mathfrak{a}$をイデアルとする。
  また、$x\in A$が含まれる同値類を$x+\mathfrak{a}\in A/\mathfrak{a}$と書く\footnote{
    すなわち、$y\in x+\mathfrak{a}\iff x\equiv y \mod \mathfrak{a}$
  }。
  このとき$A/\mathfrak{a}$は、
  \[
    +:A/\mathfrak{a}\times A/\mathfrak{a}\to A/\mathfrak{a};(x+\mathfrak{a},y+\mathfrak{a})\mapsto x+y+\mathfrak{a}
  \]
  を加法、
  \[
    \cdot:A/\mathfrak{a}\times A/\mathfrak{a}\to A/\mathfrak{a};(x+\mathfrak{a},y+\mathfrak{a})\mapsto xy+\mathfrak{a}
  \]
  を乗法とし、
  \[0+\mathfrak{a}, 1+\mathfrak{a}\]
  をそれぞれ加法単位元、乗法単位元として、可換環をなす。
\end{theorem}
\begin{proof}
  任意の$x+\mathfrak{a}, y+\mathfrak{a}, z+\mathfrak{a}\in A/\mathfrak{a}$に対して、
  \begin{itemize}
    \item 和に関する結合法則:$x+\mathfrak{a}+(y+\mathfrak{a}+z+\mathfrak{a})=x+\mathfrak{a}+(y+z)+\mathfrak{a}=(x+(y+z))+\mathfrak{a}=((x+y)+z)+\mathfrak{a}=(x+y)+\mathfrak{a}+z+\mathfrak{a}=(x+\mathfrak{a}+y+\mathfrak{a})+z+\mathfrak{a}$
    \item 和に関する単位元:$0+\mathfrak{a}+x+\mathfrak{a}=(0+x)+\mathfrak{a}=x+\mathfrak{a}$、$x+\mathfrak{a}+0+\mathfrak{a}=(x+0)+\mathfrak{a}=x+\mathfrak{a}$
    \item 和に関する逆元:$x+\mathfrak{a}+(-x)+\mathfrak{a}=(x-x)+\mathfrak{a}=0+\mathfrak{a}$、$-x+\mathfrak{a}+x+\mathfrak{a}=(-x+x)+\mathfrak{a}=0+\mathfrak{a}$
    \item 積に関する結合法則:$(x+\mathfrak{a})\{(y+\mathfrak{a})(z+\mathfrak{a})\}=(x+\mathfrak{a})(yz+\mathfrak{a})=x(yz)+\mathfrak{a}=(xy)z+\mathfrak{a}=(xy+\mathfrak{a})(z+\mathfrak{a})=\{(x+\mathfrak{a})(y+\mathfrak{a})\}(z+\mathfrak{a})$
    \item 積に関する単位元:$(1+\mathfrak{a})(x+\mathfrak{a})=1\cdot x+\mathfrak{a}=x+\mathfrak{a}$、$(x+\mathfrak{a})(1+\mathfrak{a})=x\cdot1+\mathfrak{a}=x+\mathfrak{a}$
    \item 積の可換性:$(x+\mathfrak{a})(y+\mathfrak{a})=xy+\mathfrak{a}=yx+\mathfrak{a}=(y+\mathfrak{a})(x+\mathfrak{a})$
    \item 分配法則:$(x+\mathfrak{a})(y+\mathfrak{a}+z+\mathfrak{a})=(x+\mathfrak{a})(y+z+\mathfrak{a})=x(y+z)+\mathfrak{a}=xy+xz+\mathfrak{a}=xy+\mathfrak{a}+xz+\mathfrak{a}$
  \end{itemize}
\end{proof}

剰余環には以下のような自然な準同型が存在して、重要なものだったりします。
\begin{theorem}
  $A$を可換環、$\mathfrak{a}$をそのイデアルとする。
  このとき、写像
  \[
    \pi:A\to A/\mathfrak{a};x\mapsto x+\mathfrak{a}
  \]
  は環準同型である。
\end{theorem}
\begin{proof}
  任意の$x,y\in A$に対して、
  \begin{itemize}
    \item $\pi(x+y)=x+y+\mathfrak{a}=x+\mathfrak{a}+y+\mathfrak{a}=\pi(x)+\pi(y)$
    \item $\pi(xy)=xy+\mathfrak{a}=(x+\mathfrak{a})(y+\mathfrak{a})=\pi(x)\pi(y)$
    \item $\pi(1)=1+\mathfrak{a}$
  \end{itemize}
\end{proof}

\subsection{剰余環の例}

\begin{example}
  $\mathbb{Z}/n\mathbb{Z}$は$n$個の元からなる可換環である。
  \[
    \mathbb{Z}/n\mathbb{Z}=\{0+n\mathbb{Z},1+n\mathbb{Z},2+n\mathbb{Z},\dots,n-1+n\mathbb{Z}\}
  \]
  たとえば$n=3$のときの和と積の表は次のようになる。
  \[
    \begin{array}{c|ccc}
      + & 0 & 1 & 2\\\hline
      0 & 0 & 1 & 2\\
      1 & 1 & 2 & 0\\
      2 & 2 & 0 & 1
    \end{array}
    \qquad
    \begin{array}{c|ccc}
      \cdot & 0 & 1 & 2\\\hline
      0 & 0 & 0 & 0\\
      1 & 0 & 1 & 2\\
      2 & 0 & 2 & 1
    \end{array}
  \]
\end{example}

\begin{example}
  $\mathbb{R}[X]/(X^2+1)$において、$X$を含む同値類$X+(X^2+1)$は虚数と同じ役割を果たす。
  実際、
  \[
    (X+(X^2+1))(X+(X^2+1))=X^2+(X^2+1)
  \]
  であるが、もちろん$X^2+1\equiv0\mod(X^2+1)$のため、$X^2\equiv-1\mod(X^2+1)$。
  ゆえに
  \[
    (X+(X^2+1))(X+(X^2+1))=-1+(X^2+1)
  \]
\end{example}
次に紹介する定理を使って、これが実際に$\mathbb{C}$と同型であることを証明します。



\section{環準同型定理}

剰余環には、「イデアルで環をつぶす」という感覚があります。
それを如実に表わすのが環準同型定理でしょう。
準備のために、いくつか準備が必要です。
\begin{theorem}
  $A,B$を可換環、$f:A\to B$を環準同型とする。
  このとき、
  \[
    \operatorname{Ker}(f):=\{a\in A\mid f(a)=0\}
  \]
  は$A$のイデアルである。
\end{theorem}
\begin{proof}
  $f(0)=0$であるから、$0\in\operatorname{Ker}(f)$。
  また$x,y\in\operatorname{Ker}(f)$、すなわち$f(x)=f(y)=0$ならば、$f(x+y)=f(x)+f(y)=0+0=0$。ゆえに$x+y\in\operatorname{Ker}(f)$。
  最後に、$a\in A$、$x\in\operatorname{Ker}(f)$とおくと、$f(ax)=f(a)f(x)=f(a)\cdot0=0$。ゆえに$ax\in\operatorname{Ker}(f)$。
\end{proof}
\begin{definition}
  $A,B$を可換環、$f:A\to B$を環準同型とする。
  このとき、イデアル
  \[
    \operatorname{Ker}(f):=\{a\in A\mid f(a)=0\}
  \]
  を、$f$の\textbf{カーネル}(\textit{kernel})と呼ぶ。
\end{definition}
\begin{definition}
  $A,B$を可換環とし、$A\subset B$を満たすとする。
  このとき、包含写像$i:A\hookrightarrow B$が環準同型となるとき、$A$は$B$の\textbf{部分環}であるという。
\end{definition}
\begin{theorem}
  $A,B$を可換環、$f:A\to B$を環準同型とする。
  このとき、
  \[
    \operatorname{Im}(f):=\{f(x)\in B\mid x\in A\}
  \]
  は$B$の部分環である。
\end{theorem}
\begin{proof}
  $f$が環準同型であることから自明。
\end{proof}
\begin{theorem}[環準同型定理(well-definedness)]
  $A,B$を可換環、$f:A\to B$を環準同型とする。
  このとき、
  \[
    \overline{f}:A/\operatorname{Ker}(f)\to\operatorname{Im}(f);x+\operatorname{Ker}(f)\mapsto f(x)
  \]
  はうまく定義された写像である。
\end{theorem}
\begin{proof}
  $x-y\in\operatorname{Ker}(f)$、すなわち$f(x-y)=0$を満たすとする。
  このとき、$0=f(x-y)=f(x)-f(y)$ゆえに$f(x)=f(y)$
\end{proof}
\begin{theorem}[環準同型定理]
  $A,B$を可換環、$f:A\to B$を環準同型とする。
  このとき、写像
  \[
    \overline{f}:A/\operatorname{Ker}(f)\to\operatorname{Im}(f);x+\operatorname{Ker}(f)\mapsto f(x)
  \]
  は環同型写像である。
\end{theorem}
\begin{proof}
  \textbf{準同型であること} $x+\operatorname{Ker}(f),y+\operatorname{Ker}(f)\in A/\operatorname{Ker}(f)$に対して、
  \begin{itemize}
    \item $\overline{f}(x+\operatorname{Ker}(f)+y+\operatorname{Ker}(f))=\overline{f}(x+y+\operatorname{Ker}(f))=f(x+y)$。
    一方で$\overline{f}(x+\operatorname{Ker}(f))+\overline{f}(y+\operatorname{Ker}(f))=f(x)+f(y)=f(x+y)=\overline{f}(x+\operatorname{Ker}(f)+y+\operatorname{Ker}(f))$
    \item $\overline{f}((x+\operatorname{Ker}(f))(y+\operatorname{Ker}(f)))=\overline{f}(xy+\operatorname{Ker}(f))=f(xy)$。
    一方で$\overline{f}(x+\operatorname{Ker}(f))\overline{f}(y+\operatorname{Ker}(f))=f(x)f(y)=f(xy)=\overline{f}((x+\operatorname{Ker}(f))(y+\operatorname{Ker}(f)))$
    \item $\overline{f}(1+\operatorname{Ker}(f))=f(1)=1$
  \end{itemize}

  \textbf{単射性} $x+\operatorname{Ker}(f),y+\operatorname{Ker}(f)\in A/\operatorname{Ker}(f)$に対して、$f(x)=f(y)$を満たすとする。
  このとき、$0=f(x)-f(y)=f(x-y)$ゆえ$x-y\in\operatorname{Ker}(f)$となる。
  従って$x+\operatorname{Ker}(f)=y+\operatorname{Ker}(f)$。

  \textbf{全射性} 任意の$f(x)\in\operatorname{Im}(f)$に対して、$\overline{f}(x+\operatorname{Ker}(f))=f(x)$。
\end{proof}

\begin{example}
  $\mathbb{R}[X]/(X^2+1)\cong\mathbb{C}$
\end{example}
\begin{proof}
  $\mathbb{R}[X]$から$\mathbb{C}$への写像を
  \[
    \varphi:\mathbb{R}[X]\to\mathbb{C};f(X)\mapsto f(\sqrt{-1})
  \]
  で定義すると、これは環準同型である。
  実際、$\varphi(f(X)+g(X))=f(\sqrt{-1})+g(\sqrt{-1})=\varphi(f(X))+\varphi(g(X))$、$\varphi(f(X)g(X))=f(\sqrt{-1})g(\sqrt{-1})=\varphi(f(X))\varphi(g(X))$、$\varphi(1)=1$を満たしている。

  また、$\varphi$は全射である。
  実際、任意の複素数$\alpha:=a+b\sqrt{-1}\in\mathbb{C}$に対して、$f(X):=a+bX$とおけば、あきらかに$\varphi(f(X))=\alpha$。
  すなわち$\operatorname{Im}(\varphi)=\mathbb{C}$である。

  準同型定理を使うため、$\operatorname{Ker}(\varphi)=(X^2+1)$を証明する。
  まず任意の$f(X)\in(X^2+1)$は、$f(X)=g(X)(X^2+1)$となる$g(X)\in\mathbb{R}[X]$が存在するため、
  \[
    \varphi(f(X))=\varphi(g(X)(X^2+1))=g(\sqrt{-1})(\sqrt{-1}^2+1)=g(\sqrt{-1})\cdot0=0
  \]
  ゆえに$f(X)\in\operatorname{Ker}(\varphi)$。
  逆に、任意の$f(X)\in\operatorname{Ker}(\varphi)$とする。
  このとき多項式の除法より、$f(X)=(X^2+1)q(X)+r(X)$となる$q(X),r(X)\in\mathbb{R}[X]$が存在し、$\deg r<2$である。
  したがって、$r(X)=a+bX$ (${}^\exists a,b\in\mathbb{R}$)と表せる。
  このとき、$f(X)\in\operatorname{Ker}(\varphi)$すなわち$\varphi(f(X))=f(\sqrt{-1})=0$であったから、
  \[
    0=f(\sqrt{-1})=(\sqrt{-1}^2+1)q(\sqrt{-1})+r(\sqrt{-1})=r(\sqrt{-1})=a+b\sqrt{-1}=0
  \]
  従って$a=b=0$。
  ゆえに$r(X)=0$となり、$f(X)=(X^2+1)q(X)\in(X^2+1)$である。
  
  以上より、$\operatorname{Ker}(\varphi)=(X^2+1)$が示された。
  したがって、環準同型定理より、$\mathbb{R}[X]/(X^2+1)\cong\operatorname{Im}(\varphi)=\mathbb{C}$が成り立つ。
\end{proof}

環準同型定理はこんな感じで、準同型があればいつでも気楽に使えます。
大学の演習問題で綺麗な準同型作って準同型定理でビシッと決めると、その時だけに分泌される脳内物質の存在を感じます。



\section{ユークリッド整域}

先ほど\textbf{多項式の割り算}をしれっと行いました。
高校数学でもやったことあると思います。
整数の時と同じように、多項式$f$を$g$で割るという操作
\[
  f=gq+r,\quad {}^\exists q,r \text{ s.t. } \deg(r)<\deg(g)
\]
ができるのです。
こうみると、\textbf{ユークリッド互除法}とまったく同じことを多項式でやっていると言えます。
しかし残念ながら、任意の可換環がそのようなアルゴリズムが存在するとは限りません。

まず良くない可能性は、0じゃない元同士の積が0になることがあるということです。
\begin{example}
  $\mathbb{Z}/6\mathbb{Z}$において、$2\times3=0$。
\end{example}
このようなことが起きると、例えば「$f$を$g$で割ります」といっても、$gq=0$となる$q$が存在すると、$f=r$になってしまって「割り算とはいったい…?」という状態になります。
こういう不都合な元には名前がついています。
\begin{definition}
  $A$を可換環とする。$a\in A\setminus\{0\}$が\textbf{零因子}であるとは、ある$b\in A$が存在して、$ab=0$を満たす時を言う。
\end{definition}

零因子がない環というのは素敵なものですので、素敵な名前がついています。
\begin{definition}
  零因子を持たない可換環$A$を\textbf{整域}(\textit{Domain})と呼ぶ。
\end{definition}
別の言い方をすれば、可換環$A$が整域であることとは
\begin{itemize}
  \item 任意の$a\in A$、$b\in A$に対して、$ab=0$ならば$a=0$または$b=0$
  \item $(0)\subset A$は素イデアルである
\end{itemize}
ということです。

さて、整域においてでさえユークリッドの互除法を導くことができません。
整数においても多項式においても、\textbf{剰余の大きさが割った元より小さくなっていかないといけない}のですが、そもそも\textbf{元の大きさ}を可換環の定義では行っていません。
なので、付け加える必要があります。
新たな環の出現です!
\begin{definition}
  整域$A$が\textbf{ユークリッド環}(\textit{Euclidean domain}\footnote{
    直訳すると「ユークリッド整域」になりそうですが、最近の本はユークリッド環と訳すのが主流のような雰囲気をそこはかとなく感じています。
  })であるとは、写像
  \[
    |\cdot|:A\to \mathbb{Z}_{\geq0}
  \]
  が存在して、次を満たす時を言う。
  すなわち、任意の$a,b\in A$に対して、$b\neq0$ならば、ある$q,r\in A$が存在して、
  \[
    a=bq+r
  \]
  を満たし、$r=0$でないならば
  \[|r|<|b|\]
  である。
\end{definition}

  \chapter{可換環論 初級}

こちらは$\text{ユークリッド環}\subset\text{PID}$の紹介と証明から、Noether環に至るまでを中心に、その他もろもろ解説していきます。

\section{ユークリッド環}

前章の例題で、\textbf{多項式の割り算}をしれっと行いました。
高校数学でもやったことあると思います。
整数の時と同じように、多項式$f$を$g$で割るという操作
\[
  f=gq+r,\quad {}^\exists q,r \text{ s.t. } \deg(r)<\deg(g)
\]
ができるのです。
こうみると、\textbf{ユークリッド互除法}とまったく同じことを多項式でやっていると言えます。
しかし残念ながら、任意の可換環がそのようなアルゴリズムが存在するとは限りません。

まず良くない可能性は、0じゃない元同士の積が0になることがあるということです。
\begin{example}
  $\mathbb{Z}/6\mathbb{Z}$において、$2\times3=0$。
\end{example}
このようなことが起きると、例えば「$f$を$g$で割ります」といっても、$gq=0$となる$q$が存在すると、$f=r$になってしまって「割り算とはいったい…?」という状態になります。
こういう不都合な元には名前がついています。
\begin{definition}
  $A$を可換環とする。$a\in A\setminus\{0\}$が\textbf{零因子}であるとは、ある$b\in A$が存在して、$ab=0$を満たす時を言う。
\end{definition}

零因子がない環というのは素敵なものですので、素敵な名前がついています。
\begin{definition}
  零因子を持たない可換環$A$を\textbf{整域}(\textit{Domain})と呼ぶ。
\end{definition}
別の言い方をすれば、可換環$A$が整域であることとは
\begin{itemize}
  \item 任意の$a\in A$、$b\in A$に対して、$ab=0$ならば$a=0$または$b=0$
  \item $(0)\subset A$は素イデアルである
\end{itemize}
ということです。

さて、整域においてでさえユークリッドの互除法を導くことができません。
整数においても多項式においても、\textbf{剰余の大きさが割った元より小さくなっていかないといけない}のですが、そもそも\textbf{元の大きさ}を可換環の定義では行っていません。
なので、付け加える必要があります。
新たな環の出現です!
\begin{definition}
  整域$A$が\textbf{ユークリッド環}(\textit{Euclidean domain}\footnote{
    直訳すると「ユークリッド整域」になりそうですが、最近の本はユークリッド環と訳すのが主流のような雰囲気をそこはかとなく感じています。
  })であるとは、写像
  \[
    |\cdot|:A\setminus\{0\}\to \mathbb{Z}_{\geq0}
  \]
  が存在して、次を満たす時を言う。
  すなわち、任意の$a,b\in A$に対して、$b\neq0$ならば、ある$q,r\in A$が存在して、
  \[
    a=bq+r
  \]
  を満たし、$r=0$でないならば
  \[|r|<|b|\]
  である。

  上記の条件を満たす写像$|\cdot|:A\setminus\{0\}\to \mathbb{Z}_{\geq0}$を\textbf{ユークリッドノルム}と呼ぶ。
\end{definition}

\subsection{ユークリッド環の例}

非常に当たり前ですが、体はユークリッド環です。
\begin{example}
  体$K$は、
  \[
    |\cdot|:K\setminus\{0\}\to\mathbb{Z}_{\geq0};x\mapsto 0
  \]
  をユークリッドノルムとするユークリッド環である。
\end{example}
\begin{proof}
  $a,b\in K$に対して、$b\neq0$ならば、$q=b^{-1}a$、$r=0$とおけば
  \[
    bq=b(b^{-1}a)=a
  \]
  となるから、$|\cdot|$はユークリッドノルムである。

  整域であることを示す。
  $a,b\in K$に対して$ab=0$とすれば、$a\neq0$ならば、両辺に$a^{-1}$を掛けることで$b=0$を得る。
  ゆえに$K$は整域である\footnote{
    こういう整域であることの証明方法は随所で見られます。
    ロジックが理解できていれば説明は不要でしょうけれども、突然このように言われるとギョッとするかもしれません。
    このような証明が成り立つ理由は、整域の定義の一つ「$ab=0$ならば$a=0$または$b=0$」と排中律から来ています。
    つまり、$ab=0$かつ$a\neq0$から$b=0$を示すことができれば、\textbf{あるいは$a=0$のとき}という排中律がすぐさま$a=0$を示すためです。
  }。
\end{proof}

もちろん、今まで見てきた整数環$\mathbb{Z}$や、体$K$上の一変数多項式環$K[X]$はユークリッド環です。
\begin{example}
  整数環$\mathbb{Z}$は、通常の絶対値
  \[
    \mathbb{Z}\setminus\{0\}\to\mathbb{Z}_{\geq0};n\mapsto|n|
  \]
  をユークリッドノルムとするユークリッド環である。
\end{example}
\begin{proof}
  ユークリッドの互除法そのものであるから自明。
\end{proof}

\begin{example}
  $K$を体とする。このとき一変数多項式環$K[X]$は、多項式$f$の$X$に関する次数$\deg(f)$をユークリッドノルムとするユークリッド環である。
\end{example}
ほぼ自明ですが証明します。
\begin{proof}
  まず整域であることを示すために、$f,g\in K[X]\setminus\{0\}$かつ$fg=0$とする。
  もし$\deg(f)=\deg(g)=0$であれば、すなわち$f,g\in K$であるから、$f=0$または$g=0$。
  ゆえにどちらか一方の次数は$0$でないとする(背理法の仮定)。
  このとき、$f$の最高次の係数を$a_f\in K\setminus\{0\}$、$g$の最高次の係数を$a_g\in K\setminus\{0\}$とすれば、
  \[
    fg=a_fa_gX^{\deg(f)+\deg(g)}+\cdots
  \]
  となっていて、かつ$K$は整域ゆえ$a_fa_g\neq0$かつ、$\deg(f)>0$または$\deg(g)>0$だったから$\deg(f)+\deg(g)>0$。
  従って$fg\neq0$となってしまい、矛盾する。
  よって$\deg(f)=\deg(g)=0$でなければならず、従って上述の通り$f=0$または$g=0$ゆえ、$K[X]$は整域である。

  $\deg$がユークリッドノルムであることを示すために、$f,g\in K[X]$、$g\neq0$とする。
  \begin{itemize}
    \item $\deg(f)<\deg(g)$ならば、$q=0,r=f$とおいて、$f=qg+r$かつ、$\deg(r)<\deg(g)$を満たしている。
    \item $\deg(f)\geq\deg(g)$のとき、$f$は
    \[
      f=a_nX^n+\cdots+a_{\deg(g)}X^{\deg(g)}+r_0,\quad\deg(r_0)<\deg(g)
    \]
    とおくことができる。
    ここで$g$の最高次の係数を$b$とおくと、明らかに
    \begin{align*}
      \deg(f-a_nb^{-1}X^{n-\deg(g)}g)\leq n-1
    \end{align*}
    そこで、
    \[
      f_1:=f-a_nb^{-1}X^{n-\deg(g)}g
    \]
    とおくと、この最高次の係数も、ある$c\in K$によって
    \[
      \deg(f_1-cX^{n-1-\deg(g)}g)\leq n-2
    \]
    と消していくことができる\footnote{
      これは$\deg(f_1)=n-1$のときにこうします。
      もしかすると$f-a_nb^{-1}X^{n-\deg(g)}g$の時点で$n-1$次以下の項も消えてるかもしれません。
      まあそうであっても、やることは同じです。
    }。
    これを繰り返せば、ある$q,r\in K[X]$が存在して、
    \[
      f=qg+r,\quad\deg(r)<\deg(g)
    \]
    を満たすことがわかる。
  \end{itemize}
\end{proof}



\section{単項イデアル整域}

一変数多項式環はユークリッド環でしたが、2変数にするとユークリッド環ではなくなります。
その理由は以下になります。
\begin{lemma}
  $K[X,Y]$のイデアル$(X,Y)$は単項イデアルではない。
\end{lemma}
\begin{proof}
  もし単項イデアルであるとすると、ある$f\in K[X,Y]$が存在して、$(X,Y)=(f)$となる。
  このとき$X\in(f)$かつ$Y\in(f)$であるから、
  \[
    X=fg,\quad Y=fh\quad({}^\exists g,h\in K[X,Y])
  \]
  と書ける。
  次数を比較すると、$\deg_X(X)=\deg_X(f)+\deg_X(g)=1$であるから、$\deg_X(f)$は0か1である。
  ここで$\deg_X$は$X$に関する次数を表わす。
  もし$\deg_X(f)=1$とすると、$Y=fh$において左辺の$X$次数は0であるから、$\deg_X(h)=-1$とならざるを得ず矛盾する。
  したがって$\deg_X(f)=0$、すなわち$f$は$X$を含まない多項式である。
  同様に$Y=fh$から$\deg_Y(f)=0$も言えるため、$f$は定数、すなわち$f\in K$である。
  
  ここで、もし$f=0$ならば$(f)=\{0\}$となり$(X,Y)$と一致しない。
  ゆえに$f\in K\setminus\{0\}$である。
  したがって$1=f\cdot f^{-1}\in(f)=(X,Y)$となる。
  すなわち、ある$p,q\in K[X,Y]$が存在して
  \[
    1=Xp+Yq
  \]
  と書けることになるが、両辺に$(0,0)$を代入すると$1=0$となり矛盾する。
  以上より、$(X,Y)$を生成する単一の多項式$f$は存在しない。
\end{proof}

なぜこれで$K[X,Y]$がユークリッド環でないことが分かるかというと、次の定理が知られているからです。
\begin{theorem}
  $A$をユークリッド環とする。このとき、$A$の任意のイデアルは単項イデアルである。
\end{theorem}
\begin{proof}
  $\mathfrak{a}$を$A$のイデアルとする。
  $\mathfrak{a}=(0)$ならば$(0)$という単項イデアルであるからよい。
  $\mathfrak{a}\neq(0)$とする。
  このとき、$\mathfrak{a}$の$0$でない元の中で、ノルム$|x|$が最小となる元を一つ選び、$d$とする。
  この$d$がイデアルすべてを生成すること、すなわち$\mathfrak{a}=(d)$を示そう。
  
  $d\in\mathfrak{a}$であるから、$(d)\subset\mathfrak{a}$は明らか。
  逆に任意の$a\in\mathfrak{a}$をとる。
  ユークリッド環の定義より、
  \[
    a = dq + r \quad (r=0 \text{ または } |r|<|d|)
  \]
  となる$q,r \in A$が存在する。
  変形すると$r = a - dq$である。
  $a\in\mathfrak{a}$かつ$d\in\mathfrak{a}$より、$r\in\mathfrak{a}$である。
  
  ここで、もし$r\neq 0$だとすると、$|r|<|d|$となり、$d$がノルム最小の元として選ばれたことに矛盾する。
  したがって$r=0$でなければならない。
  ゆえに$a=dq \in (d)$となり、$\mathfrak{a}\subset(d)$。
  以上より$\mathfrak{a}=(d)$である。
\end{proof}

これにより$K[X,Y]$がユークリッド環でないことが分かったばかりか、整数環$\mathbb{Z}$や多項式環$K[x]$のイデアルが、すべて1つの元で生成されることが保証されました!
このような整域には名前を付けてあげましょう。
\begin{definition}
  整域$A$が\textbf{単項イデアル整域}(\textit{principal ideal domain}, 略してPID)であるとは、任意のイデアルが単項イデアルであるときをいう。
\end{definition}
先ほど示したことは、\textbf{ユークリッド環はPIDである}ということです。



\section{Noether環}

すべてのイデアルが単項イデアル、という面白い可換環を知ったことで、もう一つ興味が出てきます。
ちょっとだけPIDの定義を拡張して、「任意のイデアルが有限生成だったら?」ということです。
この疑問を持った君はするどい!
これは実は代数幾何学で空気のように扱われる概念です。
\begin{definition}[Noether環]
  任意のイデアルが有限生成であるような可換環$A$を\textbf{Noether環}(\textit{Noetherian ring})と呼ぶ。
\end{definition}

実はもともとNoether環は、不変式論と呼ばれる分野での\textbf{Hilbertの基底定理}および、代数的整数論におけるデデキントによる\textbf{イデアルの昇鎖列が止まるか問題}に対して提唱されました。

\subsection{コラム:Noether環はどこから?私は代数的整数論から}

Fermat予想へのLam\'eとKummerの試みに話を戻しましょう。
彼らは$\mathbb{Z}[\zeta_{23}]$で素因数分解の一意性が成り立たないことに気が付いたのでした(リウヴィル(Liouville)などの指摘もあり)。
この一意性問題を、Kummerは数そのものではなく、それが生成するイデアルだと思うことで「素イデアル分解の一意性」として解決します。

$\mathbb{Z}[\zeta_{23}]$で考えるのは少し大変なので、よく知られた例である
\[
  A=\mathbb{Z}[\sqrt{-5}]
\]
を考えましょう。
この環では、
\[
  6=2\cdot3=(1+\sqrt{-5})(1-\sqrt{-5})
\]
というように、素因数分解の一意性が成り立っていない例が見られます。
イデアルではどう考えるかというと、まずイデアルの積を考えないといけないですね。
\begin{definition}
  $A$を可換環、$\mathfrak{a}$と$\mathfrak{b}$を$A$のイデアルとする。
  このとき、$\mathfrak{a}$と$\mathfrak{b}$の元すべての組の積が生成するイデアル、つまり
  \[
    \mathfrak{a}\mathfrak{b}:=\left(\{ab\in A\mid a\in\mathfrak{a}, b\in\mathfrak{b}\}\right)
  \]
  を、\textbf{イデアル$\mathfrak{a}$と$\mathfrak{b}$の積}と呼ぶ。
\end{definition}

簡単なところから計算例をお見せしましょう。
\begin{example}
  $\mathbb{Z}$において、
  \[
    (6)=(2)(3)
  \]
\end{example}
\begin{proof}
  $(2)=\{2n\mid n\in\mathbb{Z}\}$、$(3)=\{3m\mid m\in\mathbb{Z}\}$であるから、$(2)(3)$の元は、$6k$という形をしている。
  すなわち$(2)(3)\subset(6)$。
  逆に$n\in(6)$とすると、これはすなわち$n=6m$ (${}^\exists m\in\mathbb{Z}$)。
  このとき、$2\in(2)$、$3m\in(3)$ゆえに$2\cdot 3m=6m=n\in(2)(3)$。
  すなわち$(6)\subset(2)(3)$、ゆえに$(6)=(2)(3)$。
\end{proof}

実は定義から明らかに、もし$\mathfrak{a}=(a_1,\dots,a_n)$、$\mathfrak{b}=(b_1,\dots,b_m)$なら、
\[
  \mathfrak{a}\mathfrak{b}=(a_1b_1,a_1b_2,\dots,a_1b_m,a_2b_1,\dots,a_2b_m,\dots,a_nb_m)
\]
です。

さて、$\mathbb{Z}[\sqrt{-5}]$ではどうでしょうか?
まず$(6)=(2)(3)$は$\mathbb{Z}[\sqrt{-5}]$でも正しいです。
しかしイデアルの世界では、これをさらに分解することができるのです!

まず、実のところ$(2)$や$3$は$\mathbb{Z}[\sqrt{-5}]$では素イデアルではないという問題があります。
\begin{example}
  イデアル$(2)\subset\mathbb{Z}[\sqrt{-5}]$は素イデアルではない。
\end{example}
\begin{proof}
  $1+\sqrt{-5}\in\mathbb{Z}[\sqrt{-5}]$は$(2)$に含まれない。
  実際、もし$1+\sqrt{-5}\in(2)$とすると、ある$a+b\sqrt{-5}\in\mathbb{Z}[\sqrt{-5}]$が存在して、$1+\sqrt{-5}=2(a+b\sqrt{-5})$。
  ところがこれらの複素数が一致するためには、$1=2a$、$1=2b$が成り立たなければならないが、そのような$a,b\in\mathbb{Z}$は存在しない。
  ゆえに$1+\sqrt{-5}\notin\mathbb{Z}[\sqrt{-5}]$。

  ところが、$(1+\sqrt{-5})^2=1+2\sqrt{-5}-5=-4+2\sqrt{-5}=2(-2+\sqrt{-5})\in(2)$。
\end{proof}
\begin{example}
  イデアル$(3)\subset\mathbb{Z}[\sqrt{-5}]$は素イデアルではない。
\end{example}
\begin{proof}
  $1+\sqrt{-5},1-\sqrt{-5}\notin(3)$である。これは上記と同様である。
  一方で$(1+\sqrt{-5})(1-\sqrt{-5})=1+5=6\in(3)$
\end{proof}

素イデアルというのは、極大イデアルが素イデアルであったことからわかるように、「かなり大きい」集合になっています。
素イデアルでない$(2)$や$(3)$を素イデアルにするには、元が足りないのです。
そこで上記証明でも障害となった二つの元に注目します。
\begin{example}
  $(2,1+\sqrt{-5})$は極大イデアルであり、従って素イデアルである。
\end{example}
\begin{proof}
  準同型
  \[
    f:\mathbb{Z}\to\mathbb{Z}[\sqrt{-5}]/(2,1+\sqrt{-5});n\mapsto(2,1+\sqrt{-5})
  \]
  は全射である。
  実際、任意の$a+b\sqrt{-5}+(2,1+\sqrt{-5})\in\mathbb{Z}[\sqrt{-5}]/(2,1+\sqrt{-5})$に対して、
  \[
    a-b=(a+b\sqrt{-5})-b(1+\sqrt{-5}) \text{ 従って } a-b\equiv a+b\sqrt{-5}\mod(2,1+\sqrt{-5})
  \]
  であるから、$f(a-b)=a+b\sqrt{-5}+(2,1+\sqrt{-5})$。

  また$\operatorname{Ker}(f)$を求めるために、$f(n)=0+(2,1+\sqrt{-5})$すなわち$n+(2,1+\sqrt{-5})=0+(2,1+\sqrt{-5})$とする。
  このとき、$n\in(2,1+\sqrt{-5})$であるから、ある$a+b\sqrt{-5},c+d\sqrt{-5}\in\mathbb{Z}[\sqrt{-5}]$が存在して、
  \[
    n=2(a+b\sqrt{-5})+(1+\sqrt{-5})(c+d\sqrt{-5})=2a+c-5d+(2b+c+d)\sqrt{-5}
  \]
  を満たす。
  この複素数の等式を満たすには、
  \[
    \begin{cases}
      n=2a+c-5d\\
      2b+c+d=0
    \end{cases}
  \]
  を満たさなければならない。
  第二式より$c=-2b-d$であるから、これを第一式に代入すると
  \[
    n=2a-2b-6d
  \]
  となる。
  $a,b,d$は任意の整数を動くから、$n$は任意の偶数の値を取ることができる。
  すなわち$\operatorname{Ker}(f)\subset2\mathbb{Z}$。
  逆に任意の偶数$2m$に対して、$2m\in(2,1+\sqrt{-5})$であるから$f(2m)=0+(2,1+\sqrt{-5})$。
  ゆえに$\operatorname{Ker}(f)=2\mathbb{Z}$である。

  従って準同型定理より、
  \[
    \mathbb{Z}/2\mathbb{Z}\cong\mathbb{Z}[\sqrt{-5}]/(2,1+\sqrt{-5})
  \]
  となる。左辺は体であるから、$(2,1+\sqrt{-5})$は極大イデアルである。
\end{proof}

同様にして次が成り立ちます。
\begin{example}
  $(3,1\pm\sqrt{-5})$は$\mathbb{Z}[\sqrt{-5}]$の極大イデアルであり、従って素イデアルである。
\end{example}
\begin{proof}
  \[
    f:\mathbb{Z}\to\mathbb{Z}[\sqrt{-5}]/(3,1\pm\sqrt{-5})
  \]
  が全射であり、$\operatorname{Ker}(f)=3\mathbb{Z}$になることが、上記と同様にわかる。
  従って準同型定理より$\mathbb{Z}/3\mathbb{Z}\cong\mathbb{Z}[\sqrt{-5}]/(3,1\pm\sqrt{-5})$であるが、左辺は体であるから$(3,1\pm\sqrt{-5})$は極大イデアルである。
\end{proof}

さてここまでやったら、色々試してみればいいでしょう。
数学書は綺麗になった理論しか読者に見せませんが、裏では泥臭い計算が行われているものです。
多分Kummerもいろんなイデアルの積を考えまくったことでしょう。
とはいえ結論、次のようになります。
\begin{example}
  $\mathbb{Z}[\sqrt{-5}]$において、
  \[
    (2)=(2,1+\sqrt{-5})^2
  \]
\end{example}
\begin{proof}
  $(2)\subset(2,1+\sqrt{-5})^2$を示すためには、$2\in(2,1+\sqrt{-5})^2$を示せばよい。
  \[
    (2,1+\sqrt{-5})^2=(4,2+2\sqrt{-5},-4+2\sqrt{-5})
  \]
  であるから、
  \[
    2=-4+(2+2\sqrt{-5})-(-4+2\sqrt{-5})\in(4,2+2\sqrt{-5},-4+2\sqrt{-5})=(2,1+\sqrt{-5})^2
  \]
  逆に、$(2,1+\sqrt{-5})^2\subset(2)$を示すには、$4,2+2\sqrt{-5},-4+2\sqrt{-5}\in(2)$を示せばよいが、これは明らか。
  従って題意が示された。
\end{proof}
\begin{example}
  $\mathbb{Z}[\sqrt{-5}]$において、
  \[
    (3)=(3,1+\sqrt{-5})(3,1-\sqrt{-5})
  \]
\end{example}
\begin{proof}
  $(3)\subset(3,1+\sqrt{-5})(3,1-\sqrt{-5})$を示すためには、$3\in(3,1+\sqrt{-5})(3,1-\sqrt{-5})$を示せばよい。
  \[
    (3,1+\sqrt{-5})(3,1-\sqrt{-5})=(9,3+3\sqrt{-5},3-3\sqrt{-5},6)
  \]
  であるから、
  \[
    3=9-6\in(3,1+\sqrt{-5})(3,1-\sqrt{-5})
  \]
  逆に、$(3,1+\sqrt{-5})(3,1-\sqrt{-5})\subset(3)$を示すには、$9,3+3\sqrt{-5},3-3\sqrt{-5},6\in(2)$を示せばよいが、これは明らか。
  従って題意が示された。
\end{proof}
上記の素イデアルを
\begin{align*}
  \mathfrak{p}_2&:=(2,1+\sqrt{-5})\\
  \mathfrak{p}_3&:=(3,1+\sqrt{-5})\\
  \overline{\mathfrak{p}_3}&:=(2,1-\sqrt{-5})
\end{align*}
とおけば、以上から$\mathbb{Z}[\sqrt{-5}]$における$(6)$の素イデアル分解
\[
  (6)=\mathfrak{p}_2^2\mathfrak{p}_3\overline{\mathfrak{p}_3}
\]
が得られました。

Kummerが考えたこの素イデアル分解ですが、実は任意の可換環で一意性が成り立つとは限りませんし、そもそも分解が有限回で終わるとも限りません。
しかし、Noether環であれば、少なくとも「分解操作がいつか終わる」ことだけは保証できます。
理由はイデアルの「大きさ」に関する直感の逆転にあります。
通常の数では、割れば割るほど数字は小さくなりますが、イデアルの世界では「割り切れる」ことは「包含される」ことを意味します。
実際、6は2より大きい数ですが、イデアルとしては$(6)\subset(2)$というように、約数の方が集合として大きくなるのです。
したがって、「分解を繰り返していく」という操作は、イデアルの包含関係において
\[
  \mathfrak{a}_1 \subset \mathfrak{a}_2 \subset \mathfrak{a}_3 \subset \cdots
\]
というように、集合がどんどん大きくなっていく「昇鎖」を作ることに他なりません。
分解が必ず終わるためには、この拡大がどこかで止まる必要があります。
すなわち、任意の昇鎖列に対してある番号$N$が存在し、
\[
  \mathfrak{a}_N = \mathfrak{a}_{N+1} = \mathfrak{a}_{N+2} = \cdots
\]
となって停留する(これ以上大きくならない)ことが条件となります。
これをイデアルの\textbf{昇鎖条件}(\textit{Ascending Chain Condition}, ACC)といいます。
実は、この条件こそがNoether環の定義なのです。

\begin{theorem}
  可換環$A$について、次は同値である。
  \begin{enumerate}
    \item $A$はイデアルの昇鎖条件(以降ACC)を満たす。
    \item $A$の任意のイデアルは有限生成である。
  \end{enumerate}
\end{theorem}
\begin{proof}
  (1 $\implies$ 2)
  $A$はイデアルのACCを満たすとし、$\mathfrak{a}$をそのイデアルとする。
  背理法を用いる。もし$\mathfrak{a}$が有限生成でないと仮定する。
  まず、$a_1 \in \mathfrak{a}$を選び、イデアル$\mathfrak{a}_1 = (a_1)$を作る。
  $\mathfrak{a}$は有限生成ではないので、$\mathfrak{a} \neq \mathfrak{a}_1$である。
  そこで、$a_2 \in \mathfrak{a} \setminus \mathfrak{a}_1$を選び、$\mathfrak{a}_2 = (a_1, a_2)$を作る。
  まだ$\mathfrak{a}$は有限生成でないので、$\mathfrak{a} \neq \mathfrak{a}_2$である。
  これを繰り返して、$a_n \in \mathfrak{a} \setminus \mathfrak{a}_{n-1}$を選び続け、$\mathfrak{a}_n = (a_1, \dots, a_n)$を作ると、
  \[
    \mathfrak{a}_1 \subsetneq \mathfrak{a}_2 \subsetneq \mathfrak{a}_3 \subsetneq \dots
  \]
  という真に増大していくイデアルの列ができてしまう。
  これは$A$が昇鎖条件を満たすことに矛盾する。
  したがって、$\mathfrak{a}$は有限生成でなければならない。

  (2 $\implies$ 1)
  $A$の任意のイデアルが有限生成であるとする。
  イデアルの昇鎖列 $\mathfrak{a}_1 \subset \mathfrak{a}_2 \subset \dots$ を考える。
  ここで、これらの和集合
  \[
    \mathfrak{a} := \bigcup_{i=1}^{\infty} \mathfrak{a}_i
  \]
  を考えると、これもイデアルになる。
  実際、明らかに$0\in\mathfrak{a}$。
  また任意の$x,y\in\mathfrak{a}$に対して、$x\in\mathfrak{a}_n$、$y\in\mathfrak{a}_m$とすると、$n\leq m$ならば$x,y\in\mathfrak{a}_m$であるから$x+y\in\mathfrak{a}_m\subset\mathfrak{a}$。
  逆に$m\leq n$でも同様。
  最後に、$a\in A$かつ$x\in\mathfrak{a}$とすると、ある$n$に対して$x\in\mathfrak{a}_n$ゆえ$ax\in\mathfrak{a}_n\subset\mathfrak{a}$。

  さて、仮定より$\mathfrak{a}$は有限生成なので、生成元 $x_1, \dots, x_r$ が存在する。
  すなわち $\mathfrak{a} = (x_1, \dots, x_r)$。
  各生成元 $x_j$ は和集合のどこかの段階で入ってきたはずなので、ある番号 $n_j$ が存在して $x_j \in \mathfrak{a}_{n_j}$ となる。
  ここで $N = \max(n_1, \dots, n_r)$ とおくと、すべての $j$ について $x_j \in \mathfrak{a}_N$ となる。
  したがって、
  \[
    \mathfrak{a} = (x_1, \dots, x_r) \subset \mathfrak{a}_N \subset \mathfrak{a}_{N+1} \subset \dots \subset \mathfrak{a}
  \]
  となるため、$\mathfrak{a}_N = \mathfrak{a}_{N+1} = \dots = \mathfrak{a}$ となり、列は停留する。
\end{proof}

\subsection{コラム:Noether環はどこから?私は不変式論から}

不変式とは、ざっくり言えば「座標変換しても値が変わらない式」のことです。
最も有名な例は、2次方程式の判別式でしょう。
\[
  f(x,y)=ax^2+2bxy+cy^2
\]
という2変数2次形式を考えます(計算の都合上$2b$としています)。
この式の形を決めるのは係数$a,b,c$ですが、これらが作る判別式
\[
  D=b^2-ac
\]
は、この式の「ある性質」を表わしています。

座標変換を行ってみましょう。
行列
\[
  M = \begin{pmatrix}
    p & q \\
    r & s
  \end{pmatrix}
\]
を用いて、変数を
\[
  \begin{pmatrix}
    x \\ y
  \end{pmatrix}
  =
  M
  \begin{pmatrix}
    x' \\ y'
  \end{pmatrix}
\]
と取り替えたとします。
これを元の式$f(x,y)$に代入して整理すると、新しい係数$a', b', c'$を持つ式
\[
  f(x',y')=a'x'^2+2b'x'y'+c'y'^2
\]
が得られます。当然、係数$a', b', c'$は元の$a,b,c$と$p,q,r,s$が複雑に混ざった値になります。
しかし、驚くべきことに、新しい係数で作った判別式$D' = (b')^2 - a'c'$を計算すると、
\[
  D'=(\det(M))^2D
\]
という非常に綺麗な関係が成り立つのです。
特に行列式が1の変換($\det(M)=1$)に限れば、$D'=D$となり、値が全く変わりません。
このように、座標を変えても不変な式のことを\textbf{不変式}と呼びます。

19世紀の数学者たちは考えました。
「もっと変数を増やしたり、次数を上げたりしても、このような不変式をすべて見つけ出すことができるだろうか?」
彼らの目標は、あらゆる不変式を
\[
  I=F(J_1,J_2,\dots,J_n)
\]

というように、有限個の「基本となる不変式$J_1,\dots,J_n$」の多項式として書き表すことでした。
これを「不変式環の有限生成問題」といいます。

これは「泥沼」でした。
変数が2個、3個と増えるにつれ、計算量は爆発的に増大します。
当時の計算の大家ゴルダン(Gordan)は、凄まじい計算力で2変数の場合を解決しましたが、3変数以上では誰も歯が立ちませんでした。
「基本となる不変式が有限個である」ことを示すために、具体的にそれらを見つけようとすると、宇宙が終わるほどの時間がかかってしまうのです。

そこに現れたのが、Noetherの師であるHilbertです。
彼は具体的な$J_i$を計算するのをやめました。
代わりにこう考えたのです。
「不変式たちが生成する\textbf{イデアル}を考えよう」

彼は、多項式環の強力な性質を証明しました。
\begin{theorem}[Hilbertの基底定理]
  体$K$上の多項式環$R=K[X_1, \dots, X_n]$の任意のイデアル$\mathfrak{a}$は有限生成である。
  すなわち、有限個の元$f_1, \dots, f_m$が存在して、
  \[
    \mathfrak{a}=(f_1,\dots,f_m)
  \]
  と書ける。
\end{theorem}
\begin{proof}
  1変数の場合、$R=K[X]$がNoether環であることを示せばよい。
  これは次数をユークリッドノルムとして$K[x]$がユークリッド環になることからPIDであり、従って全てのイデアルは一つ、すなわち有限の元で生成される。

  一般の場合、$A:=[X_1,\dots,X_{n-1}]$がNoether環であると仮定したとき、$A[X_n]$もまたNoetherであることを示す。
  これはより一般に、次の定理によってわかる。
\end{proof}
\begin{theorem}
  $A$がNoether環ならば、$A[X]$もNoether環である。
\end{theorem}
\begin{proof}
  $\mathfrak{f}\subset A[X]$を任意のイデアルとする。
  $\mathfrak{f}$に含まれる多項式の最高次係数全体からなる集合
  \[
    \mathfrak{a}:=\{a\in A\mid {}^\exists aX^n+a_{n-1}X^{n-1}+\cdots+a_0\in\mathfrak{f}\} \cup \{0\}
  \]
  を考えると、$\mathfrak{a}$は$A$のイデアルとなる。
  
  実際、$a,b\in\mathfrak{a}$とし、それぞれを最高次係数に持つ多項式を$f=aX^n+\cdots$、$g=bX^m+\cdots$とする($f,g\in\mathfrak{f}$)。
  一般性を失わず$n\leq m$とすると、$X^{m-n}f \in \mathfrak{f}$であり、これと$g$の和をとることで
  \[
    X^{m-n}f + g = (a+b)X^m + \cdots \in \mathfrak{f}
  \]
  となるため、$a+b\in\mathfrak{a}$である。
  また$A$倍についても同様に閉じており、$\mathfrak{a}$はイデアルである。

  $A$はNoether環であるから、$\mathfrak{a}$は有限生成である。
  \[
    \mathfrak{a}=(a_1,\dots,a_n)
  \]
  とおく。各$a_i$に対して、それを最高次係数としてもつ多項式$f_i \in \mathfrak{f}$を選び、その次数を$r_i=\deg(f_i)$とする。ここで
  \[
    r:=\max\{r_1, \dots, r_n\}
  \]
  とおく。さらに、これらの生成元で生成される$A[X]$のイデアルを
  \[
    \mathfrak{f}':=(f_1,\dots,f_n)
  \]
  とおく。明らかに$\mathfrak{f}'\subset\mathfrak{f}$である。

  任意の$f \in \mathfrak{f}$をとる。
  $f$の次数$m$が$m \geq r$であれば、その最高次係数$a$は$\mathfrak{a}$の元であるから$a=\sum c_i a_i$ ($c_i \in A$) と書ける。
  このとき、
  \[
    f - \sum_{i=1}^n c_i X^{m-r_i} f_i
  \]
  を考えると、これは$\mathfrak{f}$の元であり、かつ$m$次の項が消去されているため、次数は$m$より小さくなる。
  この操作を繰り返すことで、任意の$f \in \mathfrak{f}$に対して
  \[
    f = h + g, \quad h \in \mathfrak{f}', \quad \deg(g) < r
  \]
  となる$g \in \mathfrak{f}$が存在することがわかる。
  もし$g=0$ならば$f \in \mathfrak{f}'$であるが、一般には$g \neq 0$の可能性があるため、次数が$r$未満の多項式について考える必要がある。

  各整数$k$ ($0 \leq k < r$) に対して、次数が$k$であるような$\mathfrak{f}$の元の最高次係数全体(および$0$)の集合を$\mathfrak{a}_k$とおく。
  すなわち、
  \[
    \mathfrak{a}_k := \{ c \in A \mid \exists p \in \mathfrak{f}, \deg(p) = k, \text{の最高次係数が } c \} \cup \{0\}
  \]
  これらはそれぞれ$A$のイデアルとなるため、Noether性より有限生成である。
  各$k$について、その生成元を$c_{k,1}, \dots, c_{k, s_k}$とし、
  それらを最高次係数に持つ多項式 $g_{k,j} \in \mathfrak{f}$ ($\deg g_{k,j} = k$)を選んで固定する。

  これら有限個の多項式全体
  \[
    S := \{f_1, \dots, f_n\} \cup \{ g_{k,j} \mid 0 \leq k < r, \, 1 \leq j \leq s_k \}
  \]
  が$\mathfrak{f}$を生成することを示す。

  先ほどの剰余多項式$g$($\deg g < r$)を考える。
  $g=0$なら証明終了である。
  $g \neq 0$とし、$\deg(g) = d$ ($d < r$)、その最高次係数を$c$とする。
  定義より$c \in \mathfrak{a}_d$であるから、$c$は$\mathfrak{a}_d$の生成元$c_{d,j}$の一次結合で書ける。
  したがって、適当な係数$b_j \in A$を用いて
  \[
    g - \sum_{j=1}^{s_d} b_j g_{d,j}
  \]
  を作ると、これは$\mathfrak{f}$の元であり、かつ$d$次の項が消去されるため次数は$d$未満となる。
  
  この次数引き下げの操作を繰り返せば、最終的に次数は$0$より小さく(すなわち$0$多項式に)なる。
  ゆえに$g$は$\{g_{k,j}\}$の$A$-係数線形結合で書けることがわかり、$f$は$S$の元で生成されるイデアルに含まれる。
  
  以上より$\mathfrak{f}$は有限生成であり、$A[X]$はNoether環である。
\end{proof}

Hilbertはこの定理を使い、「具体的な形はわからないが、有限個の基底が存在することだけは間違いない」と証明してしまったのです。
これを見たGordanは「これは数学ではない、神学だ!」と叫んだとか。
こうして、「具体的な計算」の限界を「イデアルの有限生成性」という抽象的な性質が突破しました。
この「任意のイデアルが有限生成である」という性質こそ、後にEmmy Noetherが「Noether環」として定義したものの正体です。
前節で見たように、この「有限生成性」は、先ほどの「昇鎖条件」と全く同じ意味を持つのです。

ちなみに、この「変換しても変わらない性質に着目する」という思想は、同時代のフェリックス・クライン(Felix Klein)による「\textbf{エルランゲン・プログラム}」(幾何学を変換群の観点で統一する試み)と完全に共鳴するものです。
不変式論は、Kleinが描いた幾何学の設計図を、代数の計算によって具体的に実現しようとする試みだったとも言えるでしょう。
やがてこの流れは、計算(不変式)から構造(群・環・体)へと主役を移し、Noetherによる現代代数学へと結実していきます。



\section{やり残したこと}

話の流れ的にどこにも差し込みにくかった定理をここに証明します。

\begin{theorem}
  $A$を可換環、$\mathfrak{a}$を$A$のイデアルとする。
  このとき、$\mathfrak{a}$が素イデアルであることと、$A/\mathfrak{a}$が整域であることは同値である。
\end{theorem}
\begin{proof}
  $\mathfrak{a}$が素イデアルであるとし、$x+\mathfrak{a},y+\mathfrak{a}\in A/\mathfrak{a}$は$xy+\mathfrak{a}=0+\mathfrak{a}$が成り立つとする。
  すなわち$xy\in\mathfrak{a}$であるが、$\mathfrak{a}$は素イデアルであるから、$x\in\mathfrak{a}$または$y\in\mathfrak{a}$が成り立つ。
  これはすなわち$x+\mathfrak{a}=0+\mathfrak{a}$または$y+\mathfrak{a}=0+\mathfrak{a}$であるから、$A/\mathfrak{a}$は素イデアルとなる。

  逆に$A/\mathfrak{a}$が整域ならば$\mathfrak{a}$が素イデアルであることは、上記を逆にたどればわかる。
\end{proof}








  \chapter{可換環論:中級}\label{chapter-commutative-rings-second-course}

第\ref{chapter-commutative_rings_introduction}章では環とイデアルの基礎を、第\ref{chapter-commutative_rings_first_course}章ではNoether性という有限性の概念を学びました。
これくらいで一応、可換環論初級は卒業でしょうか。
初級って言ったって、僕が言ってるだけですけど。
ともかくこれで、\mycite{Har77}の2ページ目までの可換環論は攻略しました。

意気揚々と読み進めていくと、3ページ目Proposition 1.2の証明がほとんど書かれてないです。
大半は簡単なのですが、Prop 1.2(d)は\textbf{Hilbertの零点定理}と呼ばれるまあまあの大定理です。
これも可換環論の結果扱いということで、参考文献にぶん投げられています。
取り上げられているのはLangのでっかい本、我らがアティマク、古典のZariski-Samuelが挙げられています。
んで、例えば君が\mycite{At-Mc-translate}しか持ってなかったら残念!
演習問題です。
代数幾何あるあるですね。

今回はHilbertの零点定理を攻略していきましょう。



\section{イデアルの根基}

まず突然現れている「$\sqrt{\mathfrak{a}}$」ってなんやねんという話ですが、これはイデアル$\mathfrak{a}$の\textbf{根基}と呼ばれるもので、イデアルの元の重複具合を可能な限り取り除いたものと言えましょう。
例えば$\mathbb{C}[x]$において、イデアル$(x)$と$(x^2)$は代数的には異なりますが、幾何学的な零点集合(つまり$x=0$となる点と$x^2=0$となる点)はどちらも原点$0$であり、区別がつきません。
この違いを吸収するのが根基です。
\begin{definition}[根基]
  環$A$のイデアル$\mathfrak{a}$に対して、
  \[
    \sqrt{\mathfrak{a}} := \{x \in A \mid {}^\exists n > 0, x^n \in \mathfrak{a}\}
  \]
  を$\mathfrak{a}$の\textbf{根基}(\textit{radical})と呼ぶ。
\end{definition}

根基が実際にイデアルになっていることを証明しましょう。
\begin{theorem}
  環$A$のイデアル$\mathfrak{a}$に対して、
  \[
    \sqrt{\mathfrak{a}} := \{x \in A \mid {}^\exists n > 0, x^n \in \mathfrak{a}\}
  \]
  は$A$のイデアルである。
\end{theorem}
\begin{proof}
  まず、$0^1=0\in\mathfrak{a}$ゆえに$0\in\sqrt{\mathfrak{a}}$。

  次に$x,y\in\sqrt{\mathfrak{a}}$とおくと、ある$n,m>0$が存在して、$x^n,y^m\in\mathfrak{a}$。
  このとき、
  \[
    (x+y)^{n+m}=\sum_{i=0}^{n+m}{}_{n+m}C_ix^{n+m-i}y^i
  \]
  を考えると、各$x^{n+m-i}y^i$は$\mathfrak{a}$に含まれる。
  実際、$i\leq m$ならば$x^{n+m-i}y^i=(x^{m-i}y^i)x^n\in\mathfrak{a}$。
  また$i>m$ならば$x^{n+m-i}y^i=(x^{n+m-i}y^{i-m})y^m\in\mathfrak{a}$。

  最後に、$a\in A$、$x\in\sqrt{\mathfrak{a}}$とすると、ある$n>0$が存在して、$x^n\in\mathfrak{a}$。
  このとき、
  \[
    (ax)^n=a^nx^n\in\mathfrak{a}
  \]
  ゆえに$ax\in\sqrt{\mathfrak{a}}$。
\end{proof}

\subsection{根基の例}

簡単な例をいくつか見ていきましょう。
まずは一般的にすぐに分かる例を挙げます。
\begin{example}
  $A$を環、$\mathfrak{a}$をそのイデアルとすると、$\sqrt{\mathfrak{a}}=A \iff \mathfrak{a}=A$
\end{example}
\begin{proof}
  明らかに$\sqrt{A}=A$ゆえ、$\sqrt{\mathfrak{a}}=A$ならば$\mathfrak{a}=A$を示す。
  仮定より$1\in\sqrt{\mathfrak{a}}$ゆえ、$1=1^n\in\mathfrak{a}$ (${}^\exists n>0$)。
  よって$1\in\mathfrak{a}$となり$\mathfrak{a}=A$が従う。
\end{proof}

より具体的な例を見ていきましょう。
PID $\mathbb{Z}$において、根基は「$\sqrt{\mathfrak{a}}$」という記号のイメージ通りの挙動をします。
\begin{example}
  $\mathbb{Z}$において、
  \begin{itemize}
    \item $\sqrt{(4)}=(2)$
    \item $\sqrt{(9)}=(3)$
    \item $\sqrt{(12)}=(6)$
    \item $\sqrt{(125)}=(5)$
  \end{itemize}
  など。一般に、$a=p_1^{e_1}\cdots p_n^{e_n}$のとき、
  \[
    \sqrt{(p_1^{e_1}\cdots p_n^{e_n})}=(p_1\cdots p_n)
  \]
\end{example}

証明はいらないでしょう。
$\sqrt{(125)}=(5)$のように、単に平方根をとっているのではなく、とにかく重複した素因子のべき乗を排除するのです。

\subsection{根基の性質}

根基の性質で最初に伝えておくべき重要な性質は、次の簡単な定理です。
\begin{theorem}
  環$A$のイデアル$\mathfrak{a}$に対して、
  \[
    \mathfrak{a}\subset\sqrt{\mathfrak{a}}
  \]
\end{theorem}
\begin{proof}
  根基の定義から、$x\in\mathfrak{a}$ならば$x^1=x\in\mathfrak{a}$ゆえに$x\in\sqrt{\mathfrak{a}}$、つまり$\mathfrak{a}\subset\sqrt{\mathfrak{a}}$。
\end{proof}

剰余環の自然な準同型
\[
  \varphi:A\to A/\mathfrak{a};a\mapsto a+\mathfrak{a}
\]
は、「$\mathfrak{a}$を含むイデアルとそれらの包含関係のなす順序集合」と、「$A/\mathfrak{a}$のイデアルとそれらの包含関係のなす順序集合」の間の順序同型でした。
従って$\mathfrak{a}\subset\sqrt{\mathfrak{a}}$から、$\sqrt{\mathfrak{a}}$に対応する$A/\mathfrak{a}$に対応するイデアル
\[
  \mathfrak{N}_{A/\mathfrak{a}}:=\varphi(\sqrt{\mathfrak{a}})
\]
が存在します。
これはほとんど明らかに以下の性質を持ちます。
\begin{theorem}
  $A$を環、$\mathfrak{a}$をイデアルとする。
  $\varphi:A\to A/\mathfrak{a}$を剰余環の自然な準同型とし、$\mathfrak{N}_{A/\mathfrak{a}}:=\varphi(\sqrt{\mathfrak{a}})$とおく。
  このとき、
  \[
    a+\mathfrak{a}\in\mathfrak{N}_{A/\mathfrak{a}} \iff (a+\mathfrak{a})^n=0\; ({}^\exists n>0)
  \]
\end{theorem}
\begin{proof}
  \begin{align*}
    a+\mathfrak{a}\in\mathfrak{N}_{A/\mathfrak{a}} &\iff \varphi(a)\in\varphi(\sqrt{\mathfrak{a}})\\
    &\iff a\in\sqrt{\mathfrak{a}}\\
    &\iff a^n\in\mathfrak{a} ({}^\exists n>0)\\
    &\iff a^n+\mathfrak{a}=0\in A/\mathfrak{a} ({}^\exists n>0)\\
    &\iff (a+\mathfrak{a})^n=0 ({}^\exists n>0)
  \end{align*}
\end{proof}

一般に、
\begin{definition}
  $A$を環とする。
  $a\in A$が\textbf{冪零元}であるとは、ある$n>0$が存在して、$a^n=0$を満たす時をいう。
\end{definition}
\begin{definition}
  $A$を環とする。
  このとき、$A$の冪零元すべての集合がなすイデアル
  \[
    \mathfrak{N}_A:=\sqrt{(0)}=\{a\in A\mid a^n=0\; ({}^\exists n>0)\}
  \]
  を\textbf{冪零元根基}という。
\end{definition}
ので、「自然な準同型$\varphi:A\to A/\mathfrak{a}$による$\sqrt{\mathfrak{a}}$の像は冪零元根基$\mathfrak{N}_{A/\mathfrak{a}}$」ということです。



\section{分数環}

有理数体$\mathbb{Q}$は、整数環$\mathbb{Z}$から「$0$以外の元での割り算」を許すことで作られました。
一般の環でも、特定の元での割り算を許容する操作を考えることができます。

\subsection{積閉集合と分数環の定義}

割り算を許容する環を新たに作るために、分母に``なることができる"集合を一般的に考えます。
\begin{definition}[積閉集合]
  環$A$の部分集合$S$が\textbf{積閉集合}であるとは、以下を満たすときを言う。
  \begin{itemize}
    \item $1 \in S$
    \item $x, y \in S \implies xy \in S$
  \end{itemize}
\end{definition}

環$A$の積閉集合$S$に対して、$S$を分母にすることができる環を
\[
  B=\left\{\frac{a}{s}\mid a\in A, s\in S\right\}
\]
で定義できそうです。
しかしこれでは「約分」が考慮されておらず、同一視されるべき元が別々に見えてしまっています($1/2\neq2/4$みたいに見えている)。
分数が約分できる状況というのは、
\[
  \frac{a}{s}=\frac{b}{t} \iff at=bs \iff at-bs=0
\]
です。これを$B$の同値関係として導入すべきでしょうか?
答えはNO。これでは同値関係のうち、推移律を満たしません。
実際、$a/s=b/t, b/t=c/u$とすると、$at-bs=0$に$u$を、$bu-ct=0$に$s$を掛けて足し合わせることで、
\[
  0=(atu-bsu)+(bsu-cst)=atu-cst=t(au-cs)
\]
となって、$au-cs=0$とは限らないことが分かります。
なぜなら、これがゼロ因子かもしれないからです。

そこで以下の同値関係で割った集合を考えるべきでしょう。
\begin{theorem}
  $A$を環、$S$をその積閉集合とする。
  このとき、$A\times S$上の以下の関係は、同値関係である。
  \[
    (a,s) \sim (b,t) \iff {}^\exists u \in S \text{ s.t. } u(at-bs)=0
  \]
\end{theorem}
\begin{proof}
  \begin{description}
    \item[\textbf{反射律}] $(a,s)\in A\times S$に対して、$1\cdot(as-as)=0$なので$(a,s)\sim(a,s)$
    \item[\textbf{対称律}] $(a,s)\sim(b,t)$すなわち${}^\exists u\in S$ s.t. $u(at-bs)=0$とすると、$u(bs-at)=-u(at-bs)=0$ゆえに$(b,t)\sim(a,s)$
    \item[\textbf{推移律}] $(a_1,s_1)\sim(a_2,s_2)$かつ$(a_2,s_2)\sim(a_3,s_3)$すなわち${}^\exists t,u\in S$ s.t. $t(a_1s_2-a_2s_1)=0$、$u(a_2s_3-a_3s_2)=0$とする。
    このとき、
    \[
      0=s_3u\{t(a_1s_2-a_2s_1)\}+s_1t\{u(a_2s_3-a_3s_2)\}=tus_2a_1s_3-tus_2a_3s_1=tus_2(a_1s_3-a_3s_1)
    \]
    であり、$S$は積閉集合だから$tus_2\in S$より$(a_1,s_1)\sim(a_3,s_3)$
  \end{description}
\end{proof}

ゆえに集合として、$S$を分母として許容する新たな環が定義できました。
\begin{definition}[分数環]
  $A$を環、$S$をその積閉集合とする。
  このとき、$A\times S$上の同値関係
  \[
    (a,s) \sim (b,t) \iff {}^\exists u \in S \text{ s.t. } u(at-bs)=0
  \]
  に関する商集合
  \[
    S^{-1}A:=A\times S/\sim
  \]
  を、積閉集合$S$による$A$の\textbf{分数環}という。
  $S^{-1}A$の元を
  \[
    \frac{a}{s}=a/s
  \]
  などと書き表す。
\end{definition}

分数環は本来、自然に導入される環構造を含めて分数環と呼びます。
\begin{theorem}
  $A$を環、$S$をその積閉集合とする。
  このとき、$S^{-1}A$に以下の演算
  \begin{align*}
    \frac{a}{s}+\frac{b}{t}&:=\frac{at+bs}{st}\\
    \frac{a}{s}\cdot\frac{b}{t}&:=\frac{ab}{st}
  \end{align*}
  を入れることによって、$S^{-1}A$は環である。
  加法単位元は$0/1$、乗法単位元は$1/1$である。
\end{theorem}
\begin{proof}
  まず演算がうまく定義されていることを確認する。
  $a/s=a'/s'$、$b/t=b'/t'$とするとき、すなわちある$u_1,u_2\in S$が存在して、
  \[
    u_1(as'-a's)=0,\quad u_2(bt'-b't)=0
  \]
  である。
  このとき、和は
  \[
    u_1u_2\{(at+bs)s't'-(a't'+b's')st\}=u_2tt'u_1(as'-a's)+u_1ss'u_2(bt'-b't)=0
  \]
  より
  \[
    \frac{at+bs}{st}=\frac{a't'+b's'}{s't'}
  \]
  積は
  \[
    u_1u_2(abs't'-a'b'st)=u_1u_2(abs't'-a'bst'+a'bst'-a'b'st)=u_2bt'u_1(as'-a's)+u_1a'su_2(bt'-b't)=0
  \]
  より
  \[
    \frac{ab}{st}=\frac{a'b'}{s't'}
  \]
  ゆえに演算はwell-definedである。
  \begin{itemize}
    \item \textbf{演算$+$に関して可換群} 
    \begin{itemize}
      \item \textbf{結合法則} $a_1/s_1+(a_2/s_2+a_3/s_3)=a_1/s_1+(a_2s_3+a_3s_2)/s_2s_3=\{a_1s_2s_3+(a_2s_3+a_3s_2)s_1\}/s_1s_2s_3=\{(a_1s_2+a_2s_1)s_3+a_3s_1s_2\}/s_1s_2s_3=(a_1s_2+a_2s_1)/s_1s_2+a_3/s_3=(a_1/s_1+a_2/s_2)+a_3/s_3$
      \item \textbf{単位元} $a/s+0/1=(a\cdot1+0\cdot s)/s\cdot1=a/s$、$0/1+a/s=(0\cdot s+a\cdot1)/1\cdot s=a/s$
      \item \textbf{逆元} $a/s+(-a/s)=(a\cdot s-a\cdot s)/s^2=0/s^2=0/1$ ($\because$ $1\cdot(0\cdot1-0\cdot s^2)=0$)
      $(-a/s)+a/s=0/1$も同様。
      \item \textbf{可換性} $a/s+b/t=(at+bs)/st=(bs+at)/ts=b/t+a/s$
    \end{itemize}
    \item \textbf{演算$\cdot$に関して可換モノイド}
    \begin{itemize}
      \item \textbf{結合法則} $a_1/s_1\cdot (a_2/s_2\cdot a_3/s_3)=a_1/s_1\cdot(a_2a_3/s_2s_3)=a_1(a_2a_3)/s_1(s_2s_3)=(a_1a_2)a_3/(s_1s_2)s_3=(a_1a_2/s_1s_2)\cdot a_3/s_3=(a_1/s_1\cdot a_2/s_2)\cdot a_3/s_3$
      \item \textbf{単位元} $a/s\cdot 1/1=a\cdot1/s\cdot1=a/s$、$1/1\cdot a/s=1\cdot a/1\cdot s=a/s$
      \item \textbf{可換性} $a/s\cdot b/t=ab/st=ba/ts=b/t\cdot a/s$
    \end{itemize}
    \item \textbf{分配法則} $a_1/s_1\cdot(a_2/s_2+a_3/s_3)=a_1/s_1\cdot(a_2s_3+a_3s_2)/s_2s_3=a_1(a_2s_3+a_3s_2)/s_1s_2s_3=(a_1a_2s_3+a_1a_3s_2)/s_1s_2s_3=a_1a_2s_3/s_1s_2s_3+a_1a_3s_2/s_1s_2s_3=a_1a_2/s_1s_2+a_1a_3/s_1s_3=(a_1/s_1\cdot a_2/s_2)+(a_1/s_1\cdot a_3/s_3)$
  \end{itemize}
\end{proof}

さて、僕たちは積閉集合$S$にゼロ因子どころか、0が入ることを許容しています。
もし0が入っていると、分数環$S^{-1}A$は$1=0$の零環に潰れてしまいます。
実際、任意の$a/s\in S^{-1}A$に対して、$0\in S$が存在「してしまい」、
\[
  0\cdot(a\cdot 1-0\cdot s)=0
\]
をみたすので、$a/s=0/1$です。

僕たちは零環、すなわち$1=0$となる環を認めているので、このことに特に問題はないです。
一方$1\neq0$を満たすもののみを環とする主義では、積閉集合の定義に$0\neq S$を追加する必要があります。


\subsection{自然な準同型}

剰余環のときと同様に、局所化の元となった環$A$からの、「自然」と呼ぶことができる準同型が存在します。
\begin{theorem}
  $A$を環、$S$をその積閉集合とする。
  このとき、写像
  \[
    \varphi_S:A\to S^{-1}A;a\mapsto a/1
  \]
  は準同型である。
\end{theorem}
これはもはや自明と言ってよいでしょう。
それよりも、この準同型の性質を見たいです。
もし単射であれば、$A$は$S^{-1}A$の部分環とみなすことができ、$\mathbb{Z}\subset\mathbb{Q}$の直感とも合います。
しかしゼロ因子の存在により、状況は少し複雑になっています。
\begin{theorem}
  $A$を環、$S$をその積閉集合とする。
  このとき、自然な準同型
  \[
    \varphi_S:A\to S^{-1}A;a\mapsto a/1
  \]
  のカーネルは
  \[
    \operatorname{Ker}(\varphi_S)=\{a\in A\mid {}^\exists s\in S \text{ s.t. } as=0\}
  \]
  である。
\end{theorem}
\begin{proof}
  \begin{align*}
    a\in\operatorname{Ker}(\varphi_S)&\iff\varphi(a)=\frac{a}{1}=\frac01\\
    &\iff{}^\exists s \text{ s.t. } s(a\cdot1-0\cdot1)=0\\
    &\iff{}^\exists s \text{ s.t. } as=0
  \end{align*}
\end{proof}

従って、$S$がゼロ因子をもっていると単射ではないのです。
なのでそもそも$A$がゼロ因子を持たない場合、つまり整域の場合、どんな積閉集合$S$をとっても自然に$A\subset S^{-1}A$とみなすことができます。

\subsection{$\mathbb{Z}$から$\mathbb{Q}$、あるいは整域の商体}

$\mathbb{Z}$の部分集合
\[
  S:=\mathbb{Z}\setminus\{0\}
\]
は明らかに積閉集合です。
この分数環$S^{-1}\mathbb{Z}$が有理数体になっているというのが、小学校から学んできた普通の事実です。
\[
  S^{-1}\mathbb{Z}=\mathbb{Q}
\]

一般の整域$A$に対しても、同様のことが言えます。
\begin{theorem}
  $A$を整域、$S:=A\setminus\{0\}$とする。
  このとき、$S$は積閉集合である。
\end{theorem}
\begin{proof}
  もし$s,t\in S$かつ$st=0$を満たすものがあるとすると、$A$は整域であったから、$s=0$または$t=0$を満たす。
  ところがこれは、$s,t\in S=A\setminus\{0\}$であったことに矛盾する。
\end{proof}

また、$A$が整域で$S:=A\setminus\{0\}$のとき、
\[
  K:=S^{-1}A
\]
は明らかに体です(急に明らかと言われてびっくりするかもですが、手を動かし始めると本当に示すべきことがないです)。
このようにして得られる体には名前がついています。
\begin{definition}
  $A$を整域、$S:=A\setminus\{0\}$とする。
  このとき、体
  \[
    S^{-1}A
  \]
  を、整域$A$の\textbf{商体}という。
\end{definition}

\subsection{局所化$A_\mathfrak{p}$}

素イデアルが作る積閉集合、およびその分数環は代数幾何学では重要です。

\begin{theorem}
  $A$を環、$\mathfrak{p}$をその素イデアルとする。
  このとき、$S=A\setminus\mathfrak{p}$は$A$の積閉集合である。
\end{theorem}
\begin{proof}
  $1\in\mathfrak{p}$とすると$\mathfrak{p}=A$となっておかしいため、$1\notin\mathfrak{p}$。すなわち$1\in S=A\setminus\mathfrak{p}$。

  $x,y\in S$ならば$xy\in S$であることは、対偶を取れば「$xy\in \mathfrak{p}$ならば$x \in \mathfrak{p} \lor y \in \mathfrak{p}$」であり、これは$\mathfrak{p}$が素イデアルであることの定義に他ならない。
\end{proof}

\begin{definition}[局所化]
  $A$を環、$\mathfrak{p}$をその素イデアルとする。
  積閉集合$S:=A\setminus\mathfrak{p}$による$A$の分数環を$A_\mathfrak{p}$と書き、$A$の素イデアル$\mathfrak{p}$による\textbf{局所化}(\textbf{localization})という。
\end{definition}

「局所化」というネーミングの意味合いを説明するには、幾何学的な観点を導入せざるを得ないでしょう。
代数幾何学では、環$A$は多様体全体で「正則」な関数全体のなす環をイメージして構成されます。
一方で局所環は、多様体(variety)上の1点、ないし、部分多様体の近傍において正則な関数全体のなす環をイメージしています。

例えば$k$を代数閉体、$P:=(a,b)$を$k^2$の点とします。
$k^2$に対応する環は、$A=k[x,y]$です。
このとき、$P$の近傍では、$(x,y)=(a,b)$で0にならない正則関数で割っても正則であるはずです。
つまり、$k^2$の点$P$の近傍で正則な関数は一般的に、
\[
  \frac{f}{g},\quad\text{ただし$g$は$(x-a)$でも$(y-b)$でも割り切れない}
\]
と書けてしかるべきです。
一方で点$P$に対応する素イデアルは$\mathfrak{p}=(x-a,y-b)$ですから、$\mathfrak{p}$による$A=k[x,y]$の局所化は上記の環に一致します。
これが局所化のイメージです。
$A$が整域だったら、$A$は自然に$A_\mathfrak{p}=S^{-1}A$の部分環とみなせるのでしたから、大方において局所化すると環が大きくなります。

多様体(manifold)論から、ほとんど同じことではあるのですが、違った見方を紹介します。
ある点$P\in M$の近傍で定義される関数は、ようは点$P$で発散したりしてなければよいのです。
そこで、点$P$の近傍で定義された関数全体のなす環は
\[
  \mathcal{O}_{M,P}:=\{\left<U,f\right>\mid \text{$U\subset M$: 開集合}, P\in U, f:U\to\mathbb{C}\}
\]
と書いてもよいことがわかります。
ここで、各元は
\begin{align*}
  \left<U,f\right>=\left<V,g\right> \iff f|_W=g|_W,\quad\exists W\subset U\cap V\text{:$P$の開近傍}
\end{align*}
という同値関係で同一視されています。
環$\mathcal{O}_{M,P}$は複雑かもしれませんが、次のような意味できわめて単純です。
\begin{theorem}
  $M$を多様体、$P\in M$とする。
  環$\mathcal{O}_{M,P}$の極大イデアルは
  \[
    \mathfrak{m}_P:=\{\left<U,f\right>\mid f(P)=0\}
  \]
  の一つのみである。
\end{theorem}
多様体の定義はろくにやってないので、証明は適当に述べます。
まず$\mathfrak{m}_P$が極大であることは次のようにわかります。
$\mathfrak{m}_P\subsetneq\mathfrak{a}\subset\mathcal{O}_{M,P}$とすると、$\mathfrak{a}$には$P\in M$で0にならない関数$f$が入ってくるはずです。
$f$は連続関数なので、$P$の十分小さな近傍$U$で常に0になりません。
そしたら$\left<U,1/f\right>\in\mathcal{O}_{M,P}$であり、
\[
  1=\left<U,1/f\right>\cdot\left<U,f\right>\in\mathfrak{a}
\]
ゆえに$\mathfrak{a}=\mathcal{O}_{M,P}$がわかり、$\mathfrak{m}_P$が極大イデアルであることがわかります。
またこのように、$\mathcal{O}_{M,P}\setminus\mathfrak{m}_P$の元はすべて単元なので、$\mathcal{O}_{M,P}$でない任意のイデアルは$\mathfrak{m}_P$に含まれなければいけません。

この性質を環の言葉に一般化して、次のように定義します。
\begin{definition}
  環$A$が\textbf{局所環}(\textit{local ring})であるとは、$A$が唯一の極大イデアルをもつときをいう。
\end{definition}

環の局所化は、実際に局所環になっています。
\begin{theorem}
  環$A$の素イデアル$\mathfrak{p}$による局所化$A_\mathfrak{p}$は局所環である。
\end{theorem}
\begin{proof}
  \[
    \mathfrak{m}:=\left\{\frac{a}{s}\in A_\mathfrak{p}\mid a\in\mathfrak{p}, s\in A\setminus\mathfrak{p}\right\}
  \]
  とおく。このとき、$\mathfrak{m}$が$A_\mathfrak{p}$の唯一の極大イデアルであることを示す。

  \begin{itemize}
    \item イデアルであること\\
    $0\in\mathfrak{p}$より$0=0/1\in\mathfrak{m}$は自明。
    $a/s,b/t\in\mathfrak{m}$とおくと、すなわち$a,b\in\mathfrak{p}$より$at+bs\in\mathfrak{p}$。
    ゆえに$a/s+b/t=(at+bs)/st\in\mathfrak{m}$。
    最後に、$a/s\in A_\mathfrak{p}$かつ$x/t\in\mathfrak{m}$とすると、$x\in\mathfrak{p}$ゆえ$a/s\cdot x/t=ax/st\in\mathfrak{m}$。
    以上より$\mathfrak{m}$はイデアルである。
    \item $A_\mathfrak{p}\setminus\mathfrak{m}$が単元のみからなること、従って$\mathfrak{m}$が唯一の極大イデアルであること\\
    $a/s\in A_\mathfrak{p}\setminus\mathfrak{m}$とする。
    このとき、$\mathfrak{m}$の定義から$a\notin\mathfrak{p}$ゆえ、$a\in S$。
    従って$s/a\in A_\mathfrak{p}$。
  \end{itemize}
\end{proof}

\begin{comment}
\section{有限生成$A$-代数}

(環準同型による代数の定義。有限生成代数と有限生成加群の違いを強調)

\section{ヒルベルトの零点定理}

準備は整いました。ここから舞台を代数的閉体$K$上の多項式環$A = K[x_1, \dots, x_n]$に移します。

\subsection{代数的集合とイデアル}

($V(\mathfrak{a})$ と $I(Z)$ の定義)

ここで素朴な疑問が生じます。
イデアル$\mathfrak{a}$から出発して、その零点集合$V(\mathfrak{a})$を考え、さらにその消失イデアル$I(V(\mathfrak{a}))$を考えたとき、元の$\mathfrak{a}$に戻るでしょうか?
答えはNoです。先ほど見たように、$\mathfrak{a}=(x^2)$なら$I(V(\mathfrak{a}))=(x)$になってしまいます。
では、$\sqrt{\mathfrak{a}}$をとれば一致するのか?
その答えが、次の定理です。

\subsection{弱零点定理}

まずは「点」に対応する極大イデアルの形を決定します。

\begin{theorem}[Hilbert's Nullstellensatz, weak form]
  $K$を代数的閉体とする。$K[x_1, \dots, x_n]$の任意の極大イデアル$\mathfrak{m}$は、
  \[
    \mathfrak{m} = (x_1 - a_1, \dots, x_n - a_n), \quad (a_1, \dots, a_n \in K)
  \]
  の形に書ける。
\end{theorem}

この証明には、次の代数的な補題が強力な役割を果たします。
(ザリスキの補題を紹介)

\subsection{強零点定理とRabinowitschのトリック}

変数を一つ増やすことで、魔法のように弱形式から一般系を導くことができます。これをRabinowitschのトリックと呼びます。

\begin{theorem}[Hilbert's Nullstellensatz, strong form]
  $K$を代数的閉体、$A=K[x_1, \dots, x_n]$とする。
  任意のイデアル$\mathfrak{a} \subset A$に対して、
  \[
    I(V(\mathfrak{a})) = \sqrt{\mathfrak{a}}
  \]
\end{theorem}
\end{comment}
  \printbibliography
\end{document}
